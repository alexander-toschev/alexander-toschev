\chapter*{Список сокращений и условных обозначений}             % Заголовок
\addcontentsline{toc}{chapter}{Список сокращений и условных обозначений}  % Добавляем его в оглавление

\textbf{selectLinkedObject(obj:Resource, linkName:String): Link<Resource>}~--- Описание метода. selectLinkedObject~--- название метода. (obj:Resource, linkName:String)~--- параметры метода. linkName~--- имя параметра. String тип данных. Link<Resource>~--- тип возвращаемых данных. Если метод данных не возвращает, то ничего не указывается.\\

\textbf{TU}~--- Сокращение от ThinkingUnderstanding.\\

\textbf{TLC}~--- Thinking Life Cycle.\\

\textbf{НДФЛ}~--- Налог на доходы физически лиц.\\

\textbf{ПО}~--- Программное обеспечение.\\

\textbf{ФБ}~--- Федеральный бюджет.\\

\textbf{ПФР}~--- Пенсионный фонд России.\\

\textbf{ТФОМС}~--- Территориальный фонд обязательного медицинского страхования.\\

\textbf{ФФОМС}~--- Федеральный фонд обязательного медицинского страхования.\\

\textbf{ФСС}~--- Фонд социального страхования.\\

\textbf{БД}~--- База данных.\\

\textbf{мс.}~--- Миллисекунды.\\


\clearpage
