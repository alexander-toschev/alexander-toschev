\chapter*{ВВЕДЕНИЕ}							% Заголовок
\addcontentsline{toc}{chapter}{ВВЕДЕНИЕ}	% Добавляем его в оглавление

\newcommand{\actuality}{}
\newcommand{\aim}{\textbf{Целью}}
\newcommand{\tasks}{задачи}
\newcommand{\defpositions}{\textbf{Основные положения, выносимые на~защиту:}}
\newcommand{\novelty}{\textbf{Научная новизна:}}
\newcommand{\influence}{\textbf{Научная и практическая значимость}}
\newcommand{\reliability}{\textbf{Степень достоверности}}
\newcommand{\probation}{\textbf{Апробация работы.}}
\newcommand{\contribution}{\textbf{Личный вклад.}}
\newcommand{\publications}{\textbf{Публикации.}}
{\actuality}
В настоящее время в области IT набрали большую популярность системы удаленной поддержки информационной инфраструктуры предприятия, так называемый «Аутсорсинг». Ввиду развития рынка компаниям становится невыгодно держать свой штат службы поддержки, и они отдают свою информационную инфраструктуру сторонней компании.
Ввиду возросшей популярности данного бизнеса и появлением большого количества игроков на рынке возникла большая конкурентность, которая потребовала увеличения эффективности и сокращения издержек, что в свою очередь привело к необходимости системного анализа области и выработке решению сложившихся проблем. В контексте решения этой проблемы рассматривается модель области и модель системы, которая увеличивает эффективность работы путем частичной (в некоторых случаях полной) автоматизации обработки инцидентов, начиная с разбора входящих инцидентов на естественном языке и заканчивая применением найденного решения. 
Главными требованиями к подобной системе являются:
\begin{enumerate}
  \item Обработка запросов на естественном языке
  \item Возможность обучения
  \item Общение с человеческим специалистом
  \item Проведение логических рассуждений: аналогия, дедукция, индукция
  \item Умение абстрагировать решение одной проблемы и, экстраполируя его, применить для других решений
  \item Способность ответить на запрос пользователя
\end{enumerate}

На данный момент многие компании ведут разработку подобных систем. Примером является набирающая популярность IBM Watson. Подобный класс систем также называется вопросно-ответными системами. \\
В данной работе была создана подобная система на основе исследования целевой области и построения ее модели.
 Акцент был сделан на создании мыслящей системы для решения широкого круга проблем. \\
{\aim} данной работы является исследование целевой области, создание ее модели, выработка проблем области, оценка подходов к решению проблем,  создание архитектуры и реализация базового прототипа программного комплекса обеспечивающего разбор и формализацию входного запроса пользователя и поиск решения данной проблемы.

Для~достижения поставленной цели необходимо было решить следующие {\tasks}:
\begin{enumerate}
  \item Провести теоретико-множественный и теоретико-информационный анализ сложных систем в области поддержки информационной инфраструктуры
  \item Вычислить технико-экономическую возможность автоматизации целевой области
  \item Создать модель целевой области
  \item Исследовать модели мышления и выбрать наиболее подходящую
  \item На основе выбранной модели мышления разработать модель проблемно-ориентированной системы управления, принятия решений и оптимизации технических объектов в области обслуживания информационной структуры предприятия
  \item Создать архитектуру приложения на основе модели
  \item Реализовать прототип на основе архитектуры
  \item Провести апробацию прототипа на тестовых данных
\end{enumerate}

\defpositions
\begin{enumerate}
  \item Теоретико-множественный и теоретико-информационный анализ сложных систем в области поддержки информационной инфраструктуры
  \item Модель проблемно-ориентированной системы управления, принятия решений и оптимизации технических объектов в области обслуживания IT, ее технико-эконономическое обоснование  
  \item Прототип программной реализации модели проблемно-ориентированной системы управления, принятия решений и оптимизации технических объектов в области обслуживания IT  
  \item Апробация системы на контрольных примерах и ее результаты
\end{enumerate}

\novelty
\begin{enumerate}
  \item Была создана модель проблемно-ориентированной системы управления, принятия решений и оптимизации технических объектов в области обслуживания информационной структуры предприятия на основе модели мышления
  \item Доказана применимость модели для других областей
  \item Была представлена новая модель данных для модели мышления и оригинальный способ ее хранения, обеспечивающий быстрый доступ
  \item Было выполнено оригинальное исследование моделей мышления в области обслуживания информационной структуры предприятия
  \item На основе модели была создана архитектура системы и ее прототип 
\end{enumerate}

\influence\ 
Система, разрабатываемая в рамках данной работы носит значимый практический характер. Идея работы зародилась из производственных проблем в IT отрасли, с которыми автор сталкивался каждый день. Только глубокое понимание проблем помогло выбрать правильное решение. Более подробное описание представлено в Главе 1.
\reliability\ полученных результатов обеспечивается результатами выполнения тестов на контрольных примерах. Результаты находятся в соответствии с результатами, полученными другими авторами, экспертными системами и специалистами. \\ 

\probation\
Основные результаты работы докладывались на:
\begin{itemize}
	\item RCDL-2014
	\item AINL-2013
	\item WCIT-2012
	\item AMSTA-2015
\end{itemize}

\contribution\ Автор принимал активное участие в исследовании целевой области, разработке архитектуры приложения, реализации прототипа, проработки теории, тестировании прототипа.

\publications\ Основные результаты по теме диссертации изложены в 6 печатных изданиях  \cite{Lobachevskii},\cite{WCIT-2012},\cite{RCDL-2014},\cite{AINL-2013},\cite{ISGZ},\cite{AMSTA-2015}, 
2 из которых изданы в журналах Scopus, 1 в журнале РИНЦ  \cite{ISGZ}, 
4 в тезисах докладов \cite{Lobachevskii},\cite{WCIT-2012},\cite{AINL-2013},\cite{ISGZ}, \cite{IJSE-1}.



 % Характеристика работы по структуре во введении и в автореферате не отличается (ГОСТ Р 7.0.11, пункты 5.3.1 и 9.2.1), потому её загружаем из одного и того же внешнего файла, предварительно задав форму выделения некоторым параметрам
%% регистрируем счётчики в системе totcounter
\regtotcounter{totalcount@figure}
\regtotcounter{totalcount@table}       % Если поставить в преамбуле то ошибка в числе таблиц
\regtotcounter{TotPages}               % Если поставить в преамбуле то ошибка в числе страниц

\textbf{Объем и структура работы.} Диссертация состоит из введения, четырех глав, заключения и пяти приложений. Полный объём диссертации составляет \formbytotal{TotPages}{страниц}{у}{ы}{} 
с~\formbytotal{totalcount@figure}{рисунк}{ом}{ами}{ами}
и~\formbytotal{totalcount@table}{таблиц}{ей}{ами}{ами}. Список литературы содержит  
\formbytotal{citenum}{наименован}{ие}{ия}{ий}.
\clearpage



  \begin{longtable}{| p{5cm} |p{10cm} |}
 \caption[Словарь терминов]{Словарь терминов}\label{Glossary} \\ 
 \hline
 
 \multicolumn{1}{|c|}{\textbf{Термин}} & \multicolumn{1}{c|}{\textbf{Значение}}  \\ \hline 
\endfirsthead
\multicolumn{2}{c}%
{{\bfseries \tablename\ \thetable{} -- продолжение}} \\
\hline \multicolumn{1}{|c|}{\textbf{Термин}} &
\multicolumn{1}{c|}{\textbf{Значение}}  \\ \hline 
\endhead

\hline \multicolumn{2}{|r|}{{Продолжение следует}} \\ \hline
\endfoot

\hline \hline
\endlastfoot
\hline
База Знаний	& База данных приложения, представленная в виде онтологии знаний \\
 \hline
WayToThink	& Путь мышления. Основан на определении Марвина Мински \cite{EmotionMachine}. Класс объектов, которые модифицируют данные \\
 \hline
Critic	& Критик. Основан на определении Марвина Мински \cite{EmotionMachine}. Класс объектов, которые выступают триггерами при наступление определенного события \\
 \hline
ThinkingLifeCycle	& TLC. Основан на определении Марвина Мински \cite{EmotionMachine}. Класс объектов, которые выступают основными объектами для запуска в приложении - рабочими процессами \\
 \hline
Selector	& Компонент, отвечающий за выборку данных из Базы Знаний \\
\hline
Instinctive	& Инстинктивный уровень \\
\hline
Learned	& Уровень обученных реакций \\
\hline
Deliberative	& Уровень рассуждений \\
\hline
Reflective	& Рефлексивный уровень \\
\hline
Self-Reflective Thinking	 & Саморефлексивный уровень \\
\hline
Self-Conscious Reflection	& Самосознательный уровень \\
\hline
ThinkingUnderstanding	& Система, созданная в рамках работы. Дословный перевод "Мышление-Понимание".  \\
\hline
TU	& Сокращение от ThinkingUnderstanding.  \\
\hline
\end{longtable}

\clearpage

\begin{longtable}{|p{7cm}|p{8cm}|}
 \caption[Принятые аннотации для изложения]{Принятые аннотации для изложения}\label{AnnotationsList} \\ 
 \hline
 
 \multicolumn{1}{|c|}{\textbf{Аннотация}} & \multicolumn{1}{c|}{\textbf{Описание}}  \\ \hline 
\endfirsthead
\multicolumn{2}{c}%
{{\bfseries \tablename\ \thetable{} -- продолжение}} \\
\hline \multicolumn{1}{|c|}{\textbf{Аннотация}} &
\multicolumn{1}{c|}{\textbf{Описание}}  \\ \hline 
\endhead

\hline \multicolumn{2}{|r|}{{Продолжение следует}} \\ \hline
\endfoot

\hline \hline
\endlastfoot
\hline
   selectLinkedObject(obj:Resource, linkName:String): Link<Resource>  & Описание метода. selectLinkedObject - название метода. (obj:Resource, linkName:String) - параметры метода. linkName - имя параметра. String тип данных. Link<Resource> - тип возвращаемых данных. Если метод данных не возвращает, то ничего не указывается.\\
   \hline
    \end{longtable}
\clearpage