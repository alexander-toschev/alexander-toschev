\chapter*{ВВЕДЕНИЕ}							% Заголовок
\addcontentsline{toc}{chapter}{ВВЕДЕНИЕ}	% Добавляем его в оглавление
В настоящее время в области IT набрало большую популярность системы удаленной поддержки информационной инфраструктуры, так называемый «Аутсорсинг». Ввиду развития рынка компаниям становится невыгодно держать свой штат службы поддержки, и они отдают свою инфраструктуру сторонней компании.
Ввиду возросшей интенсивности данного бизнеса возникла потребность автоматизации работы. В данном контексте рассматривается автоматизация обработки инцидентов, начиная с разбора инцидентов на естественном языке и заканчивая поиском решения и применением решения.
Главными требованиями к системе являются
\begin{enumerate}
  \item Обработка естественного языка
  \item Возможность обучения
  \item Общение с специалистом
  \item Проведение логических рассуждений: аналогия, дедукция, индукция
  \item Умения абстрагировать решение и экстраполировать его на другие решения
\end{enumerate}
На данный момент многие компании ведут разработку подобных систем. Примером такой системы является набирающая популярность система IBM Watson \cite{Watson}. Подобный класс система также называют вопросно-ответными системами. Другим примером является система Wolfram Alpha \cite{WolframAplha}.
В данной работе был сделан акцент на попытку создания мыслящий системы на основе модели мышления Марвина Мински \cite{EmotionMachine}.

\textbf{Целью} данной работы является создание архитектуры и реализация базового прототипа программного комплекса обеспечивающего разбор и формализацию входного запроса пользователя и поиск решения данной проблемы.

Для~достижения поставленной цели необходимо было решить следующие задачи:
\begin{enumerate}
  \item Провести теоретико-множественный и теоретико-информационный анализ сложных систем в области поддержки информационной инфраструктуры
  \item Вычислить возможность автоматизации целевой области
  \item Создать модель целевой области
  \item Исследовать модель мышления Марвина Мински
  \item На основе модели мышления Мински разработать модель проблемно-ориентированной системы управления, принятия решений и оптимизации технических объектов в области обслуживания IT  
  \item \item Создать архитектуру приложения на основе модели
  \item Реализовать прототип на основе архитектуры
  \item Провести апробацию прототипа на тестовых данных
\end{enumerate}

\textbf{Основные положения, выносимые на~защиту:}
\begin{enumerate}
  \item Теоретико-множественный и теоретико-информационный анализ сложных систем в области поддержки информационной инфраструктуры
  \item Модель проблемно-ориентированной системы управления, принятия решений и оптимизации технических объектов в области обслуживания IT  
  \item Прототип программной реализации модели проблемно-ориентированной системы управления, принятия решений и оптимизации технических объектов в области обслуживания IT  
  \item Экстраполяция программной системы для других областей
\end{enumerate}

\textbf{Научная новизна:}
\begin{enumerate}
  \item Была создана модель проблемно-ориентированной системы управления, принятия решений и оптимизации технических объектов в области обслуживания IT на основе модели мышления Марвина Мински
  \item Была представлена новая модель данных для модели мышления и оригинальный способ хранения 
  \item Было выполнено оригинальное исследование модели мышления 
\end{enumerate}

\textbf{Научная и практическая значимость} 

\textbf{Степень достоверности} полученных результатов обеспечивается результатами выполнения тестов на контрольных примерах. Результаты находятся в соответствии с результатами, полученными другими авторами и экспертными системами. \\
Практическая значимость обеспечивается потребностью автоматизации целевой области. Более подробное описание представлено в Главе \ref{chapt1}.

\textbf{Апробация работы}
Основные результаты работы докладывались~на:
\begin{itemize}
	\item RCDL-2014
	\item AINL-2013
	\item WCIT-2012
	\item AMSTA-2015
\end{itemize}


\textbf{Словарь терминов}
\begin{table} [htbp]
   \centering
   \parbox{15cm}{\caption{Глоссарий}\label{Glossary}}
%  \begin{center}
  \begin{tabular}{| p{5cm} ||p{5cm}|| p{5cm} |}
  \hline
  \hline
Термин & Значения \\
  \hline
  \hline
База Знаний	& База данных приложения, представленная в виде онтологии знаний \\
 \hline
WayToThink	& Путь мышления. Основан на определении Марвина Мински \cite{EmotionMachine}. Класс объектов, которые модифицируют данные \\
 \hline
Critic	& Критик. Основан на определении Марвина Мински \cite{EmotionMachine}. Класс объектов, которые выступают триггерами при наступление определенного события \\
 \hline
ThinkingLifeCycle	& TLC. Основан на определении Марвина Мински \cite{EmotionMachine}. Класс объектов, которые выступают основными объектами для запуска в приложении - рабочими процессами \\

 \hline
  \hline
\end{tabular}
%  \end{center}
\end{table}

\textbf{Личный вклад.} Автор принимал активное участие в разработке архитектуры приложения, реализации прототипа, проработки теории, тестировании.

\textbf{Публикации.} Основные результаты по теме диссертации изложены в 6 печатных изданиях  \cite{Lobachevskii},\cite{WCIT-2012},\cite{RCDL-2014},\cite{AINL-2013},\cite{ISGZ},\cite{AMSTA-2015}, 
2 из которых изданы в журналах Scopus 
\cite{RCDL-2014}, \cite{AMSTA-2015} 
6 в тезисах докладов \cite{Lobachevskii},\cite{WCIT-2012},\cite{RCDL-2014},\cite{AINL-2013},\cite{ISGZ},\cite{AMSTA-2015}.

\textbf{Объем и структура работы.} Диссертация состоит из введения, шести глав, заключения и пяти приложений. Полный объем диссертации составляет 80 страниц с 40 рисунками и 5 таблицами. Список литературы содержит 21 наименований.


\clearpage