\chapter*{Введение}							% Заголовок
\addcontentsline{toc}{chapter}{Введение}	% Добавляем его в оглавление
В настоящее время в области IT набрало большую популярность системы удаленной поддержки информационной инфраструктуры, так называемый «Аутсорсинг». Ввиду развития рынка компаниям становится невыгодно держать свой штат службы поддержки, и они отдают свою инфраструктуру сторонней компании.
Ввиду возросшей интенсивности данного бизнеса возникла потребность автоматизации работы. В данном контексте рассматривается автоматизация обработки инцидентов, начиная с разбора инцидентов на естественном языке и заканчивая поиском решения и применением решения.
Главными требованиями к системе являются
\begin{enumerate}
  \item Обработка естественного языка
  \item Возможность обучения
  \item Общение с специалистом
  \item Проведение логических рассуждений: аналогия, дедукция, индукция
  \item Умения абстрагировать решение и экстраполировать его на другие решения
\end{enumerate}
На данный момент многие компании ведут разработку подобных систем. Примером такой системы является набирающая популярность система IBM Watson \cite{Watson}. Подобный класс система также называют вопросно-ответными системами. Другим примером является система Wolfram Alpha \cite{WolframAplha}.
В данной работе был сделан акцент на попытку создания мыслящий системы на основе модели мышления Марвина Мински \cite{EmotionMachine}.

\textbf{Целью} данной работы является создание архитектуры и реализация базового прототипа программного комплекса обеспечивающего разбор и формализацию входного запроса пользователя и поиск решения данной проблемы.

Для~достижения поставленной цели необходимо было решить следующие задачи:
\begin{enumerate}
  \item Исследовать целевую область
  \item Вычислить возможность автоматизации целевой области
  \item Исследовать модель мышления Марвина Мински
  \item Адаптировать модель для прикладной реализации
  \item Создать архитектуру приложения на основе модели
  \item Реализовать прототип на основе архитектуры
\end{enumerate}

\textbf{Основные положения, выносимые на~защиту:}
\begin{enumerate}
  \item Возможность автоматизации области предоставления удаленной поддержки информационной инфраструктуры 
  \item Прикладное применение модели мышления Марвина Мински для решения задачи автоматизации
  \item Возможность программной реализации модели мышления Марвина Мински
  \item Экстраполяция программной системы для других областей
\end{enumerate}

\textbf{Научная новизна:}
\begin{enumerate}
  \item Впервые была представлена реализация модели мышления Мински на практике
  \item Была представлена новая модель данных для модели мышления 
  \item Было выполнено оригинальное исследование модели мышления \ldots
\end{enumerate}

\textbf{Научная и практическая значимость} \ldots

\textbf{Степень достоверности} полученных результатов обеспечивается результатами выполнения тестов на контрольных примерах. Результаты находятся в соответствии с результатами, полученными другими авторами и экспертными системами

\textbf{Апробация работы}
Основные результаты работы докладывались~на:
\begin{itemize}
	\item RCDL-2014
	\item AINL-2013
	\item WCIT-2012
	\item AMSTA-2015
\end{itemize}


\textbf{Словарь терминов}
\begin{table} [htbp]
   \centering
   \parbox{15cm}{\caption{Глоссарий}\label{Glossary}}
%  \begin{center}
  \begin{tabular}{| p{5cm} ||p{5cm}|| p{5cm} |}
  \hline
  \hline
Термин & Значения \\
  \hline
  \hline
База Знаний	& База данных приложения, представленная в виде онтологии знаний \\
 \hline
WayToThink	& Путь мышления. Основан на определении Марвина Мински \cite{EmotionMachine}. Класс объектов, которые модифицируют данные \\
 \hline
Critic	& Критик. Основан на определении Марвина Мински \cite{EmotionMachine}. Класс объектов, которые выступают триггерами при наступление определенного события \\

 \hline
  \hline
\end{tabular}
%  \end{center}
\end{table}

\textbf{Личный вклад.} Автор принимал активное участие в разработке архитектуры приложения, реализации прототипа, проработки теории, тестировании.

\textbf{Публикации.} Основные результаты по теме диссертации изложены в ХХ печатных изданиях~\cite{Sychev,Sokolov,Gaidaenko,Lermontov,Management},
Х из которых изданы в журналах, рекомендованных ВАК~\cite{Sychev,Sokolov,Gaidaenko}, 
ХХ --- в тезисах докладов~\cite{Lermontov,Management}.

\textbf{Объем и структура работы.} Диссертация состоит из~введения, четырех глав, заключения и~двух приложений. Полный объем диссертации составляет ХХХ~страница с~ХХ~рисунками и~ХХ~таблицами. Список литературы содержит ХХХ~наименований.


\clearpage