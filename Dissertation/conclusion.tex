\chapter*{ЗАКЛЮЧЕНИЕ}						% Заголовок
\addcontentsline{toc}{chapter}{ЗАКЛЮЧЕНИЕ}	% Добавляем его в оглавление

%% Согласно ГОСТ Р 7.0.11-2011:
%% 5.3.3 В заключении диссертации излагают итоги выполненного исследования, рекомендации, перспективы дальнейшей разработки темы.
%% 9.2.3 В заключении автореферата диссертации излагают итоги данного исследования, рекомендации и перспективы дальнейшей разработки темы.
%% Поэтому имеет смысл сделать эту часть общей и загрузить из одного файла в автореферат и в диссертацию:

%% Согласно ГОСТ Р 7.0.11-2011:
%% 5.3.3 В заключении диссертации излагают итоги выполненного исследования, рекомендации, перспективы дальнейшей разработки темы.
%% 9.2.3 В заключении автореферата диссертации излагают итоги данного исследования, рекомендации и перспективы дальнейшей разработки темы.

Решены следующие задачи и достигнуты следующие результаты.
\begin{enumerate}
  \item Создана модель проблемно-ориентированной системы управления, принятия решений в области обслуживания информационной структуры предприятия на основе модели мышления;
  \item Представлены новая модель данных для модели мышления и оригинальный способ ее хранения, эффективный по сравнению с другими базами данных;
  \item Выполнено оригинальное исследование моделей мышления в области обслуживания информационной структуры предприятия;
  \item На основе модели созданы архитектура системы и ее прототип; 
  \item Созданы специальные алгоритмы для анализа запросов пользователей и принятия решений;
  \item Система, разработанная в рамках данной работы, включает в себя инновационные методы и алгоритмы поддержки принятия решений, использует модель мышления на базе модели мышления Мински;
  \item Представлена наглядная визуализация структуры области удаленной поддержки инфраструктуры.
\end{enumerate}

Представленные в диссертации модель мышления, ее архитектура и реализация являются уникальными~--- на данный момент времени это единственная реализация модели мышления Мински. \par
Система, разработанная в диссертации, не является узкоспециализированной. Она также подходит для других областей, где требуется поддержка принятия решений. Например, при постановке медицинского диагноза, чтобы отбросить ложные диагнозы. \par
Кроме того, в систему можно загрузить данные о взаимосвязи органов человека и болезней. Далее, к каждому заболеванию добавить симптом и способ лечения, после этого можно делать запрос с симптомами, и система выдаст список вероятных заболеваний со способами их лечения. \par
В области диагностики проблем можно обучить систему узлам автомобиля, проблемам, с ними связанными, признаками этих проблем и способами их устранения. 





Работа велась с использованием открытых технологий, без использования проприетарного программного обеспечения. Работа была презентована автору книги Object-Oriented Software Construction \cite{Meyer} Бертрану Мейеру в рамках серии лекций, проведенных при содействии Университета Иннополис в Казани в 2015 году в рамках AKSES-2015 http://university.innopolis.ru/en/research/selab/events/akses и была им отмечена.  
Работа выполнялась при помощи компании ОАО "АйСиЭл КПО ВС", в рамках работы использовались и обрабатывались данные, собранные во время работы команд ICL над поддержкой информационной структуры предприятий-заказчиков.