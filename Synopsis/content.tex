\subsection*{Общая характеристика работы}

\newcommand{\actuality}{\underline{\textbf{Актуальность темы.}}}
\newcommand{\aim}{\underline{\textbf{Целью}}}
\newcommand{\tasks}{\underline{\textbf{задачи}}}
\newcommand{\defpositions}{\underline{\textbf{Основные положения, выносимые на~защиту:}}}
\newcommand{\novelty}{\underline{\textbf{Научная новизна:}}}
\newcommand{\influence}{\underline{\textbf{Практическая значимость}}}
\newcommand{\reliability}{\underline{\textbf{Достоверность}}}
\newcommand{\probation}{\underline{\textbf{Апробация работы.}}}
\newcommand{\contribution}{\underline{\textbf{Личный вклад.}}}
\newcommand{\publications}{\underline{\textbf{Публикации.}}}

{\actuality}
В настоящее время в области IT набрали большую популярность системы удаленной поддержки информационной инфраструктуры предприятия, так называемый «Аутсорсинг». Ввиду развития рынка компаниям становится невыгодно держать свой штат службы поддержки, и они отдают свою информационную инфраструктуру сторонней компании.
Ввиду возросшей популярности данного бизнеса и появлением большого количества игроков на рынке возникла большая конкурентность, которая потребовала увеличения эффективности и сокращения издержек, что в свою очередь привело к необходимости системного анализа области и выработке решению сложившихся проблем. В контексте решения этой проблемы рассматривается модель области и модель системы, которая увеличивает эффективность работы путем частичной (в некоторых случаях полной) автоматизации обработки инцидентов, начиная с разбора входящих инцидентов на естественном языке и заканчивая применением найденного решения. 
Главными требованиями к подобной системе являются:
\begin{enumerate}
  \item Обработка запросов на естественном языке
  \item Возможность обучения
  \item Общение с человеческим специалистом
  \item Проведение логических рассуждений: аналогия, дедукция, индукция
  \item Умение абстрагировать решение одной проблемы и, экстраполируя его, применить для других решений
  \item Способность ответить на запрос пользователя
\end{enumerate}

На данный момент многие компании ведут разработку подобных систем. Примером является набирающая популярность IBM Watson. Подобный класс систем также называется вопросно-ответными системами. \\
В данной работе была создана подобная система на основе исследования целевой области и построения ее модели.
 Акцент был сделан на создании мыслящей системы для решения широкого круга проблем. \\
{\aim} данной работы является исследование целевой области, создание ее модели, выработка проблем области, оценка подходов к решению проблем,  создание архитектуры и реализация базового прототипа программного комплекса обеспечивающего разбор и формализацию входного запроса пользователя и поиск решения данной проблемы.

Для~достижения поставленной цели необходимо было решить следующие {\tasks}:
\begin{enumerate}
  \item Провести теоретико-множественный и теоретико-информационный анализ сложных систем в области поддержки информационной инфраструктуры
  \item Вычислить технико-экономическую возможность автоматизации целевой области
  \item Создать модель целевой области
  \item Исследовать модели мышления и выбрать наиболее подходящую
  \item На основе выбранной модели мышления разработать модель проблемно-ориентированной системы управления, принятия решений и оптимизации технических объектов в области обслуживания информационной структуры предприятия
  \item Создать архитектуру приложения на основе модели
  \item Реализовать прототип на основе архитектуры
  \item Провести апробацию прототипа на тестовых данных
\end{enumerate}

\defpositions
\begin{enumerate}
  \item Теоретико-множественный и теоретико-информационный анализ сложных систем в области поддержки информационной инфраструктуры
  \item Модель проблемно-ориентированной системы управления, принятия решений и оптимизации технических объектов в области обслуживания IT, ее технико-эконономическое обоснование  
  \item Прототип программной реализации модели проблемно-ориентированной системы управления, принятия решений и оптимизации технических объектов в области обслуживания IT  
  \item Апробация системы на контрольных примерах и ее результаты
\end{enumerate}

\novelty
\begin{enumerate}
  \item Была создана модель проблемно-ориентированной системы управления, принятия решений и оптимизации технических объектов в области обслуживания информационной структуры предприятия на основе модели мышления
  \item Доказана применимость модели для других областей
  \item Была представлена новая модель данных для модели мышления и оригинальный способ ее хранения, обеспечивающий быстрый доступ
  \item Было выполнено оригинальное исследование моделей мышления в области обслуживания информационной структуры предприятия
  \item На основе модели была создана архитектура системы и ее прототип 
\end{enumerate}

\influence\ 
Система, разрабатываемая в рамках данной работы носит значимый практический характер. Идея работы зародилась из производственных проблем в IT отрасли, с которыми автор сталкивался каждый день. Только глубокое понимание проблем помогло выбрать правильное решение. Более подробное описание представлено в Главе 1.
\reliability\ полученных результатов обеспечивается результатами выполнения тестов на контрольных примерах. Результаты находятся в соответствии с результатами, полученными другими авторами, экспертными системами и специалистами. \\ 

\probation\
Основные результаты работы докладывались на:
\begin{itemize}
	\item RCDL-2014
	\item AINL-2013
	\item WCIT-2012
	\item AMSTA-2015
\end{itemize}

\contribution\ Автор принимал активное участие в исследовании целевой области, разработке архитектуры приложения, реализации прототипа, проработки теории, тестировании прототипа.

\publications\ Основные результаты по теме диссертации изложены в 6 печатных изданиях  \cite{Lobachevskii},\cite{WCIT-2012},\cite{RCDL-2014},\cite{AINL-2013},\cite{ISGZ},\cite{AMSTA-2015}, 
2 из которых изданы в журналах Scopus, 1 в журнале РИНЦ  \cite{ISGZ}, 
4 в тезисах докладов \cite{Lobachevskii},\cite{WCIT-2012},\cite{AINL-2013},\cite{ISGZ}, \cite{IJSE-1}.



 % Характеристика работы по структуре во введении и в автореферате не отличается (ГОСТ Р 7.0.11, пункты 5.3.1 и 9.2.1), потому её загружаем из одного и того же внешнего файла, предварительно задав форму выделения некоторым параметрам

%Диссертационная работа была выполнена при поддержке грантов ...

%\underline{\textbf{Объем и структура работы.}} Диссертация состоит из~введения, четырех глав, заключения и~приложения. Полный объем диссертации \textbf{ХХХ}~страниц текста с~\textbf{ХХ}~рисунками и~5~таблицами. Список литературы содержит \textbf{ХХX}~наименование.

%\newpage
\subsection*{Содержание работы}
Во \underline{\textbf{введении}} обосновывается актуальность исследований, проводимых в рамках данной диссертационной работы, дается общая характеристика работы.
\underline{\textbf{Первая глава}} посвящена постановки задачи. Проводится обзор целевой области и обосновывается возможность ее автоматизации. В главе обосновывается состав команд поддержки информационной структуры предприятия. На Диаграмме \ref{img:ITSMTeamComposition} представлен качественно процентный состав в команд с точки зрения квалификации специалистов. \\
\begin{figure} [h] 
  \center
  \includegraphics [scale=0.7] {ITSMTeamComposition}
  \caption{Диаграмма состава команд} 
  \label{img:ITSMTeamComposition}  
\end{figure}

В главе приведены резльтатаnы анализа категорий проблем, которые решают специалисты \ref{img:EngineerTasks}. 
\begin{figure} [h] 
  \center
  \includegraphics [scale=0.7] {EngineerTasks}
  \caption{Диаграмма соотношений типов проблем} 
  \label{img:EngineerTasks}  
\end{figure}
\clearpage
\underline{\textbf{Вторая глава}} посвящена исследованию подходов обработки естественного языка в применение к целевой области. В главе выработан набор тестовых данных, разработаны критерии оценки работы подходов обработки естественного языка, представленных в Таблице \ref{Metrics}.
\begin{table} [htbp]
  \centering
  \parbox{15cm}{\caption{Таблица метрик}\label{Metrics}}
%  \begin{center}
  \begin{tabular}{| p{5cm} ||p{5cm}|| p{5cm} |}
  \hline
  \hline
Метрика & Описание & Формула \\
  \hline
  \hline
Аккуратность	& Понимание текста обработчиком & 
$$ 
Ac=\frac{1-x}{y}
$$ где x- количество нераспознанныx слов, y количество распознанных \\
 \hline
Успешно обработанные	& Успешно обработанные инциденты & 
$$ 
P=\frac{x}{100}
$$ где x успешно обработанные \\
 \hline
Не успешно обработанные	& Неуспешно обработанные инциденты & 
$$ 
N=\frac{y}{100}
$$ где y неуспешные инциденты \\
 \hline
Результативность	& Общая результативность обработчика & 
$$ 
R=\frac{P}{N}
$$  \\
  \hline
  Общий бал	& Общая оценка обработчика & 
$$ 
T=Ac+R
$$  \\
  \hline
  \hline
  \end{tabular}
%  \end{center}
\end{table}

На основе данных критериев был проведен анализ существующих подходов, результаты которого приведены на Диаграмме \ref{img:ParserComp}. По итогам главы был сделан вывод, что наиболее эффективен подход, использующийся в комплексе OpenCog Relex.

\begin{figure} [h] 
  \center
  \includegraphics [scale=1.0] {ParserCompare}
  \caption{Результаты обработки текстов} 
  \label{img:ParserComp}  
\end{figure}

\underline{\textbf{Третья глава}} посвящена исследованию 

В \underline{\textbf{четвертой главе}} приведено описание 

В \underline{\textbf{заключении}} приведены основные результаты работы, которые заключаются в следующем:
%% Согласно ГОСТ Р 7.0.11-2011:
%% 5.3.3 В заключении диссертации излагают итоги выполненного исследования, рекомендации, перспективы дальнейшей разработки темы.
%% 9.2.3 В заключении автореферата диссертации излагают итоги данного исследования, рекомендации и перспективы дальнейшей разработки темы.

Основные результаты работы заключаются в следующем.
\begin{enumerate}
  \item На основе анализа предметной области (поддержка информационной структуры предприятия) была выявлена потребность и возможность в автоматизации. Была построена модель предметной области. На основе модели предметной области, модели Марвина Мински была разработана модель проблемно-ориентированной системы принятия решений в области поддержки информационной структуры предприятия.  
  \item Испытания комплекса на модельных данных показали работоспособность модели и архитектуры.  
  \item Для выполнения поставленных задач был создан программный комплекс обработки, решения инцидентов и обучения на естественном языке. 
\end{enumerate}

Представленная в данной работе модель мышления, ее архитектура и реализация является уникальной в своем роде. На момент написания это была единственная реализация модели мышления Марвина Мински. \\
В работе были проведены исследования согласно паспорту специальности 05.13.01, сопоставление приведено в Таблице \ref{ResearchDescription}.

\begin{longtable}{|p{7cm}|p{9cm}|}
 \caption[Сопоставление направлений исследования специальности 05.13.01 и исследований, проведенных в работе]{Сопоставление направлений исследования специальности 05.13.01 и исследований, проведенных в работе}\label{ResearchDescription} \\ 
 \hline
 
 \multicolumn{1}{|c|}{\textbf{Направление исследования}} & \multicolumn{1}{c|}{\textbf{Результат работы}}  \\ \hline 
\endfirsthead
\multicolumn{2}{c}%
{{\bfseries \tablename\ \thetable{} -- продолжение}} \\
\hline \multicolumn{1}{|c|}{\textbf{Направление исследования}} &
\multicolumn{1}{c|}{\textbf{Результат работы}}  \\ \hline 
\endhead

\hline \multicolumn{2}{|r|}{{Продолжение следует}} \\ \hline
\endfoot

\hline \hline
\endlastfoot
\hline
   Разработка критериев и моделей описания и оценки эффективности решения задач системного анализа, оптимизации, управления, принятия решений и обработки информации & В рамках работы была разработана модель системы принятия решения и обработки информации в области решения запросов пользователя на естественном языке. \\
   \hline
   Разработка проблемно-ориентированных систем управления, принятия решений и оптимизации технических объектов & По модели, разработанной в предыдущем пункте был разработан прототип системы принятия решения Thinking-Understanding, который был испытан на модельных данных.\\
   \hline
   Методы получения, анализа и обработки экспертной информации & В рамках системы TU был разработан метод обработки экспертной информации - обучение при помощи модели мышления TU, основанной на принципах модели мышления Марвина Мински. \\
   \hline
   Разработка специального математического и алгоритмического обеспечения систем анализа, оптимизации, управления, принятия решений и обработки информации & В рамках разработки системы TU были разработаны специальные алгоритма для анализа запросов пользователя и принятия решений.\\
  \hline 
  Разработка специального математического и алгоритмического обеспечения систем анализа, оптимизации, управления, принятия решений и обработки информации & В рамках разработки системы TU были разработаны специальные алгоритма для анализа запросов пользователя и принятия решений.\\
  \hline 
  Теоретико-множественный и теоретико-информационный анализ сложных систем & В рамках работы был проведен комплексный анализ области поддержки программного обеспечения, с помощью которого была построена система данной области и выделены участки для оптимизации принятия решений.\\
  \hline
  Методы и алгоритмы интеллектуальной поддержки при принятии управленческих решений в технических системах & Система, разработанная в рамках данной работы в включает в себя инновационные методы и алгоритмы поддержки принятия решений, использующих в своей основе модель мышления на базе модели мышления Человека, описанной в книге Марвина Мински. \\ 
  \hline
  Визуализация, трансформация и анализ информации на основе компьютерных методов обработки информации & В Главе 1 представлена наглядная визуализация данных по системному анализу области удаленной поддержки инфраструктуры. \\
  \hline	
\end{longtable}

Разработанная в рамках работы системы не является узко-специализированной. Она также подходит для других областей, где требуется поддержка принятия решений. Например, при постановке медицинского диагноза, чтобы отбросить ложные диагнозы. \\
Например, систему можно обучить органам человека и их взаимосвязи. Далее можно обучить каким заболеваниям подвержен тот или иной орган. Далее к каждому заболеванию добавить симптом. После этого можно делать запрос с симптомами и система выдаст список вероятных заболеваний с их вероятностью и способы их лечения. \\
В области диагностики проблем в машиностроение. Обучить систему узлам автомобиля, проблемам с ними связанными, признаками этих проблем и способами их устранения. 






%\newpage
\renewcommand{\refname}{\large Публикации автора по теме диссертации}
\nocite{*}
%\insertbiblioauthor                          % Подключаем Bib-базы
\insertbiblioall

