\subsection*{Общая характеристика работы}

\newcommand{\actuality}{\underline{\textbf{Актуальность темы.}}}
\newcommand{\aim}{\underline{\textbf{Целью}}}
\newcommand{\tasks}{\underline{\textbf{задачи}}}
\newcommand{\scope}{\underline{\textbf{Область исследования}}}
\newcommand{\subject}{\underline{\textbf{Предметом исследования}}}
\newcommand{\methods}{\underline{\textbf{Методы исследования}}}
\newcommand{\defpositions}{\underline{\textbf{Основные положения, выносимые на~защиту:}}}
\newcommand{\novelty}{\underline{\textbf{Научная новизна:}}}
\newcommand{\influence}{\underline{\textbf{Практическая значимость.}}}
\newcommand{\reliability}{\underline{\textbf{Достоверность}}}
\newcommand{\probation}{\underline{\textbf{Апробация работы.}}}
\newcommand{\contribution}{\underline{\textbf{Личный вклад.}}}
\newcommand{\publications}{\underline{\textbf{Публикации.}}}

{\actuality}
В настоящее время в области IT набрали большую популярность системы удаленной поддержки информационной инфраструктуры предприятия, так называемый «Аутсорсинг». Ввиду развития рынка компаниям становится невыгодно держать свой штат службы поддержки, и они отдают свою информационную инфраструктуру сторонней компании.
Ввиду возросшей популярности данного бизнеса и появлением большого количества игроков на рынке возникла большая конкурентность, которая потребовала увеличения эффективности и сокращения издержек, что в свою очередь привело к необходимости системного анализа области и выработке решению сложившихся проблем. В контексте решения этой проблемы рассматривается модель области и модель системы, которая увеличивает эффективность работы путем частичной (в некоторых случаях полной) автоматизации обработки инцидентов, начиная с разбора входящих инцидентов на естественном языке и заканчивая применением найденного решения. 
Главными требованиями к подобной системе являются:
\begin{enumerate}
  \item Обработка запросов на естественном языке
  \item Возможность обучения
  \item Общение с человеческим специалистом
  \item Проведение логических рассуждений: аналогия, дедукция, индукция
  \item Умение абстрагировать решение одной проблемы и, экстраполируя его, применить для других решений
  \item Способность ответить на запрос пользователя
\end{enumerate}

На данный момент многие компании ведут разработку подобных систем. Примером является набирающая популярность IBM Watson. Подобный класс систем также называется вопросно-ответными системами. \\
В данной работе была создана подобная система на основе исследования целевой области и построения ее модели.
 Акцент был сделан на создании мыслящей системы для решения широкого круга проблем. \\
{\aim} данной работы является исследование целевой области, создание ее модели, выработка проблем области, оценка подходов к решению проблем,  создание архитектуры и реализация базового прототипа программного комплекса обеспечивающего разбор и формализацию входного запроса пользователя и поиск решения данной проблемы.

Для~достижения поставленной цели необходимо было решить следующие {\tasks}:
\begin{enumerate}
  \item Провести теоретико-множественный и теоретико-информационный анализ сложных систем в области поддержки информационной инфраструктуры
  \item Вычислить технико-экономическую возможность автоматизации целевой области
  \item Создать модель целевой области
  \item Исследовать модели мышления и выбрать наиболее подходящую
  \item На основе выбранной модели мышления разработать модель проблемно-ориентированной системы управления, принятия решений и оптимизации технических объектов в области обслуживания информационной структуры предприятия
  \item Создать архитектуру приложения на основе модели
  \item Реализовать прототип на основе архитектуры
  \item Провести апробацию прототипа на тестовых данных
\end{enumerate}

\defpositions
\begin{enumerate}
  \item Теоретико-множественный и теоретико-информационный анализ сложных систем в области поддержки информационной инфраструктуры
  \item Модель проблемно-ориентированной системы управления, принятия решений и оптимизации технических объектов в области обслуживания IT, ее технико-эконономическое обоснование  
  \item Прототип программной реализации модели проблемно-ориентированной системы управления, принятия решений и оптимизации технических объектов в области обслуживания IT  
  \item Апробация системы на контрольных примерах и ее результаты
\end{enumerate}

\novelty
\begin{enumerate}
  \item Была создана модель проблемно-ориентированной системы управления, принятия решений и оптимизации технических объектов в области обслуживания информационной структуры предприятия на основе модели мышления
  \item Доказана применимость модели для других областей
  \item Была представлена новая модель данных для модели мышления и оригинальный способ ее хранения, обеспечивающий быстрый доступ
  \item Было выполнено оригинальное исследование моделей мышления в области обслуживания информационной структуры предприятия
  \item На основе модели была создана архитектура системы и ее прототип 
\end{enumerate}

\influence\ 
Система, разрабатываемая в рамках данной работы носит значимый практический характер. Идея работы зародилась из производственных проблем в IT отрасли, с которыми автор сталкивался каждый день. Только глубокое понимание проблем помогло выбрать правильное решение. Более подробное описание представлено в Главе 1.
\reliability\ полученных результатов обеспечивается результатами выполнения тестов на контрольных примерах. Результаты находятся в соответствии с результатами, полученными другими авторами, экспертными системами и специалистами. \\ 

\probation\
Основные результаты работы докладывались на:
\begin{itemize}
	\item RCDL-2014
	\item AINL-2013
	\item WCIT-2012
	\item AMSTA-2015
\end{itemize}

\contribution\ Автор принимал активное участие в исследовании целевой области, разработке архитектуры приложения, реализации прототипа, проработки теории, тестировании прототипа.

\publications\ Основные результаты по теме диссертации изложены в 6 печатных изданиях  \cite{Lobachevskii},\cite{WCIT-2012},\cite{RCDL-2014},\cite{AINL-2013},\cite{ISGZ},\cite{AMSTA-2015}, 
2 из которых изданы в журналах Scopus, 1 в журнале РИНЦ  \cite{ISGZ}, 
4 в тезисах докладов \cite{Lobachevskii},\cite{WCIT-2012},\cite{AINL-2013},\cite{ISGZ}, \cite{IJSE-1}.



 % Характеристика работы по структуре во введении и в автореферате не отличается (ГОСТ Р 7.0.11, пункты 5.3.1 и 9.2.1), потому её загружаем из одного и того же внешнего файла, предварительно задав форму выделения некоторым параметрам

%Диссертационная работа была выполнена при поддержке грантов ...

%\underline{\textbf{Объем и структура работы.}} Диссертация состоит из~введения, четырех глав, заключения и~приложения. Полный объем диссертации \textbf{ХХХ}~страниц текста с~\textbf{ХХ}~рисунками и~5~таблицами. Список литературы содержит \textbf{ХХX}~наименование.

%\newpage
\subsection*{Содержание работы}
Во \underline{\textbf{введении}} обосновывается актуальность исследования, проводимых в рамках данной диссертационной работы, дается общая характеристика работы. Проводится обзор области и производится постановка задачи. \par
\underline{\textbf{Первая глава}} посвящена обзору интеллектуальных систем регистрации и анализа проблемных ситуаций, возникающих в ИТ-инфраструктуре предприятия. Здесь представлен сравнительный анализ систем регистрации и устранения проблемных ситуаций. В главе определяются основные требования к интеллектуальным системам регистрации и анализа проблемных ситуаций в ИТ. Одним из важных элементов подобных систем является обработка естественного языка, поэтому здесь представлен сравнительный анализ методов и программных комплексов обработки текстов в этой области. \par
Во время анализа были использованы следующие средств обработки естественного языка: Open NLP; Relex; StanfordParser.
Оценка работы проводилась при помощи метрик, представленных в таблице \ref{Metrics}. Результаты приведены на рисуноке \ref{img:ParserCompare}. Как видно максимальный результат по всем трем метрикам показывает система Relex, она и была выбрана в качестве средства обработки естественного языка.

\begin{table} [htbp]
  \centering
  \parbox{15cm}{\caption{Таблица метрик}\label{Metrics}}
%  \begin{center}
  \begin{tabular}{| p{5cm} |p{5cm}| p{5cm} |}
  \hline

\textbf{Метрика} & \textbf{Описание} & \textbf{Формула} \\
  \hline

Precision	& Точность & 
$$ 
P=\frac{tp}{tp+fp},
$$ где P~--- precision, tp~---  успешно обработанные, fp~--- ложно успешные \\
 \hline
Recall	& Чувствительность & 
$$ 
R=\frac{tp}{tp+fn},
$$ где R~--- recall, tp~--- успешно обработанные, fn~--- ложно неуспешные \\
 \hline
F	& F~--- measure (результативность) & 
$$ 
F=\frac{P*R}{P+R},
$$ где P~--- precision, R~--- recall.   \\
 \hline
  \end{tabular}
%  \end{center}
\end{table}


\begin{figure} [h] 
  \center
  \includegraphics [scale=0.8] {ParserCompare}
  \caption{Результаты анализа средств обработки естественного языка.} 
  \label{img:ParserCompare}  
\end{figure}

Кроме того, в главе был проведен анализ существующих решений автоматизации в области поддержки информационной структуры пользователя.
Все рассмотренные системы не соответствуют полному комплексу необходимых требований. В таблице \ref{Comparsion} приведены сводные данные по системам. Таблица показывает наличие той или иной функции у рассматриваемой системы. Как видно ни одно из рассмотренных решений не умеет проводить логический рассуждения. Наиболее развитым на сегодняшней день решением является программный комплекс IBM Watson.

\begin{longtable}{|p{6cm}|p{0.5cm}|p{0.5cm}|p{0.5cm}|}
 \caption[Сравнительный анализ существующих решений.]{Сравнительный анализ существующих решений.}\label{Comparsion} \\ 
 \hline
 
 \multicolumn{1}{|c|}{\textbf{Сравнительный пункт}} & \multicolumn{1}{c|}{\textbf{HP Open View}} & \multicolumn{1}{c|}{\textbf{ServiceNOW}} & \multicolumn{1}{c|}{\textbf{IBM Watson}} \\ \hline 
\endfirsthead
\multicolumn{2}{c}%
{{\bfseries \tablename\ \thetable{} -- продолжение}} \\
\hline \multicolumn{1}{|c|}{\textbf{Сравнительный пункт}} & \multicolumn{1}{c|}{\textbf{HP Open View}} & \multicolumn{1}{c|}{\textbf{ServiceNOW}} & \multicolumn{1}{c|}{\textbf{IBM Watson}}  \\ \hline 
\endhead
\endfoot

\hline \hline
\endlastfoot
\hline
   Мониторинг & Да & Да & Да \\
   \hline
   Регистрация инцидентов & Да & Да & Да\\
   \hline
   Управление системами & Да & Нет & Нет \\
   \hline 
   Создание цепи обработки (Workflow) инцидента & Да & Да & Нет \\
   \hline 
   Понимания и формализацию запросов на естественном языке & Нет & Нет & Да \\
   \hline 
   Поиск решений & Нет & Нет & Да \\
   \hline 
   Применение решений & Нет & Нет & Нет \\
   \hline
   Обучение решению инцидента & Нет & Нет & Да \\
   \hline
   Умение проводить логические рассуждения: генерализацию, специализацию, синонимичный поиск & Нет & Нет & Нет \\
   e
   
\end{longtable}
%=================
%===Second chapter
%=================

\underline{\textbf{Вторая глава}} посвящена построению модели интеллектуальной системы принятия решений для регистрации и анализа проблемных ситуаций в ИТ-инфраструктуре предприятия. В рамках главы рассматривается три принципиальных подхода для решения проблемы.
 \begin{itemize}
	\item модель Menta 0.1, построенная с использованием деревьев принятия решений;
	\item модель Menta 0.3, построенная с использованием генетических алгоритмов;
	\item модель TU 1.0, основанная на модели мышления Марвина Мински.
\end{itemize} \par

Необходимо отметить, что модель, построенная на базе нейронных сетей (поддерживающая обучение) была отброшена на предварительной стадии оценки, так как она предъявляет большие требования к производительности, что в свою очередь порождает высокую стоимость. Далее каждая модель будет рассмотрена подробно.

\textbf{Модель Menta 0.1, построенная с использованием деревьев принятия решений}.
Данная модель являлась одной из первых, которая была опробована. Модель была основана на деревьях принятия решений. В построение модели данной системы использовались следующие компоненты: обработка запросов на естественном языке; поиск решения; применение решения. \par
Системы была ориентирована на выполнение простых команд, например, добавить поле на форму. В целом работа системы описывается следующим алгоритмом:
\begin{enumerate}
	\item Получение и формализация запроса;
	\item Поиск решения при помощи Деревьев Принятия Решений;
	\item Изменение модели приложения в формате OWL;
	\item Генерация и компиляция приложения.
\end{enumerate} \par
После проведения экспериментов были выявлены следующие проблемы: отсутствие устойчивости к ошибкам входной информации: грамматическим и содержательным. Например, входной файл не имел отношения к программной системе, модель которой была в базе знаний в формате OWL; система поиска решения работала только в рамках модели одной программы;  отсутствовала функция обучения. \par



\textbf{Модель Menta 0.3, построенная с использованием генетических алгоритмов}.
В данную модель по сравнению с предыдущей были добавлены модуль логики для оценки решения и модуль генетических алгоритмов для генерации решения. В рамках модели Menta 0.3 были отработаны следующие основные компоненты будущей итоговой модели: критерии приемки (Acceptance Criteria); How-To~--- для хранения решений проанализированных проблем; формат данных OWL; использование логических вычислений для проверки решения. Система Menta 0.3 содержала внутри себя модель целевого приложения (как и Menta 0.1) и список решений тех или иных проблем (How-To). При помощи генетического алгоритма модель строила How-To решение проверяла его при помощи логического движка NARS на соответствие входным критериям приемки. С точки зрения генетических алгоритмов это~--- функция отбора особей из поколения.  \par
После проведения экспериментов были выявлены следующие проблемы: отсутсвие обучения; отсутсвие обработки естественного языка; после апробации оказалось, что критерии приемки практически описывают необходимое решение (то которое должно быть найдено), что являлось недопустимым. \par


\textbf{Модель TU 1.0, основанная на модели мышления Марвина Мински}.
Модель была построена с применением теории Марвина Мински. Эта модель сохранила следующие основные концептуальные элементы предыдущих моделей и показала свою состоятельность на контрольных примерах: Acceptance Criteria; обучение; поиск и применение решения; Отсутсвие обработки естественного языка. Данная модель является более универсальной и представляет собой верхнеуровневую архитектуру обработки запроса (мышления), где компонентами являются лучшие части предыдущих систем. Реализация модели получила название TU. \par
Одним из основных компонентов системы является
\underline{\triplet}. На рисунке \ref{img:csw} представлена схематичное изображение Критика-Селектора-Образа мышления. Критик реагирует, Селектор выбирает ресурс, Образ мышления выполняет работу. \\
\begin{figure} [h] 
  \center
  \includegraphics [scale=1.0] {CSW}
  \caption{Критик-Селектор-Образ мышления} 
  \label{img:csw}  
\end{figure}


\underline{Критик (Critic)} представляет собой определенный переключатель: внешние обстоятельства, события или иное воздействие. Например, «включился свет, и зрачки сузились», «обожглись и одернули руку». Критик активируется только тогда, когда для этого достаточно обстоятельств. Одновременно могут активироваться несколько критиков. Например, человек решает сложную задачу, идет активация множество критиков: выполнить расчет, уточнить технические детали. Кроме того, параллельно может активироваться критик переработки, сообщающий о необходимости отдыха.\par
\underline{Селектор (Selector)} занимается выбором определенных ресурсов, которыми также являются Образ мышления. \par
\underline{Образ мышления (WayToThink)}~--- это способ решения проблемы. Образ мышления может быть сложным и, например, активировать другие критики. Например,  размышляя над проблемой, специалист понимает, что нужно произвести полный перебор, и тут он решает поискать готовое решение: а может кто-то уже сделал такой перебор и можно будет его использовать. Здесь "поиск готового решения" является критиком внутри образа мышления "поиск решения".\par

На рисунке \ref{img:csw_ex} представлена расширенная модель работы триплета \triplet. Критик активирует Селектор, который возвращает ресурс Образ мышления (кругами отмечены различные ресурсы: Критики, Селекторы, Образа мышления \etc). Последний в свою очередь может активировать нового Критика или же совершить определенные действия. Например, зажегся зеленый свет светофора, значит, можно переходить дорогу. Под ресурсами здесь понимается набор знаний из базы знаний: Критики, Селекторы, Образы мышления, готовые решения. \par
Если активировалось много критиков, то проблему нужно уточнить, так как степень неопределенности слишком высока. Если проблема очень похожа на уже проанализированную, то можно действовать и судить по аналогии. \par
\begin{figure} [h] 
  \center
  \includegraphics [scale=1.0] {CSW_EX}
  \caption{Критик-Селектор-Образ мышления в разрезе ресурсов} 
  \label{img:csw_ex}  
\end{figure}
Другой важной концепцией теории являются уровни мышления. Это концепция распределяет активность мышления между 6-ю уровнями: чем выше уровень, тем сильне активность. В Таблице \ref{ThinkingLevelDescription} представлено описание уровней мышления с примерами. \par
На этом исследование моделей мышлений было завершено и были сделаны выводы. 
\begin{table} [htbp]
  \centering
  \parbox{15cm}{\caption{Описание уровней мышления модели 6-ти}\label{ThinkingLevelDescription}}
%  \begin{center}
  \begin{tabular}{| p{5cm} | p{11cm} |}
  
  \hline
\textbf{Уровень} & \textbf{Описание} \\
  \hline
  
Инстинктивный уровень	& Происходят инстинктивные реакции (врожденные). Например, коленный рефлекс. Общую формулу для этого уровня можно выразить как "Если ..., то сделать так". \\
  \hline

Уровень обученных реакций  & Используются накопленные знания, то есть те знания, которым человек обучается в течение жизни. Например, переходить дорогу на зеленый свет. Общую формулу для этого уровня можно описать как \quote{Если ..., то сделать так}. \\
  \hline

Уровень рассуждений & Мышление с использованием рассуждений. Например, если перебежать дорогу на зеленый свет, то можно успеть вовремя. На данном уровне сравниваются последствия нескольких решений и выбирается оптимальное. Общую формулу для этого уровня можно выразить как \quote{Если ..., то сделать так, тогда будет так}. \\
  \hline

Рефлексивный уровень  & Рассуждения с учетом анализа прошлых событий. Например, в прошлый раз я побежал на моргающий зеленый и чуть не попал под машину. \\

  \hline
  Саморефлексивный уровень & Построение определенной модели, с помощью которой идет оценка своих поступков. Например, мое решение не пойти на это собрание было неверным, так как я упустил столько возможностей, я был легкомысленным. \\
  \hline
  Самосознательный уровень & Оценка своих поступков с точки зрения высших идеалов и оценок окружающих. Например, \quote{А что подумают мои друзья? А как бы поступил мой герой}? \\
  \hline
  
  \end{tabular}
%  \end{center}
\end{table}


Для программной экспертной системы очень важно обладать способностью мыслить и рассуждать. Например, действовать по аналогии. Множество запросов типичны и отличаются лишь параметрами. Например, пожалуйста, установить Office, Antivirus \etc Также для экспертной системы важно уметь абстрагировать специализированные рецепты решения. Например, система научилась решать инцидент "Please install Firefox", абстрагировав данный инцидент до степени "Please install browser", система сможет теми же способами попробовать решить новый инцидент.\par
После рассмотрения нескольких моделей была выбрана модель мышления Марвина Мински, так как данная модель наиболее точно ложится на целевую область решения инцидентов в ИТ. На основе подхода Мински была построена модель системы, которая реализует основные функции: обучение, понимание инцидента, поиск решения, применение решения. 

\clearpage
%=================
%===3rd chapter
%=================
В \underline{\textbf{третьей главе}} приведено описание архитектуры и реализации системы, основанной на модели Thinking Understanding.
Архитектура представляет собой модули. Основными компоненты системы описаны в Таблице \ref{MainComponents}. Система может функционировать в режиме обучения и в режиме решения запросов. 
\begin{longtable}{|p{7cm}|p{8cm}|}
 \caption[Основные компоненты системы Thinking Understanding]{Основные компоненты системы Thinking Understanding}\label{MainComponents} \\ 
 \hline
 
 \multicolumn{1}{|c|}{\textbf{Компонент}} & \multicolumn{1}{c|}{\textbf{Описание}}  \\ \hline 
\endfirsthead
\multicolumn{2}{c}%
{{\bfseries \tablename\ \thetable{} -- продолжение}} \\
\hline \multicolumn{1}{|c|}{\textbf{Компонент}} &
\multicolumn{1}{c|}{\textbf{Описание}}  \\ \hline 
\endhead

\endfoot

\hline \hline
\endlastfoot
\hline
   TU Webservice & Основной компонент взаимодействия со внешними система, включая пользователя. \\
   \hline
   CoreService & Ядро системы, содержит основные классы.\\
   \hline
   DataService & Компонент работы с данными. \\
   \hline 
   Reasoner & Компонент вероятностной логики. \\
   \hline 
   ClientAgent & Компонент выполнения скриптов на целевой машине. \\
   \hline 
   MessageBus & Шина данных для системы. \\
   \hline 
\end{longtable}
В главе приводится детальное описание всех компонентов и подкомпонентов. Для понимание приводится описание механизма взаимодействия компонентов и общий сценарий использования системы.
 \begin{enumerate}
	\item Поступает запрос от пользователя: 
	"User had received wrong application. User has ordered Wordfinder Business Economical. However she received wrong version, she received Wordfinder Tehcnical instead of Business Economical. Please assist." («Пользователь получил неверное приложение. Пользователь заказал Wordfinder Бизнес версию. Но получил неверную версию, он получил Wordfinder Техническая версия вместо Бизнес версии. Пожалуйста, помогите»);
	\item Компонент GoalManger (Менеджер целей) устанавливает цель системы HelpUser (Помочь пользователю);
	\item Главный компонент Thinking Life Cycle (далее TLC) активирует набор компонетов Critic (Критик), привязанный к данной цели (HelpUser); 
	\item Активируется компонент PreliminaryAnnorator (Предварительный обработчик), который разбирает фразу;
	\item Компонент KnowledgeBaseAnnotator (Разбор при помощи накопленных знаний) создает семантическую сеть и ссылки на нее;
	\item Компонент Critic (Критик), привязанный к цели HelpUser на Пефликсивном уровне запускает WayToThink (Образ мышления) ProblemSolving (Решить проблему) с целью: ResolveIncident;
	\item Компонент Critic на Рефликсивном уровне выбирает WayToThink KnowingHow (Поиск рецепта решения);
	\begin{enumerate}
	\item Запускаются параллельно все компоненты Critic, которые привязаны к IncidentClassification (Классифицировать проблему) Critic, который привязан к ResolveIncident (Решить проблему) цели, в данном случае это DirectInstruction (Прямые инструкции), ProblemWithDesiredState, ProblemWithoutDesiredState;
	\item Компонент Selector (Селектор) выбирает наиболее вероятный результат работы среди всех результатов. В данном случае будет результат работы Problem Description with desired state (Проблема с желаемым состоянием);
	\item Компонент KnowingHow сохраняет варианты выбора Selector;
	\item Компонент Simulation (Моделирование) WayToThink с параметрами «создать модель текущий ситуации» создает: концепцию существующей ситуации (CurrentState), концепцию пользователя, концепцию программного обеспечения;
	\item Компонент Reformulation WayToThink (Компонент дополнения), используя результаты предыдущего шага, синтезирует артефакты, которых не хватает, чтобы получить из CurrentState DesiredState (Желаемое состояние), так как он не указан явно. WayToThink запускает Critic размышления, чтобы найти корень проблемы. Он находит CurrentState~--- Wordfinder Tehcnical, DesiredState~--- Wordfinder Business Economical;
	\item Рефлексивные Critic оценивают состояние системы~--- на каком шаге она находится, и если цель не достигнута, то запускают другой WayToThink, например, DirectInstruction;
	\item Компонент Critic генерации решения запускает KnowingHow WayToThink, ExtensiveSearch (Поиск решения);
	\item Компонент Selector выбирает наиболее вероятный образ мышления. В данном случае ExtensiveSearch, который будет находить решения, позволяющие привести систему в желаемое состояние (DesiredState). Если он не сможет, то он иницирует коммуникацию с пользователем. 
 \end{enumerate}
	 \item Рефлексивный Critic проверяет состояние системы. Если Цель достигнута, то пользователю посылается ответ.
	 \item Само Сознательные Critic активируется на данном шаге и сохраняют информацию о затратах на решение;
  \end{enumerate}\par
Для работы системы была разработана уникальная модель данных~--- TU Knowledge, которая сочетает в себе OWL и графовую базу данных. Язык OWL, родившейся для структурирования информации Web обрел широкое использование во многих схемах данных, так как давал возможность дополнительного расширенного описания взаимосвязи между данными. На рисунке \ref{img:KnowledgeClass} представлена схема данных TU Knowledge. В Таблице \ref{TUKnowledge} представлено описание схемы TU Knowledge.


%=================
%===4 chapter
%=================
В \underline{\textbf{главе 4}} приведены экспериментальные исследования эффективности работы модели TU.
Система показала свою жизнеспособность на контрольных примерах. Были проведены тесты в сравнении с работой человеческого специалиста. Был выбран контрольный список инцидентов. Сравнивался поиск решения для инцидентов. Основное время при опросе специалиста тратилось на коммуникацию. В таблице \ref{HumanComparison} приведены результаты сравнения. Тесты были выполнены на машине Intel Core i7 1700 MHz, 8GB RAM, 256 GB SSD, FreeBSD. Из результатов видно, что система работает также или лучше чем специалист.
\begin{longtable}{|p{12cm}|p{2cm}|p{2cm}|}
 \caption[Результаты сравнения с работой специалиста]{Результаты сравнения с работой специалиста}\label{HumanComparison} \\ 
 \hline
 
 \multicolumn{1}{|c|}{\textbf{Инцидент}} & \multicolumn{1}{c|}{\textbf{TSS1 (.мс)}} & \multicolumn{1}{c|}{\textbf{TU (.мс)}}  \\ \hline 
\endfirsthead
\multicolumn{2}{c}%
{{\bfseries \tablename\ \thetable{} -- продолжение}} \\
\multicolumn{1}{|c|}{\textbf{Инцидент}} & \multicolumn{1}{c|}{\textbf{TSS1 (.мс)}} & \multicolumn{1}{c|}{\textbf{TU (.мс)}}  \\ \hline 
\endhead

\endfoot

\hline \hline
\endlastfoot
\hline
  Tense is kind of concept.(Время~--- это концепция.) & 15000 & 385 \\
  
  \hline
  Please install Firefox. (Установите Firefox.)   & 9000 & 859 \\
  \hline
  Browser is an object. (Браузер~--- это объект.)   & 20000 & 400 \\
  \hline
  Firefox is a browser. (Firefox~--- это браузер.)   & 5000 & 659  \\
  \hline
  Install is an action.  (Установить~--- это действие.)   & 8000 & 486 \\
  \hline
  User miss Internet Explorer 8. (У пользователя нет Internet Explorer 8.)     & 10000 & 10589 \\
  \hline
  User needs document portal update. (Пользователю требуется обновление документов.)    & 15000 & 16543 \\
  \hline
  Add new alias Host name on host that alias is wanted to: hrportal.lalala.biz IP adress on host that alias is wanted to: 322.223.333.22 Wanted Alias:    webadviser.lalala.net. (Добавьте пожалуйста новую ссылку на hrportal.lalala.biz через 322.223.333.22.)    & 10000 & 18432  \\ 
  \hline
  Outlook Web Access (CCC) - 403 - Forbidden: Access is denied. (Нет доступа к Outlook Web Access (CCC).) & 15000 & 10342\\ 
  \hline
  PP2C~--- Cisco IP communicator. Please see if you can fix the problem with the ip phone, it's stuck on configuring ip + sometimes Server error rejected: Security etc. (PP2C~--- коммуникатор Cisco IP. Пожалуйста, помогите исправить проблему с ИП-телефоном, он застревает вовремя конфигурирования и иногда показывает ошибку «Безопасность».  & 13000 & 12343 \\ 
   \hline
  \end{longtable}

%=================
%===Conclusion
%=================
В \underline{\textbf{заключении}} приведены основные выводы по работе.
%% Согласно ГОСТ Р 7.0.11-2011:
%% 5.3.3 В заключении диссертации излагают итоги выполненного исследования, рекомендации, перспективы дальнейшей разработки темы.
%% 9.2.3 В заключении автореферата диссертации излагают итоги данного исследования, рекомендации и перспективы дальнейшей разработки темы.

Основные результаты работы заключаются в следующем.
\begin{enumerate}
  \item На основе анализа предметной области (поддержка информационной структуры предприятия) была выявлена потребность и возможность в автоматизации. Была построена модель предметной области. На основе модели предметной области, модели Марвина Мински была разработана модель проблемно-ориентированной системы принятия решений в области поддержки информационной структуры предприятия.  
  \item Испытания комплекса на модельных данных показали работоспособность модели и архитектуры.  
  \item Для выполнения поставленных задач был создан программный комплекс обработки, решения инцидентов и обучения на естественном языке. 
\end{enumerate}

Представленная в данной работе модель мышления, ее архитектура и реализация является уникальной в своем роде. На момент написания это была единственная реализация модели мышления Марвина Мински. \\
В работе были проведены исследования согласно паспорту специальности 05.13.01, сопоставление приведено в Таблице \ref{ResearchDescription}.

\begin{longtable}{|p{7cm}|p{9cm}|}
 \caption[Сопоставление направлений исследования специальности 05.13.01 и исследований, проведенных в работе]{Сопоставление направлений исследования специальности 05.13.01 и исследований, проведенных в работе}\label{ResearchDescription} \\ 
 \hline
 
 \multicolumn{1}{|c|}{\textbf{Направление исследования}} & \multicolumn{1}{c|}{\textbf{Результат работы}}  \\ \hline 
\endfirsthead
\multicolumn{2}{c}%
{{\bfseries \tablename\ \thetable{} -- продолжение}} \\
\hline \multicolumn{1}{|c|}{\textbf{Направление исследования}} &
\multicolumn{1}{c|}{\textbf{Результат работы}}  \\ \hline 
\endhead

\hline \multicolumn{2}{|r|}{{Продолжение следует}} \\ \hline
\endfoot

\hline \hline
\endlastfoot
\hline
   Разработка критериев и моделей описания и оценки эффективности решения задач системного анализа, оптимизации, управления, принятия решений и обработки информации & В рамках работы была разработана модель системы принятия решения и обработки информации в области решения запросов пользователя на естественном языке. \\
   \hline
   Разработка проблемно-ориентированных систем управления, принятия решений и оптимизации технических объектов & По модели, разработанной в предыдущем пункте был разработан прототип системы принятия решения Thinking-Understanding, который был испытан на модельных данных.\\
   \hline
   Методы получения, анализа и обработки экспертной информации & В рамках системы TU был разработан метод обработки экспертной информации - обучение при помощи модели мышления TU, основанной на принципах модели мышления Марвина Мински. \\
   \hline
   Разработка специального математического и алгоритмического обеспечения систем анализа, оптимизации, управления, принятия решений и обработки информации & В рамках разработки системы TU были разработаны специальные алгоритма для анализа запросов пользователя и принятия решений.\\
  \hline 
  Разработка специального математического и алгоритмического обеспечения систем анализа, оптимизации, управления, принятия решений и обработки информации & В рамках разработки системы TU были разработаны специальные алгоритма для анализа запросов пользователя и принятия решений.\\
  \hline 
  Теоретико-множественный и теоретико-информационный анализ сложных систем & В рамках работы был проведен комплексный анализ области поддержки программного обеспечения, с помощью которого была построена система данной области и выделены участки для оптимизации принятия решений.\\
  \hline
  Методы и алгоритмы интеллектуальной поддержки при принятии управленческих решений в технических системах & Система, разработанная в рамках данной работы в включает в себя инновационные методы и алгоритмы поддержки принятия решений, использующих в своей основе модель мышления на базе модели мышления Человека, описанной в книге Марвина Мински. \\ 
  \hline
  Визуализация, трансформация и анализ информации на основе компьютерных методов обработки информации & В Главе 1 представлена наглядная визуализация данных по системному анализу области удаленной поддержки инфраструктуры. \\
  \hline	
\end{longtable}

Разработанная в рамках работы системы не является узко-специализированной. Она также подходит для других областей, где требуется поддержка принятия решений. Например, при постановке медицинского диагноза, чтобы отбросить ложные диагнозы. \\
Например, систему можно обучить органам человека и их взаимосвязи. Далее можно обучить каким заболеваниям подвержен тот или иной орган. Далее к каждому заболеванию добавить симптом. После этого можно делать запрос с симптомами и система выдаст список вероятных заболеваний с их вероятностью и способы их лечения. \\
В области диагностики проблем в машиностроение. Обучить систему узлам автомобиля, проблемам с ними связанными, признаками этих проблем и способами их устранения. 





 

%\newpage
\renewcommand{\refname}{\large Публикации автора по теме диссертации}

%\insertbiblioauthor                          % Подключаем Bib-базы
\insertbiblioall

