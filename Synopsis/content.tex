\subsection*{Общая характеристика работы}

\newcommand{\actuality}{\underline{\textbf{Актуальность темы.}}}
\newcommand{\aim}{\underline{\textbf{Целью}}}
\newcommand{\tasks}{\underline{\textbf{задачи}}}
\newcommand{\defpositions}{\underline{\textbf{Основные положения, выносимые на~защиту:}}}
\newcommand{\novelty}{\underline{\textbf{Научная новизна:}}}
\newcommand{\influence}{\underline{\textbf{Практическая значимость}}}
\newcommand{\reliability}{\underline{\textbf{Достоверность}}}
\newcommand{\probation}{\underline{\textbf{Апробация работы.}}}
\newcommand{\contribution}{\underline{\textbf{Личный вклад.}}}
\newcommand{\publications}{\underline{\textbf{Публикации.}}}

{\actuality}
В настоящее время в области IT набрали большую популярность системы удаленной поддержки информационной инфраструктуры предприятия, так называемый «Аутсорсинг». Ввиду развития рынка компаниям становится невыгодно держать свой штат службы поддержки, и они отдают свою информационную инфраструктуру сторонней компании.
Ввиду возросшей популярности данного бизнеса и появлением большого количества игроков на рынке возникла большая конкурентность, которая потребовала увеличения эффективности и сокращения издержек, что в свою очередь привело к необходимости системного анализа области и выработке решению сложившихся проблем. В контексте решения этой проблемы рассматривается модель области и модель системы, которая увеличивает эффективность работы путем частичной (в некоторых случаях полной) автоматизации обработки инцидентов, начиная с разбора входящих инцидентов на естественном языке и заканчивая применением найденного решения. 
Главными требованиями к подобной системе являются:
\begin{enumerate}
  \item Обработка запросов на естественном языке
  \item Возможность обучения
  \item Общение с человеческим специалистом
  \item Проведение логических рассуждений: аналогия, дедукция, индукция
  \item Умение абстрагировать решение одной проблемы и, экстраполируя его, применить для других решений
  \item Способность ответить на запрос пользователя
\end{enumerate}

На данный момент многие компании ведут разработку подобных систем. Примером является набирающая популярность IBM Watson. Подобный класс систем также называется вопросно-ответными системами. \\
В данной работе была создана подобная система на основе исследования целевой области и построения ее модели.
 Акцент был сделан на создании мыслящей системы для решения широкого круга проблем. \\
{\aim} данной работы является исследование целевой области, создание ее модели, выработка проблем области, оценка подходов к решению проблем,  создание архитектуры и реализация базового прототипа программного комплекса обеспечивающего разбор и формализацию входного запроса пользователя и поиск решения данной проблемы.

Для~достижения поставленной цели необходимо было решить следующие {\tasks}:
\begin{enumerate}
  \item Провести теоретико-множественный и теоретико-информационный анализ сложных систем в области поддержки информационной инфраструктуры
  \item Вычислить технико-экономическую возможность автоматизации целевой области
  \item Создать модель целевой области
  \item Исследовать модели мышления и выбрать наиболее подходящую
  \item На основе выбранной модели мышления разработать модель проблемно-ориентированной системы управления, принятия решений и оптимизации технических объектов в области обслуживания информационной структуры предприятия
  \item Создать архитектуру приложения на основе модели
  \item Реализовать прототип на основе архитектуры
  \item Провести апробацию прототипа на тестовых данных
\end{enumerate}

\defpositions
\begin{enumerate}
  \item Теоретико-множественный и теоретико-информационный анализ сложных систем в области поддержки информационной инфраструктуры
  \item Модель проблемно-ориентированной системы управления, принятия решений и оптимизации технических объектов в области обслуживания IT, ее технико-эконономическое обоснование  
  \item Прототип программной реализации модели проблемно-ориентированной системы управления, принятия решений и оптимизации технических объектов в области обслуживания IT  
  \item Апробация системы на контрольных примерах и ее результаты
\end{enumerate}

\novelty
\begin{enumerate}
  \item Была создана модель проблемно-ориентированной системы управления, принятия решений и оптимизации технических объектов в области обслуживания информационной структуры предприятия на основе модели мышления
  \item Доказана применимость модели для других областей
  \item Была представлена новая модель данных для модели мышления и оригинальный способ ее хранения, обеспечивающий быстрый доступ
  \item Было выполнено оригинальное исследование моделей мышления в области обслуживания информационной структуры предприятия
  \item На основе модели была создана архитектура системы и ее прототип 
\end{enumerate}

\influence\ 
Система, разрабатываемая в рамках данной работы носит значимый практический характер. Идея работы зародилась из производственных проблем в IT отрасли, с которыми автор сталкивался каждый день. Только глубокое понимание проблем помогло выбрать правильное решение. Более подробное описание представлено в Главе 1.
\reliability\ полученных результатов обеспечивается результатами выполнения тестов на контрольных примерах. Результаты находятся в соответствии с результатами, полученными другими авторами, экспертными системами и специалистами. \\ 

\probation\
Основные результаты работы докладывались на:
\begin{itemize}
	\item RCDL-2014
	\item AINL-2013
	\item WCIT-2012
	\item AMSTA-2015
\end{itemize}

\contribution\ Автор принимал активное участие в исследовании целевой области, разработке архитектуры приложения, реализации прототипа, проработки теории, тестировании прототипа.

\publications\ Основные результаты по теме диссертации изложены в 6 печатных изданиях  \cite{Lobachevskii},\cite{WCIT-2012},\cite{RCDL-2014},\cite{AINL-2013},\cite{ISGZ},\cite{AMSTA-2015}, 
2 из которых изданы в журналах Scopus, 1 в журнале РИНЦ  \cite{ISGZ}, 
4 в тезисах докладов \cite{Lobachevskii},\cite{WCIT-2012},\cite{AINL-2013},\cite{ISGZ}, \cite{IJSE-1}.



 % Характеристика работы по структуре во введении и в автореферате не отличается (ГОСТ Р 7.0.11, пункты 5.3.1 и 9.2.1), потому её загружаем из одного и того же внешнего файла, предварительно задав форму выделения некоторым параметрам

%Диссертационная работа была выполнена при поддержке грантов ...

%\underline{\textbf{Объем и структура работы.}} Диссертация состоит из~введения, четырех глав, заключения и~приложения. Полный объем диссертации \textbf{ХХХ}~страниц текста с~\textbf{ХХ}~рисунками и~5~таблицами. Список литературы содержит \textbf{ХХX}~наименование.

%\newpage
\subsection*{Содержание работы}
Во \underline{\textbf{введении}} обосновывается актуальность исследования, проводимых в рамках данной диссертационной работы, дается общая характеристика работы.
\underline{\textbf{Первая глава}} посвящена постановки задачи. Проводится обзор и построение модели целевой области и обосновывается возможность ее автоматизации. На Диаграмме \ref{img:ITSMTeamComposition} представлен качественно процентный состав в команд с точки зрения квалификации специалистов. \\
\begin{figure} [h] 
  \center
  \includegraphics [scale=0.7] {ITSMTeamComposition}
  \caption{Диаграмма состава команд} 
  \label{img:ITSMTeamComposition}  
\end{figure}

В главе 1 приведены результаты анализа категорий проблем, которые решают специалисты \ref{img:EngineerTasks}. В главе приведено технико-экономическое обоснование целевого программного комплекса, где выведен необходимый порог в 50\% решения системой инцидентов самостоятельно. 
\begin{figure} [h] 
  \center
  \includegraphics [scale=0.7] {EngineerTasks}
  \caption{Диаграмма соотношений типов проблем} 
  \label{img:EngineerTasks}  
\end{figure}

%=================
%===Second chapter
%=================

\clearpage
\underline{\textbf{Вторая глава}} посвящена анализу текущих решений получения, анализа и обработки экспертной информации в области обслуживания программного обеспечения и информационной инфраструктуры. Было выбрано 3 наиболее популярных (по данным ОАО "ICL КПО ВС") на сегодняшней день решений: HPOpenView, ServiceNOW, IBMWatson. Были выработаны критерии сравнения и требования к целевой системе. В Таблице \ref{Comparsion} приведены результаты сравнения по этим критериям комплексов: Relex, OpenNLP, Stanford. В главе также был выработан набор тестовых данных, разработаны критерии оценки работы комплексов обработки естественного языка в применение к области удаленной поддержки инфраструктуры. В Таблице \ref{Metrics} приведены эти метрики.
\begin{table} [htbp]
  \centering
  \parbox{15cm}{\caption{Таблица метрик}\label{Metrics}}
%  \begin{center}
  \begin{tabular}{| p{5cm} ||p{5cm}|| p{5cm} |}
  \hline
  \hline
Метрика & Описание & Формула \\
  \hline
  \hline
Precision	& Точность & 
$$ 
P=\frac{tp}{tp+fp}
$$ где P - precision, tp -  успешно обработанные, fp - ложно успешные \\
 \hline
Recall	& Чувствительность & 
$$ 
R=\frac{tp}{tp+fn}
$$ где R-recall, tp - успешно обработанные, fn - ложно неуспешные \\
 \hline
F	& F-measure (результативность) & 
$$ 
F=\frac{P*R}{P+R}
$$ Где P - precision, R-recall.   \\
 \hline
  \end{tabular}
%  \end{center}
\end{table}

На основе данных критериев был проведен анализ существующих подходов к обработке естественного языка, результаты которого приведены на Диаграмме \ref{img:ParserComp}. По итогам главы был сделан вывод, что наиболее эффективен подход, использующейся в комплексе OpenCog Relex.

\begin{figure} [h] 
  \center
  \includegraphics [scale=0.8] {ParserCompare}
  \caption{Результаты обработки текстов} 
  \label{img:ParserComp}  
\end{figure}
\clearpage


\begin{longtable}{|p{6cm}|p{0.5cm}|p{0.5cm}|p{0.5cm}|}
 \caption[Сравнительный анализ существующих решений]{Сравнительный анализ существующих решений}\label{Comparsion} \\ 
 \hline
 
 \multicolumn{1}{|c|}{\textbf{Сравнительный пункт}} & \multicolumn{1}{c|}{\textbf{HP Open View}} & \multicolumn{1}{c|}{\textbf{ServiceNOW}} & \multicolumn{1}{c|}{\textbf{IBM Watson}} \\ \hline 
\endfirsthead
\multicolumn{2}{c}%
{{\bfseries \tablename\ \thetable{} -- продолжение}} \\
\hline \multicolumn{1}{|c|}{\textbf{Сравнительный пункт}} & \multicolumn{1}{c|}{\textbf{HP Open View}} & \multicolumn{1}{c|}{\textbf{ServiceNOW}} & \multicolumn{1}{c|}{\textbf{IBM Watson}}  \\ \hline 
\endhead

\hline \multicolumn{2}{|r|}{{Продолжение следует}} \\ \hline
\endfoot

\hline \hline
\endlastfoot
\hline
   Мониторинг & Да & Да & Да \\
   \hline
   Регистрация инцидентов & Да & Да & Да\\
   \hline
   Управление системами & Да & Нет & Нет \\
   \hline 
   Создание цепи обработки (Workflow) инцидента & Да & Да & Нет \\
   \hline 
   Понимания и формализацию запросов на естественном языке & Нет & Нет & Да \\
   \hline 
   Поиск решений & Нет & Нет & Да \\
   \hline 
   Применение решений & Нет & Нет & Нет \\
   \hline
   Обучение решению инцидента & Нет & Нет & Да \\
   \hline
   Умение проводить логические рассуждения: генерализацию, специализацию, синонимичный поиск & Нет & Нет & Нет \\
   \hline
   \textbf{Итоговые очки} & 4 & 3 & 5 \\
   \hline 
\end{longtable}

\clearpage
%=================
%===3rd chapter
%=================
В \underline{\textbf{третьей главе}} приведено описание теоретического базиса системы и ее модели. Во время работы было рассмотрено и прототипировано четыре различных подхода к моделям мышления:
\begin{itemize}
	\item Модель мышления Марвина Мински (Модель 6-ти)
	\item Модель мышления на базе нейронных сетей 
	\item Модель с использованием Деревьев Принятия Решений (Menta 0.1)
	\item Модель с использованием Генетических алгоритмов на базе модели мышления Питера Норвига  (Menta 0.3)
\end{itemize}



\textbf{Модель мышления на базе нейронных сетей}
Модель на базе нейронных сетей (поддерживающая обучение) была отброшена на предварительной стадии оценки, так как имеет большие требования производительности несовместимые с Технико Экономическим Обоснованием.

\textbf{Модель с использованием Деревьев Принятия Решений (Menta 0.1)}
Данная модель являлась одной из первых, которая была опробована. Модель была основана на деревьях принятия решений. В построение модели данной системы использовались следующие компоненты:
\begin{itemize}
	\item Обработка запросов на естественном языке
	\item Поиск решения
	\item Применение решения
\end{itemize}
Системы была ориентирована на выполнение простых команд, например, добавить поле на форму. Основаная функция модели представлена следующим потоком:
\begin{enumerate}
	\item Получение и формализация запроса
	\item Поиск решения при помощи Деревьев Принятия Решений
	\item Изменение модели приложения в формате OWL
	\item Генерация и компиляция приложения
\end{enumerate}
Основными проблемами данной модели являлось следующее:
\begin{enumerate}
	\item Отсутствие устойчивости к ошибкам входной информации: грамматическим и содержательным. Например, входной файл не имел отношения к программной системе, модель которой была в базе знаний в формате OWL
	\item Система поиска решения работала только в рамках модели одной программы
	\item Отсутствовала функция обучения 
\end{enumerate}



\textbf{Модель с использованием Генетических алгоритмов на базе модели мышления Питера Норвига (Menta 0.2-0.3)}
После работы над ошибками была предпринята попытка сделать поиск решения более универсальным. В рамках данной модели были сформированы основные компоненты системы:
\begin{itemize}
	\item Критерии Приемки (Acceptance Criteria)
	\item Формат данных OWL 
	\item Использование логических вычислений
	\item Универсальный поиск решения
\end{itemize}
Система содержала внутри себя модель приложения. При помощи генетического алгоритма модель строила из частей новую систему и проверяла ее при помощи логического движка NARS на соответствие входным критерия приемки, заданными пользователем. Основными недостатками подхода оказалось:
\begin{itemize}
	\item Отсутсвие обучения
	\item Отсутсвие обработки естественного языка
	\item После апробации оказалось, что критерии приемки практически описывают необходимое решение (то которое должно быть найдено), что являлось недопустимым. 
\end{itemize} 

\textbf{Модель мышления Марвина Мински (Модель 6-ти)}
Модель была построена с применением модели 6-ти Марвина Мински. Она содержит в себе основные концепции предыдущих моделей и показывает свою состоятельность на контрольных примерах. Реализация модели получила название TU.
\begin{itemize}
	\item Обучение
	\item Поиск и применение решения 
	\item Обработка естественного языка
\end{itemize}
Данная модель является более абстрактной и представляет собой верхнеуровневую архитектуру обработки запроса (мышления), где компонентами являются лучшие части предыдущих систем. Основным компонентом системы является
\underline{Критик-Селектор-Образ мышления}. На Рисунке \ref{img:csw} представлена схематичное изображение Критика-Селектора-Пути мышления. \\
\begin{figure} [h] 
  \center
  \includegraphics [scale=1.0] {CSW}
  \caption{Критик-Селектор-Образ мышления} 
  \label{img:csw}  
\end{figure}


\underline{Критик (Critic)} представляет собой определенный предикат: внешние обстоятельства, события или иное воздействие. Например, включился свет и зрачки сузились. Обожглись и одернули руку. Критик активируется только когда для этого достаточно входных стимулов. Одновременно могут активироваться несколько критиков. Например, человек решает сложную задачу. Идет активация множество критиков: считать, технические детали, кроме того параллельно может активироваться критик переработки, сообщающей о необходимости отдыха.\\
\underline{Селектор (Selector)} занимается выбором определенных ресурсов, которыми также являются Образы мышления. \\
\underline{Образ мышления (WayToThink)} это способ решения проблемы. Образ мышления также может активировать следующий критик. \\

На рисунке \ref{img:csw_ex} представления расширенная модель работы триплета Критик-Селектор-Образ мышления. Критик активирует селектор, который активирует образ мышления (синий круг). образ мышления в свою очередь может активировать критик или же совершить определенные действия. Например, зажегся зеленый свет светофора, значит можно переходить дорогу. \\
Если активировалось много критиков, значит проблему нужно уточнить, так как степень неопределенности слишком высока. Если проблема очень похожа, то можно судить по аналогии.
\begin{figure} [h] 
  \center
  \includegraphics [scale=1.0] {CSW_EX}
  \caption{Критик-Селектор-Образ мышления в разрезе ресурсов} 
  \label{img:csw_ex}  
\end{figure}
В Таблице \ref{ThinkingLevelDescription} представлено описание уровней мышления.
\begin{table} [htbp]
  \centering
  \parbox{15cm}{\caption{Описание уровней мышления Марвина Мински}\label{ThinkingLevelDescription}}
%  \begin{center}
  \begin{tabular}{| p{5cm} | p{11cm} |}
  \hline
  \hline
Уровень & Описание \\
  \hline
    \hline

Инстинктивный уровень	& На данном уровне происходят инстинктивные реакции (врожденные). Например, возможность дышать. Общую формулу для этого уровня можно выразить как "Если ..., то сделать так". \\
  \hline

Уровень обученных реакций  & На  данном уровне происходит мышление обученных реакций, то есть тех реакций, которыми человек обучается в течение жизни. Например, переходить дорогу на зеленых свет. Общую формулу для этого уровня можно выразить как "Если ..., то сделать так". \\
  \hline

Уровень рассуждений & а  данной уровне происходит мышление с использованием рассуждений. Если я сделаю так, то будет ... Например, если перебежать дорогу на зеленый свет, то можно успеть вовремя. На данном уровне сравниваются последствия нескольких решений и выбирается оптимальное. Общую формулу для этого уровня можно выразить как "Если ..., то сделать так, тогда будет так". \\
  \hline

Рефлексивный уровень  & На данном уровне происходит рассуждение с учетом анализа прошлых событий. Например, в прошлый раз я побежал на моргающий зеленый и чуть не попал под машину. \\

  \hline
  Саморефлексивный уровень & На данном уровне происходит оценка себя. Строится определенная модель с помощью которой идет оценка своих поступков. Например, мое решение не пойти на это собрание было неверным, так как я упустил столько возможностей, я был легкомысленный. \\
  \hline
  Самосознательный уровень & На данном уровне идет оценка поступков человека с точки зрения высших идеалов и внешних оценок. Например, а что подумают мои друзья? А как бы поступил мой герой? \\
  \hline
  
  \end{tabular}
%  \end{center}
\end{table}
\clearpage
%=================
%===4 chapter
%=================
В \underline{\textbf{Главе 4}} приведены основные результаты работы, которые заключаются в следующем:
\begin{itemize}
	\item Теоретико-множественный и теоретико-информационный анализ сложных систем в области поддержки информационной инфраструктуры
	\item Проблемно-ориентированная система управления, принятия решений и
оптимизации технических объектов в области обслуживания IT
	\item Архитектура системы, ее реализация и испытания на модельных данных
	\item Описание компонентов системы
\end{itemize}
Архитектура системы представляет собой модульную систему. Основными компоненты системы описаны в Таблице \ref{MainComponents}. Система может функционировать в режиме обучения и в режиме решения запросов. 
\begin{longtable}{|p{7cm}|p{8cm}|}
 \caption[Основные компоненты системы ThinkingUnderstanding]{Основные компоненты системы ThinkingUnderstanding}\label{MainComponents} \\ 
 \hline
 
 \multicolumn{1}{|c|}{\textbf{Компонент}} & \multicolumn{1}{c|}{\textbf{Описание}}  \\ \hline 
\endfirsthead
\multicolumn{2}{c}%
{{\bfseries \tablename\ \thetable{} -- продолжение}} \\
\hline \multicolumn{1}{|c|}{\textbf{Компонент}} &
\multicolumn{1}{c|}{\textbf{Описание}}  \\ \hline 
\endhead

\hline \multicolumn{2}{|r|}{{Продолжение следует}} \\ \hline
\endfoot

\hline \hline
\endlastfoot
\hline
   TU Webservice & Основной компонент взаимодействия со внешними система, включая пользователя. \\
   \hline
   CoreService & Ядро системы, содержит основные классы.\\
   \hline
   DataService & Компонент работы с данными. \\
   \hline 
   Reasoner & Компонент вероятностной логики. \\
   \hline 
   ClientAgent & Компонент выполнения скриптов на целевой машине. \\
   \hline 
   MessageBus & Шина данных для системы. \\
   \hline 
\end{longtable}
В главе приводится основной рабочий поток работы приложения.
 \begin{enumerate}
	\item Поступает запрос от пользователя \\
	User had received wrong application.
User has ordered Wordfinder Business Economical.
However she received wrong version, she received Wordfinder Tehcnical instead of Business Economical. Please assist.
	\item GoalManger устанавливает цель системы HelpUser
	\item Активируется набор Critic, привязанный к данной цели
	\item PreliminaryAnnorator разбирает фразу
	\item KnowledgeBaseAnnotator создает семантическую сеть и ссылки на нее
	\item Critic, привязанный к цели HelpUser на Рефликсивном уровне запускает WayToThink ProblemSolving с целью: ResolveIncident
	\item Critic на Рефликсивном уровне выбирает WayToThink KnowingHow
	\begin{enumerate}
	\item Запускаются параллельно все Critic, которые привязаны к IncidentClassification Critic, который привязан к ResolveIncident цели, в данном случае это DirectInstruction, ProblemWithDesiredState, ProblemWithoutDesiredState
	\item Selector выбирает наиболее вероятный результат работы среди всех результатов компонентов. В данном случае будет результат работы Problem Description with desired state.
	\item KnowingHow сохраняет варианты выбора Selector.
	\item Simulation WayToThink с параметрами "Создать модель текущий ситуации" создает: CurrentSituation, User, Software
	\item Reformulation WayToThink, используя результаты предыдущего шага синтезирует артефакты, которых не хватает, чтобы получить из CurrentState DesiredState, так как он не указан явно. WayToThink запускает Critic размышления, чтобы найти корень проблемы. Он находит CurrentState- Wordfinder Tehcnical, DesiredState-Wordfinder Business Economical
	\item Рефлексивные Critic оценивают состояние системы - на каком шаге она находится, и если цель не достигнута, то запускают другой WayToThink, например, DirectInstruction. 
	\item Critic генерации решения запускает KnowingHow WayToThink, ExtensiveSearch.
	\item Selector выбирает наиболее вероятный образ мышления. В данном случае ExtensiveSearch, который будет находить решения, позволяющие привести систему в необходимое состояние (DesiredState). Если он не сможет, то он иницирует коммуникацию с пользователем. 
 \end{enumerate}
	 \item Рефлексивный Critic проверяет состояние системы. Если Цель достигнута, то пользователю посылается ответ.
	 \item Само Сознательные Critic активируется на данном шаге и сохраняют информацию о затратах на решение.

\end{enumerate}
В главе представлены методика и результаты апробации работы.
%=================
%===Conclusion
%=================
В \underline{\textbf{заключении}} приведены основные выводы по работе.
%% Согласно ГОСТ Р 7.0.11-2011:
%% 5.3.3 В заключении диссертации излагают итоги выполненного исследования, рекомендации, перспективы дальнейшей разработки темы.
%% 9.2.3 В заключении автореферата диссертации излагают итоги данного исследования, рекомендации и перспективы дальнейшей разработки темы.

Основные результаты работы заключаются в следующем.
\begin{enumerate}
  \item На основе анализа предметной области (поддержка информационной структуры предприятия) была выявлена потребность и возможность в автоматизации. Была построена модель предметной области. На основе модели предметной области, модели Марвина Мински была разработана модель проблемно-ориентированной системы принятия решений в области поддержки информационной структуры предприятия.  
  \item Испытания комплекса на модельных данных показали работоспособность модели и архитектуры.  
  \item Для выполнения поставленных задач был создан программный комплекс обработки, решения инцидентов и обучения на естественном языке. 
\end{enumerate}

Представленная в данной работе модель мышления, ее архитектура и реализация является уникальной в своем роде. На момент написания это была единственная реализация модели мышления Марвина Мински. \\
В работе были проведены исследования согласно паспорту специальности 05.13.01, сопоставление приведено в Таблице \ref{ResearchDescription}.

\begin{longtable}{|p{7cm}|p{9cm}|}
 \caption[Сопоставление направлений исследования специальности 05.13.01 и исследований, проведенных в работе]{Сопоставление направлений исследования специальности 05.13.01 и исследований, проведенных в работе}\label{ResearchDescription} \\ 
 \hline
 
 \multicolumn{1}{|c|}{\textbf{Направление исследования}} & \multicolumn{1}{c|}{\textbf{Результат работы}}  \\ \hline 
\endfirsthead
\multicolumn{2}{c}%
{{\bfseries \tablename\ \thetable{} -- продолжение}} \\
\hline \multicolumn{1}{|c|}{\textbf{Направление исследования}} &
\multicolumn{1}{c|}{\textbf{Результат работы}}  \\ \hline 
\endhead

\hline \multicolumn{2}{|r|}{{Продолжение следует}} \\ \hline
\endfoot

\hline \hline
\endlastfoot
\hline
   Разработка критериев и моделей описания и оценки эффективности решения задач системного анализа, оптимизации, управления, принятия решений и обработки информации & В рамках работы была разработана модель системы принятия решения и обработки информации в области решения запросов пользователя на естественном языке. \\
   \hline
   Разработка проблемно-ориентированных систем управления, принятия решений и оптимизации технических объектов & По модели, разработанной в предыдущем пункте был разработан прототип системы принятия решения Thinking-Understanding, который был испытан на модельных данных.\\
   \hline
   Методы получения, анализа и обработки экспертной информации & В рамках системы TU был разработан метод обработки экспертной информации - обучение при помощи модели мышления TU, основанной на принципах модели мышления Марвина Мински. \\
   \hline
   Разработка специального математического и алгоритмического обеспечения систем анализа, оптимизации, управления, принятия решений и обработки информации & В рамках разработки системы TU были разработаны специальные алгоритма для анализа запросов пользователя и принятия решений.\\
  \hline 
  Разработка специального математического и алгоритмического обеспечения систем анализа, оптимизации, управления, принятия решений и обработки информации & В рамках разработки системы TU были разработаны специальные алгоритма для анализа запросов пользователя и принятия решений.\\
  \hline 
  Теоретико-множественный и теоретико-информационный анализ сложных систем & В рамках работы был проведен комплексный анализ области поддержки программного обеспечения, с помощью которого была построена система данной области и выделены участки для оптимизации принятия решений.\\
  \hline
  Методы и алгоритмы интеллектуальной поддержки при принятии управленческих решений в технических системах & Система, разработанная в рамках данной работы в включает в себя инновационные методы и алгоритмы поддержки принятия решений, использующих в своей основе модель мышления на базе модели мышления Человека, описанной в книге Марвина Мински. \\ 
  \hline
  Визуализация, трансформация и анализ информации на основе компьютерных методов обработки информации & В Главе 1 представлена наглядная визуализация данных по системному анализу области удаленной поддержки инфраструктуры. \\
  \hline	
\end{longtable}

Разработанная в рамках работы системы не является узко-специализированной. Она также подходит для других областей, где требуется поддержка принятия решений. Например, при постановке медицинского диагноза, чтобы отбросить ложные диагнозы. \\
Например, систему можно обучить органам человека и их взаимосвязи. Далее можно обучить каким заболеваниям подвержен тот или иной орган. Далее к каждому заболеванию добавить симптом. После этого можно делать запрос с симптомами и система выдаст список вероятных заболеваний с их вероятностью и способы их лечения. \\
В области диагностики проблем в машиностроение. Обучить систему узлам автомобиля, проблемам с ними связанными, признаками этих проблем и способами их устранения. 





Работа над системой продолжается. На данный момент удалось:
\begin{itemize}
	\item Добиться 50\% обработки инцидентов системой
	\item Построить рабочий прототип системы
	\item Провести апробацию созданный на основе модели 6-ти Мински системы
	\item Сформировать модель области поддержки информационной структуры предприятия
	\item Создать алгоритмы обработки инцидентов
\end{itemize}
 

%\newpage
\renewcommand{\refname}{\large Публикации автора по теме диссертации}
\nocite{*}
%\insertbiblioauthor                          % Подключаем Bib-базы
\insertbiblioall

