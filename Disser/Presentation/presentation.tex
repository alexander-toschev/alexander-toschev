\documentclass[14pt]{beamer}
\usepackage[T2A]{fontenc}
\usepackage[utf8]{inputenc}
\usepackage[english,russian]{babel}
\usepackage{amssymb,amsfonts,amsmath,mathtext}
\usepackage{cite,enumerate,float,indentfirst}
\usepackage{longtable}  
\graphicspath{{../assets/}{images/}} 

\usetheme{Pittsburgh}
\usecolortheme{whale}

\setbeamercolor{footline}{fg=blue}
\setbeamertemplate{footline}{
  \leavevmode%
  \hbox{%
  \begin{beamercolorbox}[wd=.333333\paperwidth,ht=2.25ex,dp=1ex,center]{}%
    А. С. Тощев, К(П)ФУ
  \end{beamercolorbox}%
  \begin{beamercolorbox}[wd=.333333\paperwidth,ht=2.25ex,dp=1ex,center]{}%
    Казань, 2017
  \end{beamercolorbox}%
  \begin{beamercolorbox}[wd=.333333\paperwidth,ht=2.25ex,dp=1ex,right]{}%
  Стр. \insertframenumber{} из \inserttotalframenumber \hspace*{2ex}
  \end{beamercolorbox}}%
  \vskip0pt%
}

\newcommand{\itemi}{\item[\checkmark]}

\title{\small{Диссертация на соискания ученой степени кандидата технических наук по специальности 05.13.11}}
\author{
Интеллектуальная система повышения эффективности IT службы предприятия\\
\small{%
\emph{Соискатель:} А.C. Тощев\\%
\emph{Руководитель:} проф., д.ф.-м.н. А.М.Елизаров}\\%
\vspace{30pt}%
Казанский (Приволжский)\\
федеральный университет%
\vspace{20pt}%
}
\date{\small{Казань, 2017}}

\begin{document}

\maketitle


\begin{frame}
\frametitle{Термины и обозначения}
\begin{enumerate}
    \item ITIL~-- общепринятая методология в области поддержки IT;
    \item Инцидент~-- проблема, возникшая в результате работы ПО и приведшая к полной или частичной невозможности работы;
    \item TSS1~-- системный администратор 3 категории;
    \item TU~-- интеллектуальная система повышения эффективности ИТ-службы предприятия.
  
\end{enumerate}
\end{frame}


  \begin{frame}
   \frametitle{Содержание доклада}
    \tableofcontents
   \end{frame}


\AtBeginSection[] % Do nothing for \subsection*
{
\begin{frame}
\frametitle{}
\tableofcontents[current]
	
\end{frame}
}
% 
%    INTRODUCTION
%
\section[Введение]{Введение}

\begin{frame}
\frametitle{Цели и задачи}
\begin{itemize}
  \item \textbf{Предмет исследования} процесс регистрации и устранения проблемных ситуаций, возникающих в IT-инфраструктуре предприятия;
  \item \textbf{Цель исследования} разработка интеллектуальной системы повышения эффективности деятельности IT-службы предприятия;
  \item \textbf{Актуальность} определяется потребностью предприятий IT-отрасли в интеллектуальных системах, повышающих эффективность служб, поддерживающих IT-инфраструктуру этих предприятий.
\end{itemize}
\end{frame}

\begin{frame}
\frametitle{Близкие исследования}
\begin{enumerate}
 \item Институт Чиная (Индия) - Е. Джубилсон и П. Дханавантини;
 \item Институт Ганновера (Германия) – Р. Брунс и Дж. Данкель;
 \item СПбГУ (Россия) - В.И. Золотарев;
 \item Сингапур – С. Фу и П. Леонг;
 \item IBM Watson (IBM) - А. Гоэль;
 \item GATE3 (Университет Шеффилда (Великобритания)) – Г. Каллаган;
 \item OpenCog (США) –  Б. Герцель;
 \item NARS (Китай) – П. Вонг.
\end{enumerate}
\end{frame}



\begin{frame}
\frametitle{Обзор существующих решений}
\begin{table}
	
\small
\begin{tabular} {|p{5cm}|p{1cm}|p{1.5cm}|p{1.5cm}|}

\hline
\textbf{Критерий сравнения} & HP Open View & Service NOW & IBM Watson\\
\hline
   Мониторинг & Да & Да & Да  \\
   \hline
   Регистрация инцидентов & Да & Да & Да \\
   \hline
   Управление системами & Да & Нет & Нет  \\
   \hline 
   Создание цепи обработки & Да & Да & Нет \\
   \hline 
   Запросы на естественном языке & Нет & Нет & Да \\
   \hline 
   Поиск решений & Нет & Нет & Да \\
   \hline 
   Применение решений & Нет & Нет & Нет  \\
   \hline
   Обучение & Нет & Нет & Нет \\
   \hline
   Логические рассуждения & Нет & Нет & Нет  \\
   \hline
  
\end{tabular}
\end{table}
\end{frame}

\begin{frame}
\frametitle{Методы исследования}
\begin{enumerate}
  \item \textbf{Теоретические методы}
  \begin{itemize}
    \item Имитационное моделирование;
    \item Теория баз знаний в области искусственного интеллекта.
  \end{itemize}
   \item \textbf{Специальные методы}
  \begin{itemize}
    \item Экспериментальное моделирование;
    \item Cистемное моделирование.
  \end{itemize}
  
   \item \textbf{Экспериментальные методы}
  \begin{itemize}
    \item Метод наблюдений;
    \item Метод проведения экспериментов.
  \end{itemize}

\end{enumerate}
\end{frame}


\begin{frame}
\frametitle{Соответствие паспорту специальности}

\begin{table}
	
\small

\begin{tabular} {|p{5cm}|p{5cm}|}


 \hline
\textbf{Направление исследования} & Результат работы\\

\hline
   Языки программирования и системы программирования, семантика программ & Разработана семантическая модель организации хранения знаний \\
   \hline
  Системы управления базами данных и знаний & Разработан прототип Thinking Understanding (TU) системы хранения знаний и принятия решений в сфере поддержки ИТ-инфраструктуры предприятия, который был испытан на модельных данных\\
   \hline
    \end{tabular}
\end{table}
\end{frame}

\begin{frame}
\frametitle{Соответствие паспорту специальности}

\begin{table}
	
\small

\begin{tabular} {|p{5cm}|p{5cm}|}


 \hline
\textbf{Направление исследования} & Результат работы\\

 \hline
   Модели и методы создания программ и программных систем для параллельной и распределенной обработки данных, языки и инструментальные средства параллельного программирования & Разработан метод параллельной обработки экспертной информации c возможностью обучения при помощи прототипа TU \\
   \hline
 \end{tabular}
\end{table}
\end{frame}

\begin{frame}
\frametitle{Список публикаций}
 Основные результаты по теме диссертации изложены в 10 печатных изданиях:
\begin{itemize}
	\item Scopus:2;
	\item Web of science:1;
	\item РИНЦ:4;
	\item Перечень ВАК: 2;
	\item ACM: 2.
\end{itemize}
\end{frame}

\begin{frame}
\frametitle{Список публикаций}

\begin{itemize}
	\item Тощев, А.С. К новой концепции автоматизации программного обеспечения [Текст] / А. С. Тощев // Труды Математического центра имени Н.И. Лобачевского. «Лобачевские чтения — 2011. Казань, 31 октября – 4 ноября 2011». –– 2011. –– Т. 44, No 4. –– С. 279 – 282; 
	\item Toshchev, A. Thinking-Understanding approach in IT maintenance domain automation [Text] / A. Toshchev, M. Talanov, A. Krehov // Global Journal on Technology: 3rd World Conference on Information Technology (WCIT-2012). — 2013. — Vol. 3. — P. 879 – 894;
	
\end{itemize}
\end{frame}

\begin{frame}
\frametitle{Список публикаций}

\begin{itemize}
	\item Тощев, А.С. Архитектура и реализация интеллектуального агента для автоматической обработки входящих заявок с помощью искусственного интеллекта и семантических сетей [Текст] / А.С. Тощев, М.О. Таланов // Ученые записки Института социально-гуманитарных знаний. –– 2014. –– Т. 2. –– С. 288 – 292; 
	\item Toshchev, A. Computational emotional thinking and virtual neurotransmitters [Text] / A. Toshchev, M. Talanov // International Journal of Synthetic Emotions (IJSE). — 2014. — Vol. 5. — P. 30 – 35;
	
\end{itemize}
\end{frame}


\begin{frame}
\frametitle{Список публикаций}

\begin{itemize}
	\item Toshchev, A. Appraisal, coping and high level emotions aspects of computational emotional thinking [Text] / A. Toshchev, M. Talanov // International Journal of Synthetic Emotions (IJSE). — 2015. — Vol. 6. — P. 65 – 72; 
	\item Toshchev, A. Thinking model and machine understanding in automated user request processing [Text] / A. Toshchev // CEUR Workshop Proceedings. — 2014. — Vol. 1297. — P. 224 – 226;
\end{itemize}
\end{frame}


\begin{frame}
\frametitle{Список публикаций}

\begin{itemize}
	\item Тощев, А.С. Возможности автоматизации разрешения инцидентов для области удаленной поддержки информационной инфраструктуры предприятия [Текст] / А.С. Тощев // Экономика и менеджмент систем управления. –– 2015. –– Т. 4. –– С. 293 – 295;
	\item Toshchev, A. Thinking lifecycle as an implementation of machine understanding in software maintenance automation domain [Text] / A. Toshchev, M. Talanov // 9th KES International Conference, KES-AMSTA. -- 2015. — Vol. 38. -- P. 301 – 310;
\end{itemize}
\end{frame}

\begin{frame}
\frametitle{Список публикаций}

\begin{itemize}
	\item Тощев, А.C. Вычислительная модель эмоций в интеллектуальных информационных системах [Текст] / А.C. Тощев, М.О. Таланов // Электронные библиотеки. –– 2015. –– Т. 18. –– С. 225 – 235;
	\item Тощев, А.С. Применение моделей мышления в интеллектуальных вопросно-ответных системах [Текст] / А.С. Тощев // Электронные библиотеки. –– 2015. –– Т. 18. –– С. 216 – 224.
\end{itemize}
\end{frame}


\begin{frame}
\frametitle{Выступления на конференциях}

\begin{itemize}
	\item Десятая молодежная научная школа-конференция «Лобачевские чтения —2011». Казань, 31 октября – 4 ноября 2011 года;
	\item Международная конференция ”3rd World Conference on Information Technology (WCIT-2012)”. Barcelona, 14 – 16 November 2012, Spain;
	\item II Международная конференция «Искусственный интеллект и естественный язык (AINL-2013)». Санкт-Петербург, 17 – 18 мая 2013 года
	
	
\end{itemize}
\end{frame}


\begin{frame}
\frametitle{Выступления на конференциях}

\begin{itemize}
	\item VI Международная научно-практическая конференция «Электронная Казань 2014». Казань, 22 – 24 апреля 2014 года;
\item XVI Всероссийская научная конференция «Электронные библиотеки: перспективные методы и технологии, электронные коллекции (RCDL--2014)». Дубна, 13 -– 16 октября 2014 года;
\item Семинары по программной инженерии ”All-Kazan Software Engineering Seminar (AKSES-2015)”. Kazan, 9 April 2015;


	
	
\end{itemize}
\end{frame}

\begin{frame}
\frametitle{Выступления на конференциях}

\begin{itemize}
	\item Международная конференция ”Agents and multi-agent systems: Technologies and applications (AMSTA-2015)”. Sorento, 17 –- 19 June 2015, Italy.
	
\end{itemize}
\end{frame}

\begin{frame}
\frametitle{Структура диссертации}
\begin{itemize}
\item 4 главы, введение и заключение.
\begin{itemize}
  \item \textbf{Глава 1. Интеллектуальные системы регистрации и анализа проблемных ситуаций, возникающих в ИТ-инфраструктуре предприятия};
  \item  \textbf{Глава 2. Модель интеллектуальной системы принятия решений для регистрации и анализа проблемных ситуаций в ИТ-инфраструктуре предприятия};
  \item \textbf{Глава 3. Реализация модели TU 1.0 для системы интеллектуальной регистрации и устранения проблемных ситуаций};
  \item \textbf{Глава 4. Экспериментальные исследования эффективности работы модели TU}.
 \end{itemize}
\end{itemize}
\end{frame}







%
%					CHAPTER 1
%
\section[Глава 1]{Глава 1. Интеллектуальные системы регистрации и анализа проблемных ситуаций, возникающих в ИТ-инфраструктуре предприятия}
\begin{frame}
\frametitle{Исходные данные и постановка задачи}
\begin{enumerate}
  \item Задача: удаленная помощь пользователям;
  \item Диапазон исследования: 1 месяц;
  \item Количество инцидентов: 2920;
  \item Для создания системы и ее апробации были в качестве исходных данных использована информация, которая была собрана в рамках деятельности ICL.


\end{enumerate}
\end{frame}

\begin{frame}
\frametitle{Классификация заявок}
\begin{figure} [h] 
  \center
  \includegraphics [scale=0.7] {EngineerTasks}
  \label{img:EngineerTasks}  
\end{figure}


\end{frame}


\begin{frame}
\frametitle{Проблемы автоматизации}
\begin{enumerate}
 \item Неоднозначные запросы (примеры);
	\begin{enumerate}
 		\item The installation of Winrar that I got this afternoon did go wrong. During installation nothing else was running. When I tried to start Winrar I got the fault message that is attached here;
 		\item Before i went to vacation i got LOT234, please check if it installed.
	\end{enumerate}
 \item Грамматические ошибки (примеры);
  \begin{enumerate}
 		\item The installation of Winrar that I got this afternoon did go wrong. During installation nothing else was running. When I tried to start Winrar I got the fault message that is attached here.
	\end{enumerate}
	
  \item Запросы на естественном языке.
\end{enumerate}
\end{frame}


\begin{frame}
\frametitle{Имитационная модель процессов обработки инцидентов на базе СМО}
\begin{figure} [h] 
  \center
  \includegraphics [scale=0.5] {mass_service}
  \label{img:mass_service}  
\end{figure}

\end{frame}

\begin{frame}
\frametitle{Характеристики модели}
\begin{enumerate}
 \item $\lambda$ --- интенсивность входящего потока (заявок в час);
    \item $\alpha$ --- доля заявок, для которых время в очереди превышает $max(T_q)$;       
    \item $\mu$ --- величина, обратная среднему времени нахождения заявки у агента;
	\item  $n$ --- число агентов;
	\item $T_q$ --- время нахождение заявки в очереди в минутах;
	\item SLA=1-$\alpha$ --- уровень обслуживания, доля заявок, для которых время в очереди не превышает $max(T_q)$. 
	\item $T_p$ --- время удовлетворения заявки (час);
 	

\end{enumerate}
\end{frame}

\begin{frame}
\frametitle{Характеристики модели}
\begin{enumerate}
 \item $\alpha_n$ --- количество заявок;
 \item	$T_{qp}=T_q+T_p$ --- время прохождения заявки через систему (час);
 \item	$S(\mu)= \frac{R_p}{\mu} $ --- средняя стоимость выполнения одной заявки;
 \item $R_p$ --- средняя стоимость часа работы специалиста (выводится далее);
 \item \textbf{Данные для моделирования:} $T_{qp}=47,9ч$ при $n=6$; $SLA=0,82$; $\alpha=0,18$;  $\alpha_n=2920$. 
\end{enumerate}
\end{frame}





%
%					CHAPTER 2
%исследования выполнялись для усовершенстования, расширить -публикации
\section[Глава 2]{Глава 2. Модель интеллектуальной системы принятия решений}
\begin{frame}
\frametitle{Созданные модели}
\begin{figure} [h] 
  \center 
  \includegraphics [scale=0.6] {ModelEvolution}
  \caption{Созданные модели} 
  \label{img:ModelEvolution} 
\end{figure}
\begin{minipage}{11cm}
\footnotesize 
\begin{enumerate}
	\item Таланов, М. "Automating programming via concept mining, probabilistic reasoning over semantic knowledge base of SE domain";
	\item Тощев, А, Таланов,  М. "Document Thinking model and machine understanding in automated user request processing";
	\item Тощев, А. "Thinking lifecycle as an implementation of machine understanding in software maintenance domain".
\end{enumerate}
\end{minipage}
\end{frame}


\begin{frame}
\frametitle{Модель $T^3$ по модели Марвина Мински}
\begin{figure} [h] 
  \center
  \includegraphics [scale=0.6] {CSW_EX}
  \caption{Критик~--Селектор~--Путь мышления в разрезе ресурсов} 
  \label{img:csw_ex} 
\end{figure}
\end{frame}

% модель дальше
\begin{frame}
\frametitle{Формальное описание системы}
\begin{figure} [h] 
  \center
  \includegraphics [scale=0.6] {SystemOverview.png}
  \caption{Формальное описание системы. TLC~--- Thinking lifecycle. $T^3$~--- модуль, построенный по принципу $T^3$.} 
  \label{img:SystemOverview.png} 
\end{figure}

\end{frame}

% вместо возвращает - вычисляет вероятность

\begin{frame}
\frametitle{Принцип функционирования}
\begin{enumerate}
	\item Процессы:
\begin{enumerate}
	\item Слабосвязанные;
	\item Короткоживущие.
\end{enumerate}
\item Вероятностные состояния;
\item Примеры процессов;
\begin{enumerate}
	\item Лексико-семантический анализ;
	\item Классификация.
\end{enumerate}
\end{enumerate}
\end{frame}





%
%					CHAPTER 3
%
% перехожу к описанию третий главы
\section[Глава 3]{Глава 3. Реализация модель TU 1.0}



\begin{frame}
\frametitle{Диаграмма действий системы TU}
\begin{figure} [h] 
  \center
  \includegraphics [scale=0.35] {ShortLefecycle}
  \label{img:ShortLefecycle}  
\end{figure}
\end{frame}


\begin{frame}
\frametitle{Пример обработки запроса "Please install Firefox"}
\begin{enumerate}
	\item Возникает инцидент, связанный с запросом от пользователя;
	\item TLC выставляет цель ProcessIncident. К каждой цели привязан набор критиков (каждый критик действует как отдельная вероятностная машина);
	\item PreliminarySplitter обрабатывает входящий запрос с помощью обработчика Link Grammar, который преобразует данное предложение в граф. Этот граф отражает результат лексического и синтаксического разбора.
\end{enumerate}
\end{frame}



\begin{frame}
\frametitle{Результат лексического разбора (Link Grammar)}
\begin{table}
	
\small
\begin{tabular} {|p{5cm}|p{5cm}|}

\hline
 
\_advmod(install, please)
\_obj(install, Firefox)
pos(install, verb)
inflection-TAG(install, .v)
tense(install, imperative)
pos(please, adv)
 & 
inflection-TAG(please, .e)
pos(., punctuation)
noun\_number(Firefox, singular
definite\-FLAG(Firefox, T)
pos(Firefox, noun) 
    \\
   \hline
\end{tabular}
\end{table}

\begin{figure} [h] 
  \center
  \includegraphics [scale=0.45] {LexicalGraph1}
  \label{img:LexicalGraph1}  
\end{figure}
\end{frame}

%Модуль содержит ограничение на количество предложений и количество слов. Длина предложения не более 1024 символов.
%Модуль содержит ограничение на количество предложений и количество слов. Длина предложения не более 1024 символов.
%1.	Граф, изображенный на Рисунке 1, подвергается разбору и преобразуется во фразы, которые группируются в предложения. В исходном примере нет идиом, но если такие встречаются, то они также будут составлять одну фразу. Отношения, начинающиеся с подчеркивания, – это связи между нодами графа, остальное – это описания конкретного слова. Ссылка может идти также не на лист, а на нод дерева, для этого алгоритм построен рекурсивно. У каждого нода графа есть именованные ссылки, в нашем случае мы берем только ссылки вида: "_subj" (подлежащие); "_obj" (дополнение); "_iobj" (косвенное дополнение); "_advmod" (наречие); "of" (принадлежность). На выходе мы получаем граф, перефразированный в рамках объектов приложения, – AnnotatedPhrase. Кроме того, будут отфильтрованы лишнее связи, также копируются свойства pos (часть речи), tense (время);


\begin{frame}
\frametitle{Вид графа запроса после обработки}
\begin{figure} [h] 
  \center
  \includegraphics [scale=0.5] {LexicalGraph2}
  \label{img:LexicalGraph2}  
\end{figure}
Используются только "\_subj" (подлежащие); "\_obj" (дополнение); "\_iobj" (косвенное дополнение); "\_advmod" (наречие); "of" (принадлежность).
\end{frame}


%1.	Результат передается в LinkParser, который ищет соответствие между словами, выявленными в разборе, и базой знаний. Например, install уже есть в базе и он получит ссылку на концепцию в базе. Каждая концепция в базе поддерживает свойства generalization, specialization. Рассмотрим концепцию install;

\begin{frame}
\frametitle{Связь с концепциями из базы знаний}
Построенный граф передается в модуль LinkParser, который устанавливает соответствие между словами, выявленными на этапе разбора запроса, и базой знаний. Например, если термин "install" уже есть в БЗ, то он получит ссылку на соответствующую концепцию в БЗ. Каждая концепция в базе поддерживает свойства generalization, specialization. Рассмотрим концепцию install.
\begin{figure} [h] 
  \center
  \includegraphics [scale=0.5] {Generalisation1}
  \label{img:Generalisation1}  
\end{figure}
\end{frame}


%1.	TLC Выставляет цель ClassifyIncident, и запускаются машины классификации, например, машина DirectInstrutctionAnalyser. Первым этапом он смотрит, есть ли действие в запросе. Во фразе оно есть, значит, итоговая вероятность не уменьшается (по умолчанию итоговая вероятность 1), иначе взвешенный результат будет меньше, так как данный критик нацелен только на действие и объект, над которым нужно его совершить. Далее ищется объект в количестве 1 и прилинкованная к действию. Если они найдены, то итоговая вероятность не уменьшается. 
%Для достижения подсчета итоговой вероятности критик содержит в себе набор правил, который обрабатывается логической машиной PLN. На вход подается граф, изображенный на Рисунке 2, на выходе получается вероятность соответствия графа правилам. Например, правила вида «Граф содержит концепцию подлежащего» или «У нода графа действия есть связь с нодом типа дополнение». В интерпретаторе правил есть поддержка прямой логики ”forward chaining”: конъюнкция, дизъюнкция, отрицания, равно, меньше;


%1.	TLC выставляет цель SearchSolution; так как решения еще нет в базе, то начинается поиск решения путем сравнения графа исходной проблемы и хранящегося в базе знаний решений. Во время поиска идет сравнение изоморфизма графов исходной проблемы и решения, хранящегося в базе знаний. В результате подсчитывается коэффициент удаленности графов – d. Если d = 0, то графы идентичны. Во время подсчета учитывается ссылка на обобщенные концепции. Например, если есть две концепции Winrar и Archive, обе ссылаются на базовую концепцию Software, то соответствие данной вершины будет 0.5, в зависимости от отдаление базовой концепции соответствие будет падать: 0,75; 0,865 и т.д.; 



\begin{frame}
\frametitle{Процесс поиска решения}
%Во время поиска идет сравнение изоморфизма графов исходной проблемы и решения, хранящегося в базе знаний. В результате подсчитывается коэффициент удаленности графов – d. Если d = 0, то графы идентичны. Во время подсчета учитывается ссылка на обобщенные концепции. Например, если есть две концепции Winrar и Archive, обе ссылаются на базовую концепцию Software, то соответствие данной вершины будет 0.5, в зависимости от отдаление базовой концепции соответствие будет падать: 0,75; 0,865 и т.д.; 
Во время поиска идет сравнение изоморфизма графов исходной проблемы и решения, хранящегося в базе знаний.
\begin{figure} [h] 
  \center
  \includegraphics [scale=0.5] {SolutionSearch1}
  \label{img:SolutionSearch1}  
\end{figure}
\end{frame}

\begin{frame}
\frametitle{Пример обработки сложного запроса}
The installation of Winrar that I got this afternoon did go wrong. During installation nothing else was running. When I tried to start Winrar I got the fault message that is attached here.
\begin{figure} [h] 
  \center
  \includegraphics [scale=0.5] {Request2}
  \label{img:Request2}  
\end{figure}
\end{frame}


\begin{frame}
\frametitle{Пример обработки запроса с желаемым состоянием}
I have Office 2010 installed, but I need office 2016.
\begin{figure} [h] 
  \center
  \includegraphics [scale=0.5] {Request3}
  \label{img:Request3}  
\end{figure}
\end{frame}


\begin{frame}
\frametitle{Полная диаграмма действий}

\begin{figure} [h] 
  \centering
  \begin{minipage}{50pt}
  	\includegraphics [scale=0.05] {LifecycleActivity}
  \end{minipage}
  \hfill
  \begin{minipage}{150pt}
  	 \includegraphics [scale=0.3] {LifecycleShortcut}
  \end{minipage}
  \label{img:LifecycleActivity}  
\end{figure}


\end{frame}

\begin{frame}
\frametitle{Ограничения использования системы}
\begin{enumerate}
	\item Имеется ограничение на длину запроса~--- не более 1024 символов;
	\item Система не может обработать логически противоречивые запросы;
	\item Используются только 5 лексических связей;
	\item Поддерживается только английский язык;
	\item Каждый критик поддерживает не более 8 логических правил.

\end{enumerate}
\end{frame}



\begin{frame}
\frametitle{Особенности системы}
\begin{enumerate}
	\item Представление большинства объектов в Базе Знаний;
	\item Расширяемость;
	\item Масштабируемость за счет Scala Akka;
	\item Концепция ShortTermMemory и LongTermMemory;
	\item Самоконтроль системой своего состояния.
\end{enumerate}
\end{frame}


%
%					CHAPTER 4
%
\section[Глава 4]{Глава 4. Экспериментальные исследования эффективности работы модели TU}



\begin{frame}
\frametitle{Результаты работы}
\begin{table}
	
\small
\begin{tabular} {|p{8cm}|p{1cm}|p{1cm}|}

\hline
%время в милисекундах
\textbf{Запрос} & TSS1 (ms) & TU (ms) \\
\hline
  Tense is kind of concept & 15000 & 385 \\
  
  \hline
  Please install Firefox  & 9000 & 859 \\
  \hline
  Browser is an object   & 20000 & 400 \\
  \hline
  Firefox is a browser   & 5000 & 659  \\
  \hline
  Install is an action    & 8000 & 486 \\
  \hline
  User miss Internet Explorer 8     & 10000 & 10589 \\
  \hline
  User needs document portal update    & 15000 & 16543 \\
  \hline
  Add new alias Host name on host that alias is wanted to: hrportal.lalala.biz IP adress on host that alias is wanted to: 322.223.333.22 Wanted Alias:    webadviser.lalala.net    & 10000 & 18432  \\ 
  \hline
  PP2C - Cisco IP communicator. Please see if you can fix the problem with the <...> & 13000 & 12343 \\ 
   \hline
   \end{tabular}
\end{table}
\end{frame}

\begin{frame}
\frametitle{Результаты работы}
\begin{table}
	
\small
\begin{tabular} {|p{8cm}|p{2cm}|}

\hline
\textbf{Категория} & \textbf{Описание} \\
\hline
 Проблема с ПО    & 64\% \\
 \hline Проблемы во время работы  &  10\% \\
  \hline Как сделать & 10\% \\
   \hline
Проблема с оборудованием  & 0\% \\
 \hline
Установить новое ПО       & 100\% \\
 \hline Проблема с печатью        & 80\% \\
  \hline Нет доступа               & 100\% \\
  \hline
   \end{tabular}
\end{table}

\end{frame}

\begin{frame}
\frametitle{Результаты моделирования}
\begin{enumerate}
 \item \textbf{Результаты моделирования:} $T_{qp}=47,9ч$ при $n=6$; $SLA=0,82$; $\alpha=0,18$;  $\alpha_n=2920$;
 \item \textbf{Результаты моделирования с помощью TU:} $T_{qp}=32,9ч$ при $n=8$; $SLA=0,96$; $\alpha=0,04$;  $\alpha_n=2920$.
\end{enumerate}
\end{frame}


%
%					Conclusions
%
\section[Заключение]{Заключение}



%%%%%%%%%%%%%%%%%%%%%%%%%%%%%%
\begin{frame}
\frametitle{Основные результаты работы}
\begin{itemize}
   \item Создана модель проблемно-ориентированной системы управления знаниями в области обслуживания информационной инфраструктуры предприятия на основе обобщения модели мышления;
  \item Представлены новая модель данных для модели мышления и оригинальный способ их хранения, более эффективный по сравнению с классическими базами данных, использующими реляционный подход;
  \item  Выполнено оригинальное исследование моделей мышления в области обслуживания информационной инфраструктуры предприятия;
  
\end{itemize}
\end{frame}


\begin{frame}
\frametitle{Основные результаты работы}
\begin{itemize}
   \item На основе модели, разработанной в диссертации, созданы архитектура системы и ее прототип; 
   \item Представлена визуализация структуры области удаленной поддержки инфраструктуры.
\end{itemize}
\end{frame}


\begin{frame}
\frametitle{Свидетельство о регистрации ПО}
\begin{figure} [h] 
  \center
  \includegraphics [scale=0.05] {RegistrationStatement}
  \label{img:RegistrationStatement}  
\end{figure}

\end{frame}


\begin{frame}
\frametitle{Акт о внедрении ПО}
\begin{figure} [h] 
  \center
  \includegraphics [scale=0.05] {ActVnedr} 
  \label{img:ActVnedr}  
\end{figure}



\end{frame}


\begin{frame}
\begin{center}
Спасибо за внимание!
\end{center}
\end{frame}

%%%%%%%%%%%%%%%%%%%%%%%%%%%%%%
%additional

\begin{frame}
\frametitle{6 уровней мышления}
\begin{enumerate}
	\item Самосознательный уровень;
	\item Саморефлексивный уровень;
	\item Рефлексивные размышления;
	\item Уровень рассуждений;
	\item Уровень обученных реакций;
	\item Инстинктивный.
\end{enumerate}
\end{frame}

\begin{frame}
\frametitle{Обработка запроса}
\begin{figure} [h] 
	\center
	\includegraphics [scale=0.7] {RequestProcesssing}
	\label{img:RequestProcesssing}  
\end{figure}
\end{frame}

\begin{frame}
\frametitle{Обработка запроса}
\begin{figure} [h] 
	\center
	\includegraphics [scale=0.7] {RequestProcesssing2}
	\label{img:RequestProcesssing2}  
\end{figure}
\end{frame}


\begin{frame}
\frametitle{Диаграмма состояний объектов}
\begin{figure} [h] 
  \center
  \includegraphics [scale=0.35] {ObjectState}
  \label{img:ObjectState}  
\end{figure}
\end{frame}


\begin{frame}
\frametitle{Специальность}
\begin{enumerate}
    \item Специальность 05.13.11 (технические науки), «Математическое и программное обеспечение вычислительных машин, комплексов и компьютерных сетей».
\end{enumerate}
\end{frame}

\begin{frame}
\frametitle{База знаний}
\begin{enumerate}
	\item Resource;
	\item Semantic Network;
	\item Rule;
	\item KLine;
	\item Neo4j.
\end{enumerate}
\end{frame}




%Структура диссертации
\begin{frame}
\frametitle{Структура диссертации}
\begin{enumerate}
\item \textbf{Введение}
  \item \textbf{Глава 1. Интеллектуальные системы регистрации и анализа проблемных ситуаций, возникающих в ИТ-инфраструктуре предприятия}
  \begin{itemize}
    \item Обзор исследований в области интеллектуальных систем регистрации и анализа проблемных ситуаций;
    \item Сравнительный анализ систем регистрации и устранения проблемных ситуаций;
    \item Сравнительный анализ методов и комплексов обработки текстов на естественном языке;
    \item Выводы по главе 1.
  \end{itemize}
 \end{enumerate}
\end{frame}

\begin{frame}
\frametitle{Структура диссертации}
\begin{enumerate}
  \item \textbf{Глава 2. Модель интеллектуальной системы принятия решений для регистрации и анализа проблемных ситуаций в ИТ-инфраструктуре предприятия}
  \begin{itemize}
    \item Построение модели Menta 0.1 с использованием деревьев принятия решений;
    \item Модель Menta 0.3, построенная с использованием генетических алгоритмов;
    \item Модель TU 1.0, основанная на модели мышления Марвина Мински;
    \item Выводы по главе 2.
  \end{itemize}
\end{enumerate}
\end{frame}

\begin{frame}
\frametitle{Структура диссертации}
\begin{enumerate}
  \item \textbf{Глава 3. Реализация модели TU 1.0 для системы интеллектуальной регистрации и устранения проблемных ситуаций}
  \begin{itemize}
    \item Архитектура системы;
    \item Модель данных TU Knowledge;
    \item Прототип системы;
    \item Выводы по главе 3.
  \end{itemize}
   \item \textbf{Глава 4. Экспериментальные исследования эффективности работы модели TU}
  \begin{itemize}
    \item Экспериментальные данные;
    \item Оценка эффективности;
    \item Результаты экспериментов;
    \item Выводы по главе 4.
  \end{itemize}

\end{enumerate}
\end{frame}

%%%%%%%%%%%%%%%%%%%%%%%%%%%%%%%%%%%%%%%
% Questions

\begin{frame}
\frametitle{Вопросы ведущей организации}
\begin{enumerate}
\scriptsize 
  \item В диссертации практически отсутствует формальная модель как постановки задачи, так и ее решения. Рассматривается Модель TU 1.0. основанная на модели мышления Марвина Мински. Теория мышления носит довольно абстрактный характер. В результате не перекинут «мостик» от этой абстрактной схемы к поставленной проблеме;
  % С замечанием согласен, отмечу, что в докладе оно учтено;

   \item Если это система обработки заявок, почему бы не создать удобный интерфейс, где человеку не надо было бы писать на естественном языке запрос? Пользователь мог бы выбрать то, что нужно и это существенно повысило бы качество системы;
%Предполагается подключение к существующим системам

\item В существующих решениях не было сказано про класс систем IDM (Identity Management System), которые делают то же самое только с продуманной системой ролей и прав. Можно было бы позиционировать систему как дополнение (модуль распознавания текста и заполнения заявки) к какой-либо системе IDM и обосновать полезность составления заявок на естественном языке;
%Применение решения - лишь часть системы. Планируется интеграция с подобными системами.
\item При описании результатов экспериментальной апробации построенной модели (подсчете доли от общего количества плодящих сообщений тех, которые были успешно обработаны) нужно было выделить в отдельную группу те инциденты, которые связаны с заявкой на техническое обслуживание и не подлежат автоматической обработке. При этом подсчет эффективности работы системы нужно было проводить, используя только тс инциденты, обработка которых была автоматизирована.
 
%Учтено в докладе

\end{enumerate}
\end{frame}

\begin{frame}
\frametitle{Вопросы оппонента профессора Райхлина В.А.}
\begin{enumerate}
\scriptsize 
  \item У нас нет сомнений в профессионализме соискателя как системного программиста. Но возникает вопрос: как ему за сравнительно короткое время удалось реализовать столь уникальную систему? Вот ответ Минского на один из вопросов интервью (Марвин Мински. Интервью журналу Discover, январь 2007): "The Emotion Machine" читается как книга размышлений о том, как человек мыслит, но разве вашим намерением не являлось изготовление мыслящей машины? «Книга - фактически план, как строить машину. Я хотел бы быть в состоянии нанять команду программистов, чтобы создать архитектуру Emotion Machine, которая может переключаться между различными видами мышления. Никто до сих пор не построил систему, которая либо имеет, либо приобретает знания о самом мышлении для того, чтобы более эффективно решать проблемы с течением времени. Если бы я мог получить пять хороших программистов, мне кажется, я мог бы построить ее в течение трех-пяти лет». Возможно, А.С. Тощеву помогло то, что за последние годы появилось множество инструментальных средств - компонентов интеллектуальных систем (Akka Concurrency, After the deadline, Google API. Link Grammar, PLN. NARS и др.), a роль M. Мински для него сыграл М.О. Таланов;
  % Вадим Абрамович сделал интересное замечание. Работа велась на протяжении 5 лет и продолжается, значит, это замечание есть возможность учесть в будущем;

  \end{enumerate}
\end{frame}

\begin{frame}
\frametitle{Вопросы оппонента профессора Райхлина В.А.}
\begin{enumerate}
\scriptsize 
 

   \item На странице 8 диссертации читаем: «На основе обобщения модели мышления, разработанной М. Мински, создана [в диссертации - В.Р.] имитационная модель ...». Минский — признанный стратег ИИ. Предлагаемые им методологии (фреймовые представления и др.) - это не просто изощренная игра ума. а попытки филосовско-гипотетического осмысления огромного личного опыта, и они всегда были чрезвычайно плодотворными. В данном случае речь может идти только об интерпретации идей Минского;
%Да, действительно, речь идет только об интерпретации.

\item Материал главы 3 - основная содержательная часть диссертации. Но написана эта глава в стиле технического отчета. Не дается необходимых пояснений, что может явиться причиной множества ненужных диссертанту вопросов: как реализовано то или иное и почему именно так, а не иначе. Ничего не говорится о принятых ограничениях. А они, несомненно были. Объяснение - одна из важнейших функций науки. Что не понято, то не воспринято. И если автор как пионер реализации идей Минского хочет добиться признания со стороны научной общественности, ему в будущем будет полезно развить главу 3 в отдельную монографию, где будет все объяснено;
%Хорошее замечание, Вадим Абрамович дал вектор дальнейшего развития.
\end{enumerate}
\end{frame}


\begin{frame}
\frametitle{Вопросы оппонента Полякова В.Н.}
\begin{enumerate}
\scriptsize 
 

   \item В своем литобзоре диссертант ссылается на ключевую для его работы систему Relex. Приведены две ссылки на "электронные" источники в сети Интернет. В то же время существует ссылки на «бумажные» источники:
Hart. I); В Goertzel (2008). OpenCog: A Software Framework for Integrative Artificial General Intelligence (PDF). Proceedings of the First AGI Conference. Gbooks; Goertz.el, В., Ikle, M., Goertzel, I.F., Heljakka, A. Probabilistic Logic Networks, A Comprehensive Framework for Uncertain Inference, Springer, 2009, VIII, 336 p„ Hardcover ISBN 978-0-387-76871-7.
Эти ссылки автор диссертации не приводит, хотя известно, что время жизни электронных ссылок, особенно в сети Wikipedia, непредсказуемо.;
%Хотел быть современным, использовать электронные ресурсы, но не везде удалось.

\item Таблица 1.4 (Сравнительный анализ функциональности существующих решений) приведена в разделе «1.4 Выводы по главе 1», хотя ее место - в теле первой главы. Получается так, что выводы заканчиваются этой таблицей, без сопроводительного текста в конце. Кроме того, в таблице сравниваются три решения: HP Open View, ServiceNOW и IBM Watson, а выбор делается в пользу системы OpenCog Relex. Ни слова о системе OpenCog Relex в таблице нет.;
%OpenCog Relex относится к анализу обработчиков естественного языка. А IBM Watson, ServiceNOW к системе приема инцидентов от пользователя.
\item Автор не ссылается на собственные работы. Список публикаций автора приведен в разделе «Публикации» на стр. 11. Далее автор приводит краткое описание своих работ в виде одного абзаца. В теле диссертации, особенно в главах 2 и 3, ссылок на эти работы не обнаружено;
%



\end{enumerate}
\end{frame}


\begin{frame}
\frametitle{Вопросы оппонента Полякова В.Н.}
\begin{enumerate}
\scriptsize 
 \item Приведено мало статистики по работе системы (см. табл. 4.2., Результаты сравнения с работой специалиста, стр. 88). В самой диссертации рассмотрено всего 10 случаев, из-за этого точность определения эффективности предложенной модели составляет 10 \%;
%С замечанием согласен, отмечу, что часть данных является собственностью компании, и публиковать их я не имел разрешения

\item В таблице 4.2. (там же), данные по времени реакции специалиста приведена с точность до секунды, в то время как данные о работе программы приведены с точность до миллисекунды. Кроме того, нет четких данных о том, как фиксируется начало и конец обработки заявки (в случае специалиста-человека и системы). Нет данных о том, какие вычислительные ресурсы были направлены на обработку текстового запросы пользователя, а какие — на непосредственное устранение проблемы. Все это, в целом, порождает вопрос о тщательности проработки методической основы при получении экспериментальных данных. Думается, что при более глубокой отработке методики эксперимента преимущество предлагаемой модели стало бы еще более очевидным;
%В.Н. Поляков очень внимательно изучил диссертацию. Сделал очень интересные замечания и обратил внимания на тонкие места. Из систем, в которых брали данные было округление до секунд. По ресурсам использовались те же.
\item Нет абсолютных данных для таблицы 4.3 (Описание экспериментальных данных, стр. 90). Приведены только относительные в процентах . Из-за этого невозможно полностью проанализировать и верифицировать представленные результаты.
%Их можно вычислить.

\end{enumerate}
\end{frame}


%%%%%%%%%%%%%%%%%%%%%%%%%%%%%%%%%%%%%%%


\end{document} 