{\aim} диссертации является разработка интеллектуальной системы повышения эффективности деятельности ИТ-службы предприятия (ИТ~--- информационные технологии). \par
{\scope}~--- разработка cистем управления базами данных и знаний.\par
{\subject}  является процесс регистрации и устранения проблемных ситуаций, возникающих в ИТ-инфраструктуре предприятия.\par
{\methods}~--- теоретические методы: метод идеализации, метод формализации; специальные методы: системное моделирование; экспериментальные методы: метод наблюдений, проведение экспериментов.\par 
Для достижения поставленной цели были решены следующие {\tasks}:
\begin{enumerate}
  \item Провести анализ систем управления базами знаний в области поддержки информационной инфраструктуры предприятия;
  \item Разработать и построить модель проблемно-ориентированной системы управления базой знаний для принятия решений и оптимизации процесса регистрации, анализа и обработки запросов пользователей в области обслуживания информационной инфраструктуры предприятия;
  \item На основе построенной модели разработать архитектуру и создать прототип интеллектуальной системы повышения эффективности деятельности ИТ-службы предприятия;
  \item Провести апробацию прототипа на тестовых данных.
\end{enumerate}

\defpositions
\begin{enumerate}
  \item Результаты анализа систем управления базами знаний в области поддержки ИТ-инфраструктуры предприятия;
  \item Построенная модель проблемно-ориентированной системы управления базой знаний и оптимизации процессов обработки запросов пользователей в области обслуживания ИТ-инфраструктуры предприятия;
  \item Созданный прототип программной реализации модели проблемно-ориентированной системы управления базой знаний и оптимизации обработки запросов пользователей в области обслуживания ИТ-инфраструктуры предприятия;
  \item Результаты апробации прототипа проблемно-ориентированной системы управления, принятия решений и оптимизации деятельности на контрольных примерах и анализ ее результатов.
\end{enumerate}

\novelty\ проведенного исследования состоит в следующем:
\begin{enumerate}
  \item На основе обобщения модели мышления, разработанной М. Мински, создана имитационная модель проблемно-ориентированной системы управления, принятия решений в области обслуживания ИТ-инфраструктуры предприятия;
  \item Выполнено исследование возможностей использования моделей мышления применительно к области обслуживания информационной инфраструктуры предприятия;
  \item Представлены новая схема данных и оригинальный способ хранения данных для построенной модели мышления, эффективный по сравнению со стандартными способами хранения (такими, как реляционные базы данных);
  \item На основе построенного обобщения модели мышления Мински созданы архитектура системы обслуживания информационной инфраструктуры предприятия и программный прототип этой системы.
\end{enumerate}

\influence\ 
Система, разработанная в рамках данной диссертации, носит значимый практический характер. Идея работы зародилась под влиянием производственных проблем в ИТ-отрасли, с которыми автор сталкивался ежедневно в процессе разрешения различных инцидентов, возникающих в деятельности службы технической поддержки \icl~--- одном из крупнейших системообразующих предприятий ИТ-отрасли Республики Татарстан. Поэтому было необходимо выработать глубокое понимание конкретной предметной области, чтобы выбрать приемлемое решение, получившее практическое применение при организации информационной поддержки ИТ-инфраструктуры конкретного предприятия. \par
\reliability\ полученных научных результатов и выработанных практических рекомендаций базируется на корректной постановке общих и частных рассматриваемых задач,  использовании известных фундаментальных теоретических положений системного анализа, достаточном объёме данных, использованных при статистическом моделировании, и широком экспериментальном материале, использованном для численных оценок достижимых качественных показателей. \par 
Исследования, проведенные в диссертации, соответствуют паспорту специальности 05.13.11~--- Математическое и программное обеспечение вычислительных машин, комплексов и компьютерных сетей, сопоставление приведено в таблице \ref{ResearchDescription}.

\begin{longtable}{|p{7cm}|p{9cm}|}
 \caption[Сопоставление направлений исследований, предусмотренных специальностью 05.13.11, и результатов, полученных в диссертации]{Сопоставление направлений исследований предусмотренных специальностью 05.13.11, и результатов, полученных в диссертации}\label{ResearchDescription} \\ 
 \hline
 
 \multicolumn{1}{|c|}{\textbf{Направление исследования}} & \multicolumn{1}{c|}{\textbf{Результат работы}}  \\ \hline 
\endfirsthead
\multicolumn{2}{c}%
{{\bfseries \tablename\ \thetable{} -- продолжение}} \\
\hline \multicolumn{1}{|c|}{\textbf{Направление исследования}} &
\multicolumn{1}{c|}{\textbf{Результат работы}}  \\ \hline 
\endhead
\endfoot

\hline \hline
\endlastfoot
\hline
   Языки программирования и системы программирования, семантика программ & Разработана семантическая модель организации хранения знаний \\
   \hline
  Системы управления базами данных и знаний & Разработан прототип Thinking Understanding (TU) системы хранения знаний и принятия решений в сфере поддержки ИТ-инфраструктуры предприятия, который был испытан на модельных данных\\
   \hline
   Модели и методы создания программ и программных систем для параллельной и распределенной обработки данных, языки и инструментальные средства параллельного программирования & Разработан метод параллельной обработки экспертной информации c возможностью обучения при помощи прототипа TU \\
  \end{longtable}


\probation\
 Основные результаты диссертационной работы докладывались на следующих конференциях:
\begin{itemize}
	\item Десятая молодежная научная школа-конференция \quoted{Лобачевские чтения~---2011}. Казань, 31 октября~--~4 ноября 2011 года;
	\item Международная конференция "3rd World Conference on Information Technology (WCIT-2012)". Barcelona, 14~--~16 November 2012, Spain; 
	\item II Международная конференция «Искусственный интеллект и естественный язык (AINL-2013)». Санкт-Петербург, 17~--~18 мая 2013 года;
	\item VI Международная научно-практическая конференция «Электронная Казань 2014». Казань, 22~--~24 апреля 2014 года;
	\item XVI Всероссийская научная конференция «Электронные библиотеки: перспективные методы и технологии, электронные коллекции (RCDL-2014)». Дубна, 13~--~16 октября 2014 года;
	\item Семинары по программной инженерии "All-Kazan Software Engineering Seminar (AKSES-2015)". Kazan, 9 April 2015;
	\item Международная конференция "Agents and multi-agent systems: technologies and applications (AMSTA-2015)". Sorento, 17~--~19 June 2015, Italy.
\end{itemize} \par
Практическая апробация результатов работы проводилась на выгрузке инцидентов из системы регистрации запросов службы технической поддержки ИТ-инфраструктуры \icl. Созданная система показала требуемые результаты обработки данной информации (процент успешно обработанных запросов составил более чем 30\%). \par
\contribution\ Автор исследовал целевую область: провел анализ запросов пользователей и классифицировал их; построил модель целевой области и выявил возможности оптимизации. Данные для исследования (выгрузка из систем регистрации запросов пользователей \iclshort) были получены при помощи Крехова А.В.  Совместно с Талановым М.О. автор создал базовую архитектуру системы. Автор разработал компоненты системы, провел испытание системы на экспериментальных данных и отладил ее работу. \par
\publications\ Основные результаты по теме диссертации изложены в 10 печатных изданиях  \cite{Lobachevskii, WCIT-2012,  ISGZ, IJSE-1, IJSE-2, RCDL-2014, AMSTA-2015, VAK-1, EB-1, EB-2}, из которых статьи \cite{RCDL-2014, AMSTA-2015} проиндексированы в БД Scopus и входят в перечень журналов ВАК РФ, статья \cite{AMSTA-2015} также проиндексирована в БД Web of Science, работа \cite{VAK-1} опубликована в журнале из перечня ВАК РФ, статья \cite{ISGZ} проиндексирована в БД РИНЦ, работы \cite{Lobachevskii, WCIT-2012, ISGZ} опубликованы в материалах международных и всероссийских конференций, статьи \cite{IJSE-1, IJSE-2} опубликованы в международном журнале "International Journal of Synthetic Emotions"\,, входящем в индекс ACM и отмеченном наградой "IGI Global’s Seventh Annual Excellence in Research Journal Awards".



