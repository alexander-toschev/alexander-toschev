%%% Основные сведения %%%
\newcommand{\thesisAuthor}             % Диссертация, ФИО автора
{%
    \texorpdfstring{% \texorpdfstring takes two arguments and uses the first for (La)TeX and the second for pdf
        Тощев Александр Сергеевич% так будет отображаться на титульном листе или в тексте, где будет использоваться перемная
    }{%
        Тощев, Александр Сергеевич% эта запись для свойств pdf-файла. В таком виде, если pdf будет обработан программами для сбора библиографических сведений, будет правильно представлена фамилия.
    }%
}
\newcommand{\thesisUdk}                % Диссертация, УДК
{004.8}
\newcommand{\thesisTitle}              % Диссертация, название
{\texorpdfstring{\MakeUppercase{Интеллектуальная система повышения эффективности ИТ-службы предприятия}}{Интеллектуальная система повышения эффективности ИТ-службы предприятия}}
\newcommand{\thesisSpecialtyNumber}    % Диссертация, специальность, номер
{\texorpdfstring{{05.13.11}}{05.13.11}}
\newcommand{\thesisSpecialtyTitle}     % Диссертация, специальность, название
{\texorpdfstring{Математическое и программное обеспечение вычислительных машин, комплексов и компьютерных сетей}{Математическое и программное обеспечение вычислительных машин, комплексов и компьютерных сетей}}
\newcommand{\thesisDegree}             % Диссертация, научная степень
{кандидата технических наук}
\newcommand{\thesisCity}               % Диссертация, город защиты
{Казань}
\newcommand{\thesisYear}               % Диссертация, год защиты
{2016}
\newcommand{\thesisOrganization}       % Диссертация, организация
{Казанский (Приволжский) федеральный университет}

\newcommand{\thesisInOrganization}       % Диссертация, организация в предложном падеже: Работа выполнена в ...
{Институте математики и механики (ИММ) им. Н.И. Лобачевского Казанского (Приволжского) федерального университета (КФУ)}

\newcommand{\supervisorFio}            % Научный руководитель, ФИО
{Елизаров Александр Михайлович}
\newcommand{\supervisorRegalia}        % Научный руководитель, регалии
{доктор физико-математических наук, профессор}
\newcommand{\supervisorRegaliaSecond}
{заслуженный деятель науки РТ, \\ зав. кафедрой дифференциальных уравнений \\ Института математики и механики им. Н.И. Лобачевского \\ Казанского (Приволжского) федерального университета}   
\newcommand{\supervisorRegaliaSecondShort}{заслуженный деятель науки РТ,}
\newcommand{\supervisorRegaliaSynopsisSecond} % for synopsis, because we can use acronym early then in main work
{засл. деятель науки РТ, зав. кафедрой дифференциальных уравнений ИММ им. Н.И. Лобачевского КФУ}   
\newcommand{\opponentOneFio}           % Оппонент 1, ФИО
{Райхлин Вадим Абрамович}
\newcommand{\opponentOneRegalia}       % Оппонент 1, регалии
{доктор физико-математических наук, профессор}
\newcommand{\opponentOneJobPlace}      % Оппонент 1, место работы
{Казанский национальный исследовательский технический университет им. А.Н. Туполева (КНИТУ-КАИ)}
\newcommand{\opponentOneJobPost}       % Оппонент 1, должность
{доцент кафедры АСУ}

\newcommand{\opponentTwoFio}           % Оппонент 2, ФИО
{Поляков Владимир Николаевич}
\newcommand{\opponentTwoRegalia}       % Оппонент 2, регалии
{кандидат технических наук, доцент}
\newcommand{\opponentTwoJobPlace}      % Оппонент 2, место работы
{Национальный исследовательский технологический университет МИСиС}
\newcommand{\opponentTwoJobPost}       % Оппонент 2, должность
{профессор кафедры компьютерных систем}

\newcommand{\leadingOrganizationTitle} % Ведущая организация, дополнительные строки
{Федеральный исследовательский центр Информатики и Управления Российской академии наук (ФИЦ ИУ РАН), г. Москва}

\newcommand{\defenseDate}              % Защита, дата
{}
\newcommand{\defenseCouncilNumber}     % Защита, номер диссертационного совета
{Д 212.081.35}
\newcommand{\defenseCouncilTitle}      % Защита, учреждение диссертационного совета
{Казанского (Приволжского) федерального университета}
\newcommand{\defenseCouncilAddress}    % Защита, адрес учреждение диссертационного совета%%% Основные сведения %%%

%%% Основные сведения %%%
\newcommand{\thesisAuthor}             % Диссертация, ФИО автора
{Тощев Александр Сергеевич}
\newcommand{\thesisUdk}                % Диссертация, УДК
{004.8}
\newcommand{\thesisTitle}              % Диссертация, название
{Интеллектуальная система повышения эффективности IT службы предприятия}

% Обобщение Модели Минского (обобщенная)
\newcommand{\thesisSpecialtyNumber}    % Диссертация, специальность, номер
{05.13.01}
\newcommand{\thesisSpecialtyTitle}     % Диссертация, специальность, название
{Системный анализ, управление и обработка информации (информационные технологии)}
\newcommand{\thesisDegree}             % Диссертация, научная степень
{Кандидат технических наук}
\newcommand{\thesisCity}               % Диссертация, город защиты
{Казань}
\newcommand{\thesisYear}               % Диссертация, год защиты
{2015}
\newcommand{\thesisOrganization}       % Диссертация, организация
{Казанский (Приволжский) Федеральный Университет\par}

\newcommand{\supervisorFio}            % Научный руководитель, ФИО
{Елизаров А.М.}
\newcommand{\supervisorRegalia}        % Научный руководитель, регалии
{доктор физико-математических наук, профессор}

\newcommand{\opponentOneFio}           % Оппонент 1, ФИО
{Соловьев Валерий Дмитриевич}
\newcommand{\opponentOneRegalia}       % Оппонент 1, регалии
{доктор физико-математических наук, профессор}
\newcommand{\opponentOneJobPlace}      % Оппонент 1, место работы
{Казанский (Приволжский) Федеральный Университет, Институт филологии и межкультурной коммуникации им. Льва Толстого}
\newcommand{\opponentOneJobPost}       % Оппонент 1, должность
{Ведущий научный сотрудник}
\newcommand{\opponentTwoFio}           % Оппонент 2, ФИО
{\todo{Оппонет2}}
\newcommand{\opponentTwoRegalia}       % Оппонент 2, регалии
{кандидат технических наук}
\newcommand{\opponentTwoJobPlace}      % Оппонент 2, место работы
{\todo{Оппонет2}}
\newcommand{\opponentTwoJobPost}       % Оппонент 2, должность
{\todo{Оппонет2}}

\newcommand{\leadingOrganizationTitle} % Ведущая организация, дополнительные строки
{Федеральное государственное бюджетное учреждение науки Институт проблем информатики Российской академии наук}

\newcommand{\defenseDate}              % Защита, дата
{\todo{DD mmmmmmmm YYYY~г.~в~XX часов}}
\newcommand{\defenseCouncilNumber}     % Защита, номер диссертационного совета
{\todo{NN}}
\newcommand{\defenseCouncilTitle}      % Защита, учреждение диссертационного совета
{\todo{Название учреждения}}
\newcommand{\defenseCouncilAddress}    % Защита, адрес учреждение диссертационного совета
{\todo{Адрес}}

\newcommand{\defenseSecretaryFio}      % Секретарь диссертационного совета, ФИО
{\todo{Фамилия Имя Отчество}}
\newcommand{\defenseSecretaryRegalia}  % Секретарь диссертационного совета, регалии
{\todo{д-р~физ.-мат. наук}}            % Для сокращений есть ГОСТы, например: ГОСТ Р 7.0.12-2011 + http://base.garant.ru/179724/#block_30000

\newcommand{\synopsisLibrary}          % Автореферат, название библиотеки
{\todo{Название библиотеки}}
\newcommand{\synopsisDate}             % Автореферат, дата рассылки
{\todo{DD mmmmmmmm YYYY года}}
{Казань, ул. Кремлевская 18}

\newcommand{\defenseSecretaryFio}      % Секретарь диссертационного совета, ФИО
{Еникеев Арслан Ильясович}
\newcommand{\defenseSecretaryRegalia}  % Секретарь диссертационного совета, регалии
{канд. физ.-мат. наук, доцент}            % Для сокращений есть ГОСТы, например: ГОСТ Р 7.0.12-2011 + http://base.garant.ru/179724/#block_30000

\newcommand{\synopsisLibrary}          % Автореферат, название библиотеки
{}
\newcommand{\synopsisDate}             % Автореферат, дата рассылки
{}

\newcommand{\keywords}%                 % Ключевые слова для метаданных PDF диссертации и автореферата
{}


\newcommand{\icl}
{ОАО «АйСиЭл КПО-ВС (г. Казань)»}   

\newcommand{\iclshort}
{ICL}

\newcommand{\triplet}
{Критик~--~Селектор~--~Образ мышления}   

\newcommand{\tripleta}
{Критика~--~Селектора~--~Образа мышления}   

\newcommand{\etc}
{и т.~д.}   

\newcommand{\quoted}[1]{«#1»}

\newcommand{\comma}{,}

\newcommand{\tripletshort}
{${T^3}$}   
