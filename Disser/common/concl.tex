%% Согласно ГОСТ Р 7.0.11-2011:
%% 5.3.3 В заключении диссертации излагают итоги выполненного исследования, рекомендации, перспективы дальнейшей разработки темы.
%% 9.2.3 В заключении автореферата диссертации излагают итоги данного исследования, рекомендации и перспективы дальнейшей разработки темы.

Решены следующие задачи и достигнуты следующие результаты.
\begin{enumerate}
  \item Создана модель проблемно-ориентированной системы управления знаниями в области обслуживания информационной инфраструктуры предприятия на основе обобщения модели мышления;
  \item Представлены новая модель данных для модели мышления и оригинальный способ их хранения, более эффективный по сравнению с классическими базами данных, использующими реляционный подход;
  \item Выполнено оригинальное исследование моделей мышления в области обслуживания информационной инфраструктуры предприятия;
  \item На основе модели, разработанной в диссертации, созданы архитектура системы и ее прототип; 
  \item Система, разработанная в рамках данной работы, включает в себя инновационные методы и алгоритмы поддержки принятия решений, использует обобщенную модель мышления Мински;
  \item Представлена визуализация структуры области удаленной поддержки инфраструктуры.
\end{enumerate}

Представленные в диссертации модель мышления, ее архитектура и реализация являются уникальными~--- на данный момент времени это единственная реализация модели мышления Мински. \par
Система, разработанная в диссертации, не является узкоспециализированной и подходит для других областей, где требуется организация базы знаний, например, при постановке медицинского диагноза, чтобы отбросить ложные диагнозы. \par
В области диагностики проблем можно обучить систему сведениям об узлах автомобиля и проблемах, с ними связанных, признаках этих проблем и способах их устранения. \par
Работа выполнена  частично за счет средств субсидии, выделенной Казанскому федеральному университету для выполнения государственного задания в сфере научной деятельности, проекты 1.2368.2017, «Бюджет 17-97».




