\subsection*{Общая характеристика работы}

\newcommand{\actuality}{\underline{\textbf{Актуальность темы.}}}
\newcommand{\aim}{\underline{\textbf{Целью}}}
\newcommand{\tasks}{\underline{\textbf{задачи}}}
\newcommand{\scope}{\underline{\textbf{Область исследования}}}
\newcommand{\subject}{\underline{\textbf{Предметом исследования}}}

\newcommand{\defpositions}{\underline{\textbf{Основные положения, выносимые на~защиту:}}}
\newcommand{\novelty}{\underline{\textbf{Научная новизна:}}}
\newcommand{\influence}{\underline{\textbf{Практическая значимость}}}
\newcommand{\reliability}{\underline{\textbf{Достоверность}}}
\newcommand{\probation}{\underline{\textbf{Апробация работы.}}}
\newcommand{\contribution}{\underline{\textbf{Личный вклад.}}}
\newcommand{\publications}{\underline{\textbf{Публикации.}}}


{\aim} работы является разработка интеллектуальной системы повышения эффективности деятельности ИТ-службы предприятия. \par
{\scope}~--- разработка методов и алгоритмов решения задач системного анализа, оптимизации, управления, принятия решений и обработки информации в ИТ-отрасли.\par
{\subject}  является процесс регистрации и устранения проблемных ситуаций, возникающих в ИТ-инфраструктуре предприятия.\par

Для достижения поставленной цели необходимо было решить следующие {\tasks}:
\begin{enumerate}
  \item Провести теоретико-множественный и теоретико-информационный анализ сложных систем в области поддержки информационной инфраструктуры;
  \item Создать модель целевой области;
  \item Исследовать модели мышления и выбрать наиболее подходящую;
  \item На основе выбранной модели мышления разработать модель проблемно-ориентированной системы управления, принятия решений и оптимизации процесса принятия, анализа и обработки запросов пользователя в области обслуживания информационной структуры предприятия;
  \item Создать архитектуру приложения на основе модели;
  \item Реализовать на основе этой архитектуры прототип интеллектуальной вопросно-ответной системы повышения эффективности деятельности ИТ-службы предприятия;
  \item Провести апробацию прототипа на тестовых данных.
\end{enumerate}

\defpositions
\begin{enumerate}
  \item Теоретико-множественный и теоретико-информационный анализ сложных систем в области поддержки информационной инфраструктуры;
  \item Построенная модель проблемно-ориентированной системы управления, принятия решений и оптимизации технических объектов в области обслуживания информационной инфраструктуры;
  \item Созданный прототип программной реализации модели проблемно-ориентированной системы управления, принятия решений и оптимизации обработки запросов пользователя в области обслуживания информационной инфраструктуры;
  \item Апробация прототипа проблемно-ориентированной системы управления, принятия решений и оптимизации деятельности на контрольных примерах и анализ ее результатов.
\end{enumerate}

\novelty проведенного исследования состоит в следующем:
\begin{enumerate}
  \item Создана модель проблемно-ориентированной системы управления, принятия решений в области обслуживания информационной структуры предприятия на основе модели мышления;
  \item Представлены новая модель данных для модели мышления и оригинальный способ хранения для этой модели, эффективный по сравнению с другими базами данных;
  \item Выполнено оригинальное исследование моделей мышления применительно к области обслуживания информационной структуры предприятия;
  \item На основе модели мышления Мински созданы архитектура системы обслуживания информационной структуры предприятия и программный прототип этой системы.
\end{enumerate}

\influence\ 
Система, разработанная в рамках данной диссертации носит значимый практический характер. Идея работы зародилась под влиянием производственных проблем в ИТ-отрасли, с которыми автор сталкивался каждый день в процессе разрешения различных инцидентов, возникающих в деятельности службы технической поддержки \icl~--- одном из крупнейших системообразующих предприятий ИТ-области Республике Татарстан. Поэтому было необходимо выработать глубокое понимание конкретной предметной области, чтобы выбрать приемлемое решение, получившее практическое применение в работе на проекте поддержки крупной сети продуктовых магазинов. \par
\reliability\ научных исследований и практических рекомендаций
базируется на корректной постановке общих и частных рассматриваемых задач,  использовании известных фундаментальных теоретических положений системного анализа, достаточном объёме данных, использованных при статистическом моделировании, и широком экспериментальном материале, использованном для численных оценок достижимых качественных показателей. \par 
Исследования, проведенные в диссертации, соответствуют паспорту специальности 05.13.01~--- Системный анализ, управление и обработка информации, сопоставление приведено в таблице \ref{ResearchDescription}.

\begin{longtable}{|p{7cm}|p{9cm}|}
 \caption[Сопоставление направлений исследований в рамках специальности 05.13.01 и исследований, проведенных в диссертации]{Сопоставление направлений исследований в рамках специальности 05.13.01 и исследований, проведенных в диссертации}\label{ResearchDescription} \\ 
 \hline
 
 \multicolumn{1}{|c|}{\textbf{Направление исследования}} & \multicolumn{1}{c|}{\textbf{Результат работы}}  \\ \hline 
\endfirsthead
\multicolumn{2}{c}%
{{\bfseries \tablename\ \thetable{} -- продолжение}} \\
\hline \multicolumn{1}{|c|}{\textbf{Направление исследования}} &
\multicolumn{1}{c|}{\textbf{Результат работы}}  \\ \hline 
\endhead
\endfoot

\hline \hline
\endlastfoot
\hline
   Разработка критериев и моделей описания и оценки эффективности решения задач системного анализа, оптимизации, управления, принятия решений и обработки информации & В рамках работы была разработана модель системы принятия решения и обработки информации в области решения запросов пользователя на естественном языке. \\
   \hline
   Разработка проблемно-ориентированных систем управления, принятия решений и оптимизации технических объектов & По модели, разработанной в предыдущем пункте был разработан прототип системы принятия решения Thinking Understanding, который был испытан на модельных данных.\\
   \hline
    Теоретико-множественный и теоретико-информационный анализ сложных систем & В рамках работы был проведен комплексный анализ области поддержки программного обеспечения, с помощью которого была построена система данной области и выделены участки для оптимизации принятия решений.\\
  \hline
  Методы и алгоритмы интеллектуальной поддержки при принятии управленческих решений в технических системах & Система, разработанная в рамках данной работы в включает в себя инновационные методы и алгоритмы поддержки принятия решений, использующих в своей основе модель мышления на базе модели мышления Человека, описанной в книге Марвина Мински. \\ 
  \hline
  
\end{longtable}


\probation\
 Основные результаты диссертационной работы докладывались на следующих конференциях:
\begin{itemize}
	\item Десятая молодежная научная школа-конференция "Лобачевские чтения~---2011. Казань, 31 октября~--4 ноября 2011";
	\item 3rd World Conference on Information Technology (WCIT-2012); 
	\item Искусственный интеллект и естественный язык (AINL-2013);
	\item Электронная Казань~--- 2014;
	\item Электронные библиотеки: перспективные методы и технологии, электронные коллекции (RCDL-2014);
	\item Agents and multi-agent systems: technologies and applications (AMSTA-2015).
\end{itemize}
Практическая апробация результатов работы проводилась на выгрузке инцидентов из системы регистрации запросов службы технической поддержки ИТ-инфраструктуры \icl. Созданная система показала требуемые результаты (процент успешно обработанных запросов более чем 30\%) обработки данной информации. \par
\contribution\ Автор исследовал целевую область: проводил анализ запросов пользователей и классифицировал их, вместе с Талановым Максимом Олеговичем изучал модель мышления Марвина Мински; создавал базовую архитектуру систему; вместе с Талановым Максимом Олеговичем проводил разработку компонентов модели, адаптируя теорию Марвина Мински. Автор проводил испытание системы на целевых запросах; отлаживал работу системы. \par
\publications\ Основные результаты по теме диссертации изложены в 9 печатных изданиях  \cite{Lobachevskii},\cite{WCIT-2012},\cite{AINL-2013},\cite{ISGZ}, \cite{IJSE-1}, \cite{IJSE-2}, \cite{RCDL-2014}, \cite{AMSTA-2015}, \cite{VAK-1}, из которых статьи \cite{RCDL-2014},\cite{AMSTA-2015} проиндексированы в БД Scopus, статья \cite{AMSTA-2015} проиндексирована в БД Web Of Science, работа \cite{VAK-1} опубликована в журнале из списка ВАК, статья  \cite{ISGZ} проиндексирована в БД РИНЦ, работы \cite{Lobachevskii},\cite{WCIT-2012},\cite{AINL-2013},\cite{ISGZ} опубликованы в материалах международных и всероссийских конференций.




%\newpage
\renewcommand{\refname}{\large Публикации автора по теме диссертации}

%\insertbiblioauthor                          % Подключаем Bib-базы
\insertbiblioall

