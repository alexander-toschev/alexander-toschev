\chapter*{Введение}							% Заголовок
\addcontentsline{toc}{chapter}{Введение}	% Добавляем его в оглавление

В настоящее время все более популярным и распространенным становится процесс передачи функций поддержки информационной инфраструктуры (далее~--- ИТ-инфраструктуры) предприятия какой-либо внешней компании (см., например, \cite{StartToOutsource}). Это явление стало называться «ИТ-аутсорсинг» (от анг. ”out source”~--– вне источника). С развитием рынка информационных систем компаниям становится невыгодно держать свой штат службы поддержки, и они отдают эти функции сторонней компании (см. \cite{OutsourceEff}). В некоторых случаях передаются все функции поддержки пользователей: будь-то заявка на ремонт компьютера или же информационный запрос, возникающий из-за простого незнания внутренних процессов компании. В результате создается единая точка входа для пользователей, поддерживаемая сторонней компанией \cite{OutsourceSD}. Обобщая, можно сказать, что на аутсорсинг передают все, что возможно: управление персоналом, уборку помещений, обеспечение питанием, разработку программного обеспечения (далее~--- ПО) (см., например, \cite{OutsourceSoft}) \etc \par
В некоторых областях, например, в области информационных технологий (ИТ) за счет аутсорсинга экономия средств предприятия достигает 30\% (по данным Gartner \cite{OutsourceIT}).
Из-за возросшей популярности бизнеса по аутсорсингу именно в ИТ-области и появления большого количества компаний возникла сильная конкуренция \cite{AUTOS-1}, что привело к снижению цен на услуги и потребовало сокращения издержек компаний. Для поиска путей оптимизации издержек было необходимо применение методов системного анализа для решения сложившихся проблем \cite{AUTOM-1}. Также было отмечено падение рентабельности бизнеса как минимум для малых компаний \cite{OUTSOURCE-RENT}, \cite{OutsourceEff}. В контексте оптимизации издержек в настоящей диссертации рассматриваются модель области, модель системы и ее реализация, которая повышает эффективность работы специалиста технической поддержки (далее специалист) путем частичной (в некоторых случаях, полной) автоматизации обработки инцидентов (случаев, происшествий)  \cite{SDAUTOM}, начиная с разбора запросов, сформулированных на естественном языке, и заканчивая применением найденного решения. \par
Главным требованием к системе повышения эффективности ИТ-службы предприятия является замена части функций, которые сейчас выполняют специалисты:
\begin{enumerate}
  \item Обработка запросов на естественном языке~--- эта функция широко востребована и в системах анализа проблем пользователя с построением статистики «Удовлетворенность пользователя программным продуктом» \cite{TUTUB-1}. Общее понимание проблемы зависит от понимания языка, на котором общаются специалисты;
  \item Возможность обучения. Такая возможность системы позволяет упростить ее эксплуатацию и расширение. По данным исследования \cite{LEARN-1}, возможность обучения очень важна для любой интеллектуальной системы, включая системы управления роботами. Обучение обеспечивает системе большие гибкость и универсальность;
  \item Общение со специалистом. Поддержание диалога (коммуникации)~--- необходимое условие для обучения. Кроме того, социальная функция~--- неотъемлемая часть интеллектуальных систем (см., например, \cite{LEARN-2});
  \item Проведение логических рассуждений (возможность размышлять): аналогия, дедукция, индукция~--- умение обобщить решение одной проблемы и, экстраполируя его, применить для решения других. Иными словами, это возможность для системы принять правильное решение. Например, принятие решений широко используется в интеллектуальных системах управления производством  \cite{LEARN-3}.
\end{enumerate}

На данный момент времени многие компании ведут в различных областях разработку подобных систем, обладающих свойствами, описанными выше. Системы такого класса также называются \textit{вопросно-ответными}. Примером является набирающая популярность IBM Watson \cite{WATSON-PO}, \cite{WATSON-PTOP} (которая является коммерческой и закрытой, информации о ее внутреннем устройстве мало). Другой пример~--- компания HP использует результаты исследования \cite{TUTUB-2} для автоматического определения проблем и степени удовлетворенности пользователей из отчетов об использовании программного обеспечения. Также эта компания работает над автоматическим решением проблем (как описано выше). \par

В настоящей диссертации представлены результаты и апробации создания вопросно-ответной системы на основе исследования целевой области (удаленная поддержка информационной структуры предприятия) и построения модели системы. Акцент был сделан на создании интеллектуальной системы для решения широкого круга проблем. \par

Следует отметить, что большинство проблем, которые решает удаленная служба поддержки информационной структуры предприятия, носит достаточно тривиальный характер (по данным компании \icl): установить приложение; переустановить приложение; решить проблему с доступом к тому или иному ресурсу.
Названные проблемы решают специалисты технической поддержки, которая обычно делится на несколько линий по уровню умения специалистов. Каждая линия поддержки представлена своим классом специалистов. В среднем команда, обслуживающая одного заказчика, насчитывает около 60 человек. Процентное соотношение специалистов разных линий поддержки отображено на рисунке \ref{img:ITSMTeamComposition}.

\begin{table} [htbp]
  \centering
  \parbox{15cm}{\caption{Описание работы специалистов различных уровней поддержки}\label{TSSDescription}}
%  \begin{center}
  \begin{tabular}{| p{7cm} | p{7cm} |}
    \hline
\textbf{Уровень} & \textbf{Описание} \\
  \hline
    

Первая линия	& Решение уже известных, задокументированных проблем, работа напрямую с пользователем \\
  \hline

Вторая линия  & Решение ранее неизвестных проблем \\
  \hline

Третья линия & Решение сложных и нетривиальных проблем \\
  \hline

Четвертая линия  & Решение архитектурных проблем инфраструктуры \\

  \hline
  
  \end{tabular}
%  \end{center}
\end{table}



\begin{figure} [h] 
  \center
  \includegraphics [scale=0.7] {ITSMTeamComposition}
  \caption{Диаграмма состава команд} 
  \label{img:ITSMTeamComposition}  
\end{figure}

Работа специалистов первой линии поддержки состоит из множества рутинных и простых задач. На рисунке \ref{img:EngineerTasks} показано соотношение разных типов проблем, встречающихся во время работы службы поддержки, в таблице \ref{IncidentDescription} приведена расшифровка типов. Данные подготовлены на основе анализа работы команд \icl.

\begin{figure} [h] 
  \center
  \includegraphics [scale=0.7] {EngineerTasks}
  \caption{Диаграмма соотношений типов проблем} 
  \label{img:EngineerTasks}  
\end{figure}

\begin{table} [htbp]
  \centering
  \parbox{15cm}{\caption{Категории инцидентов в области удаленной поддержки инфраструктуры}\label{IncidentDescription}}
%  \begin{center}
  \begin{tabular}{| p{7cm} | p{7cm} |}
 
  \hline
\textbf{Категория} & \textbf{Описание} \\
  \hline
Проблема с ПО	& Проблема при запуске ПО на компьютере. Решается переустановкой \\
  \hline
Проблемы во время работы  & Проблема с функционированием программного обеспечения\\
    \hline
Как сделать & Запрос на инструкцию по работе с тем или иным компонентом рабочей станции \\
      \hline
Проблема с оборудованием  & Неполадки на уровне оборудования \\
  \hline
Установить новое ПО       & Требование установки нового программного обеспечения \\
  \hline
Проблема с печатью        & Установка принтера в систему \\
    \hline
Нет доступа               & Нет доступа к общим ресурсам \\
  \hline
  \end{tabular}
%  \end{center}
\end{table}

Как показывают исследования, решение части задач может быть автоматизировано. Если это будет сделано,  специалисты получат дополнительное время для решения более сложных задач. \\

\textbf{Предпосылки развития изучаемой предметной области}. 
Основной тенденцией в развитии области удаленной поддержки инфраструктуры являются попытки удешевить и улучшить стоимость предоставления услуг \cite{OutsourceEff}. \par
Компании, работающие на этом рынке, вкладывают большие средства в автоматизацию. Кроме того, современное развитие науки и техники, точнее, вычислительных мощностей \cite{SuperComputer} позволяет провести автоматизацию даже самых наукоемких процессов. Дальнейшей перспективой развития области является замена человеческих специалистов автоматизированными системами. Разработки в этом направлении ведут многие компании, например, компания HP, которая имеет свою систему регистрации различных инцидентов и сейчас ведет работу над ее автоматизацией. В качестве некоторого сравнения можно провести параллель происходящего процесса с промышленной революцией XVIII–XIX веков (см., например, \cite{IndustrialRev}). \par
Исследования в области интеллектуальных систем повышения эффективности ИТ-службы предприятия широко ведутся лидерами отрасли: компанией HP и IBM, используя технологии обработки естественного языка, они создают интеллектуальные системы обработки запросов пользователя. Исследования также ведется и в университетах, например, важно отметить подход GATE \cite{GATE-1}, исследование и развитие которого активно ведется Университетом Шеффилда. В институте Чиная ведется исследование интеллектуальных систем обработки запросов пользователя в области телекоммуникаций \cite{CHIN-1}.  \par

\textbf{Методологии, используемые в области ИТ-аутсорсинга: ITIL и ITSM}. \par
В области ИТ-аутсорсинга есть несколько готовых стандартов ведения работ, одним из которых является библиотека ITIL. Этот стандарт описывает лучшие практики организации работ в области ИТ-аутсорсинга. Используемый в библиотеке подход соответствует стандартам ISO 9000 (ГОСТ Р ИСО 9000) \cite{ITIL1, ITIL2, ITIL3}.
Наличие стандартов диктует унифицированность как постановки проблем, так и алгоритмов решения, а также способствует возможности частичной или в некоторых случаях полной автоматизации решения проблем. \par

\textbf{Оценка стоимости работы специалиста при автоматизации}.\par
По данным аналитики портала SuperJob \cite{SuperJob}, в Казани средняя зарплата системного администратора с опытом работы в 2014 году составляла 30 – 35 тыс. руб. 2014 года (из расчета на 1 час с учетом 21 рабочего дня в месяце~--- 179--208 руб. в час. В соответствии с действующим российским законодательством \cite{FiscalCodecs} расходы компании на одного работника определяются по формуле
\[
L = R + R*(F_1 +F_2+F_3),
\]
где R~--- выплата человеку в час, F1~--- НДФЛ 13\%, F2~--- совокупность отчислений в ФБ (6\%), ПФР (14\%), ТФОМС (2\%), ФФОМС (1,1\%), ФСС (2,9\%), F3~--- налог на прибыль (20\%). Таким образом, расходы компании на сотрудника варьируются от 285 до 314 руб. в час, а за 8-ми часовой рабочий день – от 2280 до 2512 руб. Далее, аренда выделенного сервера (Xeon X3, 1.7 GHz, 8GB RAM, 256GB SSD) стоит 8 900 руб./мес. (см. \cite{TimeWeb}) (53 рубля за 1 час с учетом 8-ми часового рабочего дня). Но сервер может работать 24 часа в сутки за исключением простоев на обслуживание, которые обычно составляют не более 5\% времени. Итого: сервер работает 478,8 часов в месяц. С этой точки зрения эксплуатация сервера будет стоить 18,5 руб. в час. Один сервер в своем быстродействии может заменить несколько специалистов при решении соответствующих задач. Чтобы решение было экономически эффективным, необходимо, чтобы оно сокращало расходы как минимум на 30\% (по данным \icl). Грубый подсчет на основе стоимости часа и пропорции показывает, что работа специалиста~--- это 6\% работы сервера (без учета работы сервера параллельно над несколькими задачами). Таким образом, уровень разрешения инцидентов системой в 50\% выполнит требования по прибыли примерно 186\%.

\textbf{Общая характеристика диссертации} 
\newcommand{\actuality}{\underline{\textbf{Актуальность темы.}}}
\newcommand{\aim}{{\textbf{Целью}}}
\newcommand{\tasks}{{\textbf{задачи}}}
\newcommand{\scope}{{\textbf{Область исследования}}}
\newcommand{\subject}{{\textbf{Предметом исследования}}}
\newcommand{\methods}{{\textbf{Методы исследования}}}
\newcommand{\defpositions}{{\textbf{Основные положения, выносимые на~защиту:}}}
\newcommand{\novelty}{{\textbf{Научная новизна}}}
\newcommand{\influence}{{\textbf{Практическая значимость.}}}
\newcommand{\reliability}{{\textbf{Достоверность}}}
\newcommand{\probation}{{\textbf{Апробация работы.}}}
\newcommand{\contribution}{{\textbf{Личный вклад.}}}
\newcommand{\publications}{{\textbf{Публикации.}}}

{\actuality}
В настоящее время в области IT набрали большую популярность системы удаленной поддержки информационной инфраструктуры предприятия, так называемый «Аутсорсинг». Ввиду развития рынка компаниям становится невыгодно держать свой штат службы поддержки, и они отдают свою информационную инфраструктуру сторонней компании.
Ввиду возросшей популярности данного бизнеса и появлением большого количества игроков на рынке возникла большая конкурентность, которая потребовала увеличения эффективности и сокращения издержек, что в свою очередь привело к необходимости системного анализа области и выработке решению сложившихся проблем. В контексте решения этой проблемы рассматривается модель области и модель системы, которая увеличивает эффективность работы путем частичной (в некоторых случаях полной) автоматизации обработки инцидентов, начиная с разбора входящих инцидентов на естественном языке и заканчивая применением найденного решения. 
Главными требованиями к подобной системе являются:
\begin{enumerate}
  \item Обработка запросов на естественном языке
  \item Возможность обучения
  \item Общение с человеческим специалистом
  \item Проведение логических рассуждений: аналогия, дедукция, индукция
  \item Умение абстрагировать решение одной проблемы и, экстраполируя его, применить для других решений
  \item Способность ответить на запрос пользователя
\end{enumerate}

На данный момент многие компании ведут разработку подобных систем. Примером является набирающая популярность IBM Watson. Подобный класс систем также называется вопросно-ответными системами. \\
В данной работе была создана подобная система на основе исследования целевой области и построения ее модели.
 Акцент был сделан на создании мыслящей системы для решения широкого круга проблем. \\
{\aim} данной работы является исследование целевой области, создание ее модели, выработка проблем области, оценка подходов к решению проблем,  создание архитектуры и реализация базового прототипа программного комплекса обеспечивающего разбор и формализацию входного запроса пользователя и поиск решения данной проблемы.

Для~достижения поставленной цели необходимо было решить следующие {\tasks}:
\begin{enumerate}
  \item Провести теоретико-множественный и теоретико-информационный анализ сложных систем в области поддержки информационной инфраструктуры
  \item Вычислить технико-экономическую возможность автоматизации целевой области
  \item Создать модель целевой области
  \item Исследовать модели мышления и выбрать наиболее подходящую
  \item На основе выбранной модели мышления разработать модель проблемно-ориентированной системы управления, принятия решений и оптимизации технических объектов в области обслуживания информационной структуры предприятия
  \item Создать архитектуру приложения на основе модели
  \item Реализовать прототип на основе архитектуры
  \item Провести апробацию прототипа на тестовых данных
\end{enumerate}

\defpositions
\begin{enumerate}
  \item Теоретико-множественный и теоретико-информационный анализ сложных систем в области поддержки информационной инфраструктуры
  \item Модель проблемно-ориентированной системы управления, принятия решений и оптимизации технических объектов в области обслуживания IT, ее технико-эконономическое обоснование  
  \item Прототип программной реализации модели проблемно-ориентированной системы управления, принятия решений и оптимизации технических объектов в области обслуживания IT  
  \item Апробация системы на контрольных примерах и ее результаты
\end{enumerate}

\novelty
\begin{enumerate}
  \item Была создана модель проблемно-ориентированной системы управления, принятия решений и оптимизации технических объектов в области обслуживания информационной структуры предприятия на основе модели мышления
  \item Доказана применимость модели для других областей
  \item Была представлена новая модель данных для модели мышления и оригинальный способ ее хранения, обеспечивающий быстрый доступ
  \item Было выполнено оригинальное исследование моделей мышления в области обслуживания информационной структуры предприятия
  \item На основе модели была создана архитектура системы и ее прототип 
\end{enumerate}

\influence\ 
Система, разрабатываемая в рамках данной работы носит значимый практический характер. Идея работы зародилась из производственных проблем в IT отрасли, с которыми автор сталкивался каждый день. Только глубокое понимание проблем помогло выбрать правильное решение. Более подробное описание представлено в Главе 1.
\reliability\ полученных результатов обеспечивается результатами выполнения тестов на контрольных примерах. Результаты находятся в соответствии с результатами, полученными другими авторами, экспертными системами и специалистами. \\ 

\probation\
Основные результаты работы докладывались на:
\begin{itemize}
	\item RCDL-2014
	\item AINL-2013
	\item WCIT-2012
	\item AMSTA-2015
\end{itemize}

\contribution\ Автор принимал активное участие в исследовании целевой области, разработке архитектуры приложения, реализации прототипа, проработки теории, тестировании прототипа.

\publications\ Основные результаты по теме диссертации изложены в 6 печатных изданиях  \cite{Lobachevskii},\cite{WCIT-2012},\cite{RCDL-2014},\cite{AINL-2013},\cite{ISGZ},\cite{AMSTA-2015}, 
2 из которых изданы в журналах Scopus, 1 в журнале РИНЦ  \cite{ISGZ}, 
4 в тезисах докладов \cite{Lobachevskii},\cite{WCIT-2012},\cite{AINL-2013},\cite{ISGZ}, \cite{IJSE-1}.



 % Характеристика работы по структуре во введении и в автореферате не отличается (ГОСТ Р 7.0.11, пункты 5.3.1 и 9.2.1), потому её загружаем из одного и того же внешнего файла, предварительно задав форму выделения некоторым параметрам
%% регистрируем счётчики в системе totcounter
\regtotcounter{totalcount@figure}
\regtotcounter{totalcount@table}       % Если поставить в преамбуле то ошибка в числе таблиц
\regtotcounter{TotPages}               % Если поставить в преамбуле то ошибка в числе страниц

\textbf{Объем и структура работы.} Диссертация состоит из введения, четырех глав, заключения и пяти приложений. Полный объём диссертации составляет \formbytotal{TotPages}{страниц}{у}{ы}{} 
с~\formbytotal{totalcount@figure}{рисунк}{ом}{ами}{ами}
и~\formbytotal{totalcount@table}{таблиц}{ей}{ами}{ами}. Список литературы содержит  
\formbytotal{citenum}{наименован}{ие}{ия}{ий}.
\clearpage

