\chapter*{Словарь терминов} \label{Glossary}            % Заголовок
\addcontentsline{toc}{chapter}{Словарь терминов}  % Добавляем его в оглавление

\textbf{База Знаний}~--- База данных приложения, представленная в виде онтологии знаний. \\

\textbf{WayToThink}~--- Путь мышления. Основан на определении Марвина Мински \cite{EmotionMachine}. Класс объектов, которые модифицируют данные. \\

\textbf{Critic}~--- Основан на определении Марвина Мински \cite{EmotionMachine}. Класс объектов, которые выступают триггерами при наступление определенного события. \\

\textbf{ThinkingLifeCycle}~--- Основан на определении Марвина Мински \cite{EmotionMachine}. Класс объектов, которые выступают основными объектами для запуска в приложении~--- рабочими процессами. \\

\textbf{Selector}~--- Компонент, отвечающий за выборку данных из Базы Знаний. \\

\textbf{Instinctive}~--- Инстинктивный уровень. \\

\textbf{Learned}~--- Уровень обученных реакций. \\

\textbf{Deliberative}~--- Уровень рассуждений. \\

\textbf{Reflective}~--- Рефлексивный уровень. \\

\textbf{Self-Reflective Thinking	}~--- Саморефлексивный уровень. \\

\textbf{Self-Conscious Reflection}~--- Самосознательный уровень. \\

\textbf{ThinkingUnderstanding}~--- Система, созданная в рамках работы. Дословный перевод "Мышление-Понимание".\\  

\textbf{Вариант использования}~--- Термин из стандарта UML, который описывает возможные способы функционирования системы.\\  

\textbf{Диаграмма действий}~--- Термин из стандарта UML, который описывает последовательность действий пользователя.\\   
 
\clearpage