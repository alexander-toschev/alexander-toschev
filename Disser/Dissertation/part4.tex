\chapter{Экспериментальные исследования эффективности работы модели TU}


%================================
%================================
%================================
\section{Экспериментальные данные}
В качестве экспериментальных данных были взяты выгрузки проблем из информационных систем \icl. 
Для начального обучения в систему заложено две базовые концепции: Object~--- объект, который является базовой концепцией для всех объектов и Action~--- действие, которое является базовой концепцией для всех действий. 
В таблице \ref{Test data description} представлен список основных тренировочных данных.

\begin{longtable}{|p{12cm}|p{5cm}|}
 \caption[Описание экспериментальных данных]{Описание экспериментальных данных}\label{Test data description} \\ 
 \hline
 
 \multicolumn{1}{|c|}{\textbf{Входное предложение}} & \multicolumn{1}{c|}{\textbf{Описание}}  \\ \hline 
\endfirsthead
\multicolumn{2}{c}%
{{\bfseries \tablename\ \thetable{} -- продолжение}} \\
\hline \multicolumn{1}{|c|}{\textbf{Входное предложение}} &
\multicolumn{1}{c|}{\textbf{Описание}}  \\ \hline 
\endhead

\endfoot

\hline \hline
\endlastfoot
\hline
  Tense is kind of concept~(Время~--- это концепция).  & Обучающий запрос. Создает связь между концепцией Tense и Concept \\
  
  \hline
   Please install Firefox~(Установите Firefox).  & Запрос. Пользователь просит установить Firefox. Результатом должен быть найдено решение по установки Firefox \\
  \hline
  Browser is an object~(Браузер~--- это объект).   & Обучающий запрос. Создает связь между концепцией Browser и object \\
  \hline
  Firefox is a browser~(Firefox~--- это браузер).   & Обучающий запрос. Создает связь между концепцией Firefox и browser  \\
  \hline
  Install is an action~(Установить~--- это действие).    & Обучающий запрос. Создает связь между концепцией Install и action \\
  \hline
  User miss Internet Explorer 8~(У пользователя нет Internet Explorer 8).      & Запрос. Проблема с желаемым состоянием (DesiredState) \\
  \hline
  User needs document portal update~(Пользователю требуется обновление документов).    & Запрос. Проблема с желаемым состоянием \\
  \hline
 Add new alias Host name on host that alias is wanted to: hrportal.lalala.biz IP address on host that alias is wanted to: 322.223.333.22 Wanted Alias:    webadviser.lalala.net~(Добавьте, пожалуйста, новую ссылку на hrportal.lalala.biz через 322.223.333.22).    & Запрос. Сложная проблема  \\ 
  \hline
   Outlook Web Access (CCC)~--- 403~--- Forbidden: Access is denied~(Нет доступа к Outlook Web Access (CCC)). Сложная проблема \\ 
  \hline
  PP2C~--- Cisco IP communicator. Please see if you can fix the problem with the ip phone, it's stuck on configuring ip + sometimes Server error rejected: Security etc~(PP2C~--- коммуникатор Cisco IP. Пожалуйста, помогите исправить проблему с ИП-телефоном, он застревает во время конфигурирования и иногда показывает ошибку «Безопасность»).     & Запрос. Сложная проблема \\ 
   \hline
  \end{longtable}

Полный список информации об экспериментальных данных представлен в приложении \ref{AppendixE}.




%================================
%================================
%================================
\section{Оценка эффективности}
Для верификации экспертной системы поддержки принятия решений TU была выбрана область поддержки информационной инфраструктуры предприятия, которая в рамках работы исследована и смоделирована в Главе \ref{chapt1}. 
Для доказательства жизнеспособности решения производилась верификация в 2 этапа:
\begin{itemize}
	\item Этап 1. Разбор входящего запроса на естественном языке и вычленение концепции;
	\item Этап 2. Обработка по разработанной архитектуре и реализации модели мышления.  
\end{itemize} \par
Для Этапа 1 использовалась отфильтрованная выгрузка инцидентов. Были выявлены уникальные инциденты~--- 1000. На данном этапе удалось добиться качества разбора на уровне 67\%. Успешным считался разбор, когда правильно были определены концепции, например, существительное определялось как существительное, глагол как глагол. \par
Для Этапа 2 использовалась часть инцидентов, которая представлена в предыдущий главе. На них запускался программный комплекс и анализировались результаты. Удалось добиться 95\% успешных инцидентов. Успешным считался инцидент, который был успешно сопоставлен концепциям в базе знаний. \par
Результатом успешной обработки инцидента считалось найденное решение, если же решения не было, то проверялось правильное понимание системой всех концепций, так как решения не было в базе знаний. Запуск работы системы производился при помощи автоматизированных тестов. Проверка данных также осуществлялась при помощи этой технологии. Система также может функционировать в режиме диалога и в консольном варианте, в этом режиме видно взаимодействие с пользователем. 



%================================
%================================
%================================
\section{Результаты экспериментов}

Система показала свою жизнеспособность на модельных данных. Были проведены тесты в сравнении с работой человеческого специалиста. Был выбран контрольный список инцидентов. Сравнивался поиск решения для инцидентов. Основное время при опросе специалиста тратилось на коммуникацию. В Таблице \ref{HumanComparison} приведены результаты сравнения. Тесты были выполнены на компьютере Intel Core i7 1700 MHz, 8GB RAM, 256 GB SSD, FreeBSD. 
\begin{longtable}{|p{12cm}|p{2cm}|p{2cm}|}
 \caption[Результаты сравнения с работой специалиста]{Результаты сравнения с работой специалиста}\label{HumanComparison} \\ 
 \hline
 
 \multicolumn{1}{|c|}{\textbf{Инцидент}} & \multicolumn{1}{c|}{\textbf{TSS1 (.мс)}} & \multicolumn{1}{c|}{\textbf{TU (.мс)}}  \\ \hline 
\endfirsthead
\multicolumn{2}{c}%
{{\bfseries \tablename\ \thetable{} -- продолжение}} \\
\hline \multicolumn{1}{|c|}{\textbf{Инцидент}} & \multicolumn{1}{c|}{\textbf{TSS1 (.мс)}} & \multicolumn{1}{c|}{\textbf{TU (.мс)}}  \\ \hline 
\endhead

\endfoot

\hline \hline
\endlastfoot
\hline
  Tense is kind of concept~(Время~--- это концепция) & 15000 & 385 \\
  
  \hline
  Please install Firefox~(Установите Firefox)   & 9000 & 859 \\
  \hline
  Browser is an object~(Браузер~--- это объект)   & 20000 & 400 \\
  \hline
  Firefox is a browser~(Firefox~--- это браузер)   & 5000 & 659  \\
  \hline
  Install is an action~(Установить~--- это действие)   & 8000 & 486 \\
  \hline
  User miss Internet Explorer 8~(У пользователя нет Internet Explorer 8)     & 10000 & 10589 \\
  \hline
  User needs document portal update~(Пользователю требуется обновление документов)    & 15000 & 16543 \\
  \hline
  Add new alias Host name on host that alias is wanted to: hrportal.lalala.biz IP address on host that alias is wanted to: 322.223.333.22 Wanted Alias:    webadviser.lalala.net~(Добавьте, пожалуйста, новую ссылку на hrportal.lalala.biz через 322.223.333.22)    & 10000 & 18432  \\ 
  \hline
  Outlook Web Access (CCC)~--- 403~--- Forbidden: Access is denied~(Нет доступа к Outlook Web Access (CCC)) & 15000 & 10342\\ 
  \hline
  PP2C~--- Cisco IP communicator. Please see if you can fix the problem with the ip phone, it's stuck on configuring ip + sometimes Server error rejected: Security etc~(PP2C~--- коммуникатор Cisco IP. Пожалуйста, помогите исправить проблему с ИП-телефоном, он застревает во время конфигурирования и иногда показывает ошибку «Безопасность»)  & 13000 & 12343 \\ 
   
  \end{longtable}


Основной проблемой для системы составляют инциденты с большой неоднозначностью, например, "I should have Internet Explorer, but Firefox was installed". Здесь непонятно, нужен ли пользователю браузер Firefox или нет. В этом случае система должна выявить проблему о необходимости пользователю Internet Explorer. \par 
Другой пример, который трудно однозначно решить, используя классические подходы: I install Internet Explorer previously, but i need Chrome. Здесь есть следующие наборы концепций: i, install, Internet Explorer; i, need, Chrome. Используя регулярные выражения, однозначно решить не удастся, но, используя интеллектуальное решение, эту проблему решить можно. В рамках TU сработает более вероятный Critic, который определит проблему "need Chrome"\,, базируясь на наличии концепции "previously". 
В таблице  \ref{ProblemClassComparison} приведены результаты работы системы в разрезе категорий инцидентов. \\

\begin{longtable}{|p{12cm}|p{5cm}|}
 \caption[Описание экспериментальных данных]{Описание экспериментальных данных}\label{ProblemClassComparison} \\ 
 \hline
 
 \multicolumn{1}{|c|}{\textbf{Класс проблемы}} & \multicolumn{1}{c|}{\textbf{
 \% успешных}}  \\ \hline 
\endfirsthead
\multicolumn{2}{c}%
{{\bfseries \tablename\ \thetable{} -- продолжение}} \\
\hline \multicolumn{1}{|c|}{\textbf{Входное предложение}} &
\multicolumn{1}{c|}{\textbf{Описание}}  \\ \hline 
\endhead

\endfoot

\hline \hline
\endlastfoot
\hline

Проблема с ПО    & 64\% \\
 \hline Проблемы во время работы  &  10\% \\
  \hline Как сделать & 10\% \\
   \hline
Проблема с оборудованием  & 0\% \\
 \hline
Установить новое ПО       & 100\% \\
 \hline Проблема с печатью        & 80\% \\
  \hline Нет доступа               & 100\% \\
  \hline
  \end{longtable}
  


  
Показатели, приведенные в главе 1, $\alpha$ --- доля заявок, для которых время в очереди превышает $max(T_q)$;       
$\mu$ --- величина, обратная среднему времени нахождения заявки у агента;
$n$ --- число агентов;
$T_q$ --- время нахождение заявки в очереди в часах;
$SLA$ --- уровень обслуживания (1-$\alpha$), доля заявок, для которых время в очереди не превышает $max(T_q)$. $T_p$ --- время удовлетворения заявки;
 $\alpha_n$ --- количество заявок;
 $T_{qp}=T_q+T_p$ --- время прохождения заявки через систему,
 приобрели следующие значения $T_qp=32,9$ при $n=8$; $SLA=0,96$; $\alpha=0,04$;  $\alpha_n=2920$.  Предыдущие значение было $T_{qp}=47,9$ при $n=6$; $SLA=0,82$; $\alpha=0,18$;  $\alpha_n=2920$, но увеличение $n$ не привело к увеличению задействованных в работе специалистов. Согласно таблице \ref{ProblemClassComparison} средний процент обработанных заявок составил 52\%, что составляет более половины всех заявок и требуемых 30\% (см. Главу 1).

  
\section{Выводы по главе 4}
В главе были рассмотрены экспериментальные данные, которые были использованы для верификации системы, также дается обоснование, почему были выбраны именно эти данные. На основе экспериментов была посчитана скорость работы системы в сравнении со специалистом технической поддержки. Были приведены сложные для решения примеры входных запросов пользователя и дан их разбор. 


\clearpage