%%% Реализация библиографии встроенными средствами посредством движка bibtex8 %%%

%%% Пакеты %%%
\usepackage{cite}                                   % Красивые ссылки на литературу


%%% Стили %%% ./BibTeX-Styles/utf8gost71u
% Библиографические записи в списке литературы оформляют согласно ГОСТ 7.1.
% GOST 7.0.5-2008 or GOST 7.1-2003
% ГОСТ Р 7.0.11-2011 С 1 сентября 2012  http://www.rgatu.ru/index.php?option=com_content&view=category&layout=blog&id=477&Itemid=171
% Стандарты http://www.bookchamber.ru/standarts.html
% ГОСТ ГОСТ 7.1–2003
\bibliographystyle{ugost2003}    % Оформляем библиографию по ГОСТ 7.1 (ГОСТ Р 7.0.11-2011, 5.6.7)

\makeatletter
\renewcommand{\@biblabel}[1]{#1.}   % Заменяем библиографию с квадратных скобок на точку
\makeatother
%% Управление отступами между записями
%% требует etoolbox 
%% http://tex.stackexchange.com/a/105642
%\patchcmd\thebibliography
% {\labelsep}
% {\labelsep\itemsep=5pt\parsep=0pt\relax}
% {}
% {\typeout{Couldn't patch the command}}

%%% Цитирование %%%
\renewcommand\citepunct{;\penalty\citepunctpenalty%
    \hskip.13emplus.1emminus.1em\relax}                % Разделение ; при перечислении ссылок (ГОСТ Р 7.0.5-2008)


%%% Создание команд для вывода списка литературы %%%
\newcommand*{\insertbibliofull}{
\bibliography{./biblio/othercites,./biblio/authorpapersVAK,./biblio/authorpapers,./biblio/authorconferences}         % Подключаем BibTeX-базы % После запятых не должно быть лишних пробелов — он "думает", что это тоже имя пути
}

\newcommand*{\insertbiblioauthor}{
\bibliography{./biblio/authorpapersVAK,./biblio/authorpapers,./biblio/authorconferences}         % Подключаем BibTeX-базы % После запятых не должно быть лишних пробелов — он "думает", что это тоже имя пути
}

\newcommand*{\insertbiblioother}{
\bibliography{./biblio/othercites}         % Подключаем BibTeX-базы
}

\newcommand*{\insertbiblioscopus}{
\bibliography{./biblio/biblio-scopus}         % Подключаем BibTeX-базы
}
\newcommand*{\insertbiblioall}{
\bibliography{./biblio/biblio}         % Подключаем BibTeX-базы
}


%% Счётчик использованных ссылок на литературу, обрабатывающий с учётом неоднократных ссылок
%% Требуется дважды компилировать, поскольку ему нужно считать актуальный внешний файл со списком литературы
\newtotcounter{citenum}
\def\oldcite{}
\let\oldcite=\bibcite
\def\bibcite{\stepcounter{citenum}\oldcite}
