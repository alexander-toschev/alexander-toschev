
{\aim} данной работы является разработка интеллектуальной системы повышения эффективности деятельности IT-службы предприятия.
{\scope} данной работы~--- это разработка методов и алгоритмов решения задач системного анализа, оптимизации, управления, принятия решений и обработки информации в IT-отрасли.
{\subject}  является процесс регистрации и устранения проблемных ситуаций, возникающих в IT-инфраструктуре предприятия.

Для достижения поставленной цели необходимо было решить следующие {\tasks}:
\begin{enumerate}
  \item Провести теоретико-множественный и теоретико-информационный анализ сложных систем в области поддержки информационной инфраструктуры
  \item Создать модель целевой области
  \item Исследовать модели мышления и выбрать наиболее подходящую
  \item На основе выбранной модели мышления разработать модель проблемно-ориентированной системы управления, принятия решений и оптимизации технических объектов в области обслуживания информационной структуры предприятия
  \item Создать архитектуру приложения на основе модели
  \item Реализовать прототип на основе архитектуры
  \item Провести апробацию прототипа на тестовых данных
\end{enumerate}

\defpositions
\begin{enumerate}
  \item Теоретико-множественный и теоретико-информационный анализ сложных систем в области поддержки информационной инфраструктуры
  \item Модель проблемно-ориентированной системы управления, принятия решений и оптимизации технических объектов в области обслуживания IT 
  \item Прототип программной реализации модели проблемно-ориентированной системы управления, принятия решений и оптимизации технических объектов в области обслуживания IT  
  \item Апробация системы на контрольных примерах и ее результаты
\end{enumerate}

\novelty
\begin{enumerate}
  \item Была создана модель проблемно-ориентированной системы управления, принятия решений в области обслуживания информационной структуры предприятия на основе модели мышления
  \item Была представлена новая модель данных для модели мышления и оригинальный способ ее хранения, эффективный по сравнению с другими базами данных
  \item Было выполнено оригинальное исследование моделей мышления в области обслуживания информационной структуры предприятия
  \item На основе модели была создана архитектура системы и ее прототип 
\end{enumerate}

\influence\ 
Система, разрабатываемая в рамках данной работы носит значимый практический характер. Идея работы зародилась из производственных проблем в IT отрасли, с которыми автор сталкивался каждый день. Только глубокое понимание области помогло выбрать правильное решение. 
\reliability\ научных исследований и практических рекомендаций
базируется на корректной постановке общих и частных, поставленных выше,
задач, использовании известных фундаментальных теоретических положений
технической кибернетики, достаточном объёме статистического моделирования
и экспериментальном материале исходных данных для численных оценок
достижимых качественных показателей. \par 
В работе были проведены исследования согласно паспорту специальности 05.13.01, сопоставление приведено в Таблице \ref{ResearchDescription}.

\begin{longtable}{|p{7cm}|p{9cm}|}
 \caption[Сопоставление направлений исследования специальности 05.13.01 и исследований, проведенных в работе]{Сопоставление направлений исследования специальности 05.13.01 и исследований, проведенных в работе}\label{ResearchDescription} \\ 
 \hline
 
 \multicolumn{1}{|c|}{\textbf{Направление исследования}} & \multicolumn{1}{c|}{\textbf{Результат работы}}  \\ \hline 
\endfirsthead
\multicolumn{2}{c}%
{{\bfseries \tablename\ \thetable{} -- продолжение}} \\
\hline \multicolumn{1}{|c|}{\textbf{Направление исследования}} &
\multicolumn{1}{c|}{\textbf{Результат работы}}  \\ \hline 
\endhead

\hline \multicolumn{2}{|r|}{{Продолжение следует}} \\ \hline
\endfoot

\hline \hline
\endlastfoot
\hline
   Разработка критериев и моделей описания и оценки эффективности решения задач системного анализа, оптимизации, управления, принятия решений и обработки информации & В рамках работы была разработана модель системы принятия решения и обработки информации в области решения запросов пользователя на естественном языке. \\
   \hline
   Разработка проблемно-ориентированных систем управления, принятия решений и оптимизации технических объектов & По модели, разработанной в предыдущем пункте был разработан прототип системы принятия решения Thinking-Understanding, который был испытан на модельных данных.\\
   \hline
   Методы получения, анализа и обработки экспертной информации & В рамках системы TU был разработан метод обработки экспертной информации - обучение при помощи модели мышления TU, основанной на принципах модели 6-ти Марвина Мински. \\
   \hline
   Разработка специального математического и алгоритмического обеспечения систем анализа, оптимизации, управления, принятия решений и обработки информации & В рамках разработки системы TU были созданы специальные алгоритмы для анализа запросов пользователя и принятия решений.\\
  \hline 
  Теоретико-множественный и теоретико-информационный анализ сложных систем & В рамках работы был проведен комплексный анализ области поддержки программного обеспечения, с помощью которого была построена система данной области и выделены участки для оптимизации принятия решений.\\
  \hline
  Методы и алгоритмы интеллектуальной поддержки при принятии управленческих решений в технических системах & Система, разработанная в рамках данной работы в включает в себя инновационные методы и алгоритмы поддержки принятия решений, использующих в своей основе модель мышления на базе модели мышления Человека, описанной в книге Марвина Мински. \\ 
  \hline
  Визуализация, трансформация и анализ информации на основе компьютерных методов обработки информации & Представлена наглядная визуализация данных по системному анализу области удаленной поддержки инфраструктуры. \\
  \hline	
\end{longtable}


\probation\
Основные результаты работы докладывались на:
\begin{itemize}
	\item Конференция Лобачевского - 2011
	\item WCIT-2012
	\item AINL-2013
	\item RCDL-2014
	\item AMSTA-2015
\end{itemize}
Апробация работы проводилась на выгрузка инцидентов из систем регистрации ОАО "АйСиЭл КПО-ВС". Система показала хорошие результаты обработки данной информации.
\contribution\ Автор принимал активное участие в исследовании целевой области, разработке архитектуры приложения, реализации прототипа, проработки теории, тестировании прототипа.

\publications\ Основные результаты по теме диссертации изложены в 9 печатных изданиях  \cite{Lobachevskii},\cite{WCIT-2012},\cite{AINL-2013},\cite{ISGZ}, \cite{IJSE-1}, \cite{IJSE-2}, \cite{RCDL-2014}, \cite{AMSTA-2015}, \cite{VAK-1},
1 из которых изданы в журналах Scopus \cite{RCDL-2014}, 1 в журнале Web of Science \cite{AMSTA-2015}, 1 в журнале ВАК \cite{VAK-1}, 1 в журнале РИНЦ  \cite{ISGZ}, 
4 в тезисах докладов \cite{Lobachevskii},\cite{WCIT-2012},\cite{AINL-2013},\cite{ISGZ}.



