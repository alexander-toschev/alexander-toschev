
{\aim} работы является разработка интеллектуальной системы повышения эффективности деятельности ИТ-службы предприятия. \par
{\scope}~--- разработка методов и алгоритмов решения задач системного анализа, оптимизации, управления, принятия решений и обработки информации в ИТ-отрасли.\par
{\subject}  является процесс регистрации и устранения проблемных ситуаций, возникающих в ИТ-инфраструктуре предприятия.\par
{\methods}~--- экспериментальные методы: метод наблюдений, проведение экспериментов; теоретические методы: метод идеализации, метод формализации; специальные методы: системное моделирование, системный анализ.\par 
Для достижения поставленной цели необходимо было решить следующие {\tasks}:
\begin{enumerate}
  \item Провести теоретико-множественный и теоретико-информационный анализ сложных систем в области поддержки информационной инфраструктуры;
  \item Разработать модель проблемно-ориентированной системы управления, принятия решений и оптимизации процесса принятия, анализа и обработки запросов пользователя в области обслуживания информационной структуры предприятия;
  \item На основе модели разработать архитектуру и создать прототип интеллектуальной вопросно-ответной системы повышения эффективности деятельности ИТ-службы предприятия;
  \item Провести апробацию прототипа на тестовых данных.
\end{enumerate}

\defpositions
\begin{enumerate}
  \item Теоретико-множественный и теоретико-информационный анализ сложных систем в области поддержки информационной инфраструктуры;
  \item Построенная модель проблемно-ориентированной системы управления, принятия решений и оптимизации технических объектов в области обслуживания информационной инфраструктуры;
  \item Созданный прототип программной реализации модели проблемно-ориентированной системы управления, принятия решений и оптимизации обработки запросов пользователя в области обслуживания информационной инфраструктуры;
  \item Результаты апробации прототипа проблемно-ориентированной системы управления, принятия решений и оптимизации деятельности на контрольных примерах и анализ ее результатов.
\end{enumerate}

\novelty\ проведенного исследования состоит в следующем:
\begin{enumerate}
  \item Создана модель проблемно-ориентированной системы управления, принятия решений в области обслуживания информационной структуры предприятия на основе модели мышления;
  \item Представлены новая модель данных для модели мышления и оригинальный способ хранения для этой модели, эффективный по сравнению с другими базами данных;
  \item Выполнено оригинальное исследование моделей мышления применительно к области обслуживания информационной структуры предприятия;
  \item На основе модели мышления Мински созданы архитектура системы обслуживания информационной структуры предприятия и программный прототип этой системы.
\end{enumerate}

\influence\ 
Система, разработанная в рамках данной диссертации носит значимый практический характер. Идея работы зародилась под влиянием производственных проблем в ИТ-отрасли, с которыми автор сталкивался каждый день в процессе разрешения различных инцидентов, возникающих в деятельности службы технической поддержки \icl~--- одном из крупнейших системообразующих предприятий ИТ-области Республике Татарстан. Поэтому было необходимо выработать глубокое понимание конкретной предметной области, чтобы выбрать приемлемое решение, получившее практическое применение в работе на проекте поддержки крупной сети продуктовых магазинов. \par
\reliability\ научных исследований и практических рекомендаций
базируется на корректной постановке общих и частных рассматриваемых задач,  использовании известных фундаментальных теоретических положений системного анализа, достаточном объёме данных, использованных при статистическом моделировании, и широком экспериментальном материале, использованном для численных оценок достижимых качественных показателей. \par 
Исследования, проведенные в диссертации, соответствуют паспорту специальности 05.13.01~--- Системный анализ, управление и обработка информации, сопоставление приведено в таблице \ref{ResearchDescription}.

\begin{longtable}{|p{7cm}|p{9cm}|}
 \caption[Сопоставление направлений исследований в рамках специальности 05.13.01 и исследований, проведенных в диссертации]{Сопоставление направлений исследований в рамках специальности 05.13.01 и исследований, проведенных в диссертации}\label{ResearchDescription} \\ 
 \hline
 
 \multicolumn{1}{|c|}{\textbf{Направление исследования}} & \multicolumn{1}{c|}{\textbf{Результат работы}}  \\ \hline 
\endfirsthead
\multicolumn{2}{c}%
{{\bfseries \tablename\ \thetable{} -- продолжение}} \\
\hline \multicolumn{1}{|c|}{\textbf{Направление исследования}} &
\multicolumn{1}{c|}{\textbf{Результат работы}}  \\ \hline 
\endhead
\endfoot

\hline \hline
\endlastfoot
\hline
   Разработка критериев и моделей описания и оценки эффективности решения задач системного анализа, оптимизации, управления, принятия решений и обработки информации & Разработана модель системы принятия решения и обработки информации в области решения запросов пользователя на естественном языке. \\
   \hline
   Разработка проблемно-ориентированных систем управления, принятия решений и оптимизации технических объектов & Разработан прототип системы принятия решения Thinking Understanding (TU), который был испытан на модельных данных.\\
   \hline
   Методы получения, анализа и обработки экспертной информации & Разработан метод обработки экспертной информации c возможностью обучения при помощи модели мышления TU. \\
   \hline
   Разработка специального математического и алгоритмического обеспечения систем анализа, оптимизации, управления, принятия решений и обработки информации & Созданы специальные алгоритмы для анализа запросов пользователя и принятия решений.\\
  \hline 
  Теоретико-множественный и теоретико-информационный анализ сложных систем & Проведен комплексный анализ области поддержки программного обеспечения, с помощью которого была построена система данной области и выделены участки для оптимизации принятия решений.\\
\end{longtable}


\probation\
 Основные результаты диссертационной работы докладывались на следующих конференциях:
\begin{itemize}
	\item Десятая молодежная научная школа-конференция \quoted{Лобачевские чтения~---2011}. Казань, 31 октября~--4 ноября 2011;
	\item "3rd World Conference on Information Technology (WCIT-2012)". Barcelona, 14~--16 November 2012, Spain; 
	\item «Искусственный интеллект и естественный язык (AINL-2013)». Санкт-Петербург, 17~--18 мая 2013;
	\item «VI Международная научно-практическая конференция Электронная Казань 2014». Казань, 22~--24 апреля 2014;
	\item XVI Всероссийская научная конференция «Электронные библиотеки: перспективные методы и технологии, электронные коллекции (RCDL-2014)». Дубна, 13~--16 октября 2014;
	\item "All-Kazan Software Engineering Seminar (AKSES-2015)". Kazan, 9 April 2015;
	\item "Agents and multi-agent systems: technologies and applications (AMSTA-2015)". Sorento, 17~--19 June 2015, Italy.
\end{itemize} \par
Практическая апробация результатов работы проводилась на выгрузке инцидентов из системы регистрации запросов службы технической поддержки ИТ-инфраструктуры \icl. Созданная система показала требуемые результаты (процент успешно обработанных запросов более чем 30\%) обработки данной информации. \par
\contribution\ Автор исследовал целевую область: проводил анализ запросов пользователей и классифицировал их; строил модель целевой области и искал возможности оптимизации. Совместно с Талановым Максимом Олеговичем создавал базовую архитектуру системы. Автор разрабатывал компоненты системы, проводил испытание системы на экспериментальных данных и отлаживал работу системы. \par
\publications\ Основные результаты по теме диссертации изложены в 9 печатных изданиях  \cite{Lobachevskii, WCIT-2012, AINL-2013, ISGZ, IJSE-1, IJSE-2, RCDL-2014, AMSTA-2015, VAK-1}, из которых статьи \cite{RCDL-2014, AMSTA-2015} проиндексированы в БД Scopus, статья \cite{AMSTA-2015} проиндексирована в БД Web Of Science, работа \cite{VAK-1} опубликована в журнале из списка ВАК, статья  \cite{ISGZ} проиндексирована в БД РИНЦ, работы \cite{Lobachevskii},\cite{WCIT-2012, AINL-2013, ISGZ} опубликованы в материалах международных и всероссийских конференций.



