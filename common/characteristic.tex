{\actuality}
В настоящее время в области IT набрало большую популярность системы удаленной поддержки информационной инфраструктуры, так называемый «Аутсорсинг». Ввиду развития рынка компаниям становится невыгодно держать свой штат службы поддержки, и они отдают свою инфраструктуру сторонней компании.
Ввиду возросшей интенсивности данного бизнеса возникла потребность автоматизации работы. В данном контексте рассматривается автоматизация обработки инцидентов, начиная с разбора инцидентов на естественном языке и заканчивая поиском решения и применением решения.
Главными требованиями к системе являются:
\begin{enumerate}
  \item Обработка запросов на естественном языке
  \item Возможность обучения
  \item Общение со специалистом
  \item Проведение логических рассуждений: аналогия, дедукция, индукция
  \item Умения абстрагировать решение и экстраполировать его на другие решения
  \item Способность решить запрос пользователя
\end{enumerate}

На данный момент многие компании ведут разработку подобных систем. Примером такой системы является набирающая популярность система IBM Watson. Подобный класс систем также называют вопросно-ответными системами, например, Wolfram Alpha.
В данной работе была исследована целевая область и построена ее модель. 
В данной работе был сделан акцент на попытку создания мыслящий системы на основе модели мышления Марвина Мински для решения широкого круга проблем, а не специфичных. \\
{\aim} данной работы является исследование целевой области, создание ее модели, выработка проблем области, оценка подходов к решению проблем,  создание архитектуры и реализация базового прототипа программного комплекса обеспечивающего разбор и формализацию входного запроса пользователя и поиск решения данной проблемы.

Для~достижения поставленной цели необходимо было решить следующие {\tasks}:
\begin{enumerate}
  \item Провести теоретико-множественный и теоретико-информационный анализ сложных систем в области поддержки информационной инфраструктуры
  \item Вычислить возможность автоматизации целевой области
  \item Создать модель целевой области
  \item Исследовать модель мышления Марвина Мински
  \item На основе модели мышления Мински разработать модель проблемно-ориентированной системы управления, принятия решений и оптимизации технических объектов в области обслуживания IT  
  \item Создать архитектуру приложения на основе модели
  \item Реализовать прототип на основе архитектуры
  \item Провести апробацию прототипа на тестовых данных
\end{enumerate}

\defpositions
\begin{enumerate}
  \item Теоретико-множественный и теоретико-информационный анализ сложных систем в области поддержки информационной инфраструктуры
  \item Модель проблемно-ориентированной системы управления, принятия решений и оптимизации технических объектов в области обслуживания IT  
  \item Прототип программной реализации модели проблемно-ориентированной системы управления, принятия решений и оптимизации технических объектов в области обслуживания IT  
  \item Апробация системы на контрольных примерах
\end{enumerate}

\novelty
\begin{enumerate}
  \item Была создана модель проблемно-ориентированной системы управления, принятия решений и оптимизации технических объектов в области обслуживания IT на основе модели мышления Марвина Мински
  \item Была представлена новая модель данных для модели мышления и оригинальный способ хранения 
  \item Было выполнено оригинальное исследование модели мышления 
\end{enumerate}

\influence\ 
Система, разрабатываемая в рамках данной работы носит значимый практический характер. Идея работы зародилась из производственных проблем в IT отрасли, с которыми автор сталкивался каждый день. Только глубокое понимание проблем помогло выбрать правильное решение. Более подробное описание представлено в Главе 1.
\reliability\ полученных результатов обеспечивается результатами выполнения тестов на контрольных примерах. Результаты находятся в соответствии с результатами, полученными другими авторами, экспертными системами и специалистами. \\ 

\probation\
Основные результаты работы докладывались на:
\begin{itemize}
	\item RCDL-2014
	\item AINL-2013
	\item WCIT-2012
	\item AMSTA-2015
\end{itemize}

\contribution\ Автор принимал активное участие в исследовании целевой области, разработке архитектуры приложения, реализации прототипа, проработки теории, тестировании.

\publications\ Основные результаты по теме диссертации изложены в 6 печатных изданиях  \cite{Lobachevskii},\cite{WCIT-2012},\cite{RCDL-2014},\cite{AINL-2013},\cite{ISGZ},\cite{AMSTA-2015}, 
2 из которых изданы в журналах Scopus, 1 в журнале РИНЦ  \cite{ISGZ},
\cite{RCDL-2014}, \cite{AMSTA-2015} 
3 в тезисах докладов \cite{Lobachevskii},\cite{WCIT-2012},\cite{AINL-2013},\cite{ISGZ}.



