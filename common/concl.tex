%% Согласно ГОСТ Р 7.0.11-2011:
%% 5.3.3 В заключении диссертации излагают итоги выполненного исследования, рекомендации, перспективы дальнейшей разработки темы.
%% 9.2.3 В заключении автореферата диссертации излагают итоги данного исследования, рекомендации и перспективы дальнейшей разработки темы.

В работе были выполнены следующие задачи и достигнуты следующие результаты.
\begin{enumerate}
  \item Была создана модель проблемно-ориентированной системы управления, принятия решений в области обслуживания информационной структуры предприятия на основе модели мышления
  \item Была представлена новая модель данных для модели мышления и оригинальный способ ее хранения, эффективный по сравнению с другими базами данных
  \item Было выполнено оригинальное исследование моделей мышления в области обслуживания информационной структуры предприятия
  \item На основе модели была создана архитектура системы и ее прототип 
  \item Были созданы специальные алгоритмы для анализа запросов пользователя и принятия решений. 
  \item Система, разработанная в рамках данной работы, включает в себя инновационные методы и алгоритмы поддержки принятия решений, использующих в своей основе модель мышления на базе модели мышления Человека, описанной в книге Марвина Мински. 
  \item Была представлена наглядная визуализация структуры области удаленной поддержки инфраструктуры
\end{enumerate}

Представленная в данной работе модель мышления, ее архитектура и реализация является уникальной в своем роде. На момент написания это была единственная реализация модели мышления Марвина Мински. \\
Разработанная в рамках работы системы не является узкоспециализированной. Она также подходит для других областей, где требуется поддержка принятия решений. Например, при постановке медицинского диагноза, чтобы отбросить ложные диагнозы. \\
Например, систему можно обучить органам человека и их взаимосвязи. Далее можно обучить каким заболеваниям подвержен тот или иной орган. Далее к каждому заболеванию добавить симптом. После этого можно делать запрос с симптомами и система выдаст список вероятных заболеваний со способами их лечения. \\
В области диагностики проблем в машиностроении. Обучить систему узлам автомобиля, проблемам с ними связанными, признаками этих проблем и способами их устранения. 



