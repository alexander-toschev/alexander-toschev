%% Согласно ГОСТ Р 7.0.11-2011:
%% 5.3.3 В заключении диссертации излагают итоги выполненного исследования, рекомендации, перспективы дальнейшей разработки темы.
%% 9.2.3 В заключении автореферата диссертации излагают итоги данного исследования, рекомендации и перспективы дальнейшей разработки темы.

В работе были выполнены следующие задачи и достигнуты следующие результаты.
\begin{enumerate}
  \item На основе анализа предметной области (поддержка информационной структуры предприятия) была выявлена потребность и возможность в автоматизации. Была построена модель предметной области. На основе модели предметной области, модели Марвина Мински была разработана модель проблемно-ориентированной системы принятия решений в области поддержки информационной структуры предприятия.  
  \item Испытания комплекса на модельных данных показали работоспособность модели и архитектуры.  
  \item Для выполнения поставленных задач был создан программный комплекс обработки, решения инцидентов и обучения на естественном языке. 
  \item Программный комплекс был протестирован на контрольных примерах
\end{enumerate}

Представленная в данной работе модель мышления, ее архитектура и реализация является уникальной в своем роде. На момент написания это была единственная реализация модели мышления Марвина Мински. \\
Разработанная в рамках работы системы не является узко-специализированной. Она также подходит для других областей, где требуется поддержка принятия решений. Например, при постановке медицинского диагноза, чтобы отбросить ложные диагнозы. \\
Например, систему можно обучить органам человека и их взаимосвязи. Далее можно обучить каким заболеваниям подвержен тот или иной орган. Далее к каждому заболеванию добавить симптом. После этого можно делать запрос с симптомами и система выдаст список вероятных заболеваний с их вероятностью и способами их лечения. \\
В области диагностики проблем в машиностроении. Обучить систему узлам автомобиля, проблемам с ними связанными, признаками этих проблем и способами их устранения. 



